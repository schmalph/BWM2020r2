\documentclass{article}


\usepackage[ngerman]{babel}
\usepackage[utf8]{inputenc}
\usepackage[T1]{fontenc}
\usepackage{amsmath}
\usepackage{amssymb}
\usepackage{amsthm}
\usepackage{lastpage}
\usepackage[top = 1cm, bottom = 1cm, right = 2cm, left = 6.1cm, includeheadfoot]{geometry}
\usepackage{stmaryrd}
\usepackage{mathtools}
\usepackage{gensymb}
\usepackage{blindtext}
\usepackage{fancyhdr}

\pagestyle{fancy}
\fancyhead{}
\fancyhead[R]{Philipp Pascal Schmale}
\fancyhead[L]{\leftmark}
\fancyfoot{}
\fancyfoot[R]{Seite \thepage}
\newtheorem{thm}{Satz}
\newtheorem{lem}{Lemma}[thm]
\newtheorem{cor}{Korollar}[lem]

\renewcommand\thesection{}
\renewcommand\thesubsection{}
\renewcommand\thesubsubsection{}
\newcommand{\R}{\mathbb{R}}
\newcommand{\Q}{\mathbb{Q}}
\newcommand{\N}{\mathbb{N}}
\renewcommand{\qedsymbol}{$\blacksquare$}
\newcommand\numberthis{\addtocounter{equation}{1}\tag{\theequation}}

\author{Philipp Pascal Schmale}

\title{Lösungen zu den Aufgaben der zweiten Runde des 50sten Bundeswettbewerbs Mathematik}

\begin{document}

\maketitle

\tableofcontents

\pagebreak
\fancyhead[L]{BENUTZTE HILFSMITTEL}

\section*{Benutzte Hilfsmittel}
\addcontentsline{toc}{section}{Benutzte Hilfsmittel}

\subsection*{Für Aufgabe 1 benutzte Hilfsmittel}

Für Aufgabe 1 habe ich regelmäßig einen Taschenrechner verwendet. Allerdings habe ich in der endgültigen Fassung 
darauf geachtet, dass alle Berechnungen auch ohne einen Taschenrechner nachvollziehbar sind. In der Regel habe 
ich hierzu Fußnoten verwendet. Dann steht in Klammern "`Berechnung ohne Taschenrechner"' o.Ä. und am Ende ein 
Superskript mit einer Zahl. Unten auf der Seite sind dann die Inhalte der Fußnoten mit  den entsprechenden Zahlen 
zu finden (Beispiel\footnote{) Hier steht dann die Berechnung oder Berechnungsüberprüfung, die auch ohne einen 
Taschenrechner nachvollziehbar sein sollte.}).

% ? Muss das Beispiel sein?

\subsection*{Für Aufgabe 2 benutzte Hilfsmittel}

Für sowohl die Findung als auch die Ausarbeitung der Lösung von Aufgabe 2 habe ich keinerlei Hilfsmittel benutzt.

\subsection*{Für Aufgabe 3 benutzte Hilfsmittel}

Für sowohl die Findung als auch die Ausarbeitung der Lösung von Aufgabe 3 habe ich keinerlei Hilfsmittel benutzt.

\subsection*{Für Aufgabe 4 benutzte Hilfsmittel}

Für sowohl die Findung als auch die Ausarbeitung der Lösung von Aufgabe 4 habe ich keinerlei Hilfsmittel benutzt.

\pagebreak
\fancyhead[L]{\leftmark}

\section{Aufgabe 1}

\subsection*{Hinweise zur Notation}

$\N_0$ ist die Menge der natürlichen Zahlen mit der Null, also $\{ 0, 1, 2, 3, \ldots \}$.

$\text{[g]}$ heißt, dass die Einheit, in der gerechnet wurde, Gramm ist, dies aber ggf. in Zwischenrechnungen nicht 
angegeben wurde. Es sollte stets klar sein, in welcher Einheit gerechnet wird, da die Masse die einzige physikalische 
Größe ist, von der Gebrauch gemacht wird.

"`HS"' ist kurz für (die entsprechend deklinierte Form von) "`Hilfssatz"'.

\subsection*{Kurze Antwort}

Smilla kann mindestens 1.021.615g Gold für sich garantieren.

\subsection*{Beweis (d.h. die etwas längere Antwort)}

Zu zeigen ist also der folgende Satz:
\begin{thm}
    Die größte Masse an Gold, die Smilla mit Sicherheit mindestens gewinnen kann, ist 1.021.615g.
\end{thm}

\begin{proof}[Beweis des Satzes]
    Der Beweis besteht aus zwei Teilen:
    \begin{enumerate}
        \item Dem Beweis, dass Smilla stets 1.021.615g Gold gewinnen kann
        \item Dem Beweis, dass es keine größere Masse Gold gibt, die Smilla immer gewinnen kann
    \end{enumerate}
    \renewcommand{\qedsymbol}{$\square$}
    \begin{lem}\label{smilla}
        Smilla kann stets versichern, dass sie mindestens 1.021.615g Gold gewinnt.
    \end{lem}
    \begin{proof}[Beweis des HS]
        Um mindestens 1.021.615g Gold zu gewinnen, kann Smilla folgenden vierschrittigen Plan verfolgen (wobei die 
        Schritte 2 und 3 ggf. mehrfach eintreten):
        \begin{enumerate}
            \item Die 2020 Gold-Nuggets auf drei Haufen aufteilen, und zwar nach folgendem Schema:
            \begin{enumerate}
                \item Auf Haufen 1 kommt jedes Gold-Nugget, dessen Masse in Gramm eine der Formen $4k+1$ und $4k+2$ 
                annimmt (für ein $k \in \N_0$). Das sind also die Nuggets mit den Massen 1g ($1 = 4 \cdot 0+1$), 2g 
                ($2 = 4\cdot 0+2$), 5g ($5 = 4 \cdot 1+1$), 6g ($6=4\cdot 1+2$), \dots, 2017g ($2017=4\cdot 504+1$), 
                2018g ($2018=4\cdot 504+2$).
                \item Auf Haufen 2 kommt jedes Gold-Nugget, dessen Masse in Gramm eine der Formen $4k$ und $4k+3$ 
                annimmt (für ein $k \in \N_0$) und dessen Masse nicht 2020g ist. Das sind also die Nuggets mit den 
                Massen 3g ($3=4\cdot 0+3$), 4g ($4=4\cdot 1$), 7g ($7=4\cdot 1+3$), 8g ($8=4\cdot 2$), \dots, 2016g 
                ($2016=4\cdot 504$), 2019g ($2019=4\cdot 504+3$).
                \item Auf "`Haufen"' 3 kommt das Nugget mit der Masse 2020g.
            \end{enumerate}
            \item Wenn Leo die rote Truhe wählt, das leichteste noch verfügbare Nugget von Haufen 1 hineinlegen (das 
            "`leichteste"' heißt das mit der kleinsten Masse). Falls von Haufen 1 kein Nugget mehr verfügbar ist, das 
            Nugget mit der Masse 2020g hineinlegen und falls auch das nicht mehr verfügbar ist, das leichteste noch 
            verfügbare Nugget (das dann offensichtlich von Haufen 2 kommen muss) hineinlegen.
            \item Wenn Leo die blaue Truhe wählt, das leichteste noch verfügbare Nugget von Haufen  2 hineinlegen. Falls 
            von Haufen 2 kein Nugget mehr verfügbar ist, das Nugget mit der Masse 2020g hineinlegen und falls auch das 
            nicht mehr verfügbar ist, das leichteste noch verfügbare Nugget (das dann offensichtlich von Haufen 1 
            sein muss) hineinlegen.
            \item Wenn alle Nuggets verteilt sind, die Truhe wählen, in der sich das Nugget mit der Masse 2020g 
            befindet.

        \end{enumerate}

        Anm.: Im Folgenden werde ich Haufen 1, Haufen 2 und Haufen 3 mit $H_1$ bzw. $H_2$ bzw. $H_3$ bezeichnen.

        Zunächst muss natürlich (kurz) bewiesen werden, dass die Aufteilung auf die drei Haufen so, wie sie in 
        Schritt 1 beschrieben wurde, überhaupt möglich ist. D.h., dass jedes Nugget auf genau einen Haufen kommt. 
        Oder: Dass jedes Nugget auf einen Haufen kommt und kein Nugget auf zwei Haufen kommt. Offensichtlich kann 
        jedes Nugget nur einen der Reste 0, 1, 2, 3 haben, wenn man seine Masse in g durch 4 teilt. \\
        Betrachte man zunächst die mit Rest 0: Sie kommen alle bis auf das mit der Masse 2020g auf $H_2$ und auf 
        keinen anderen (nicht auf $H_1$, da sie weder einen Rest von 1 noch einen von 2 haben, wenn man ihre 
        Masse in g durch 4 teilt; nicht auf $H_3$, da auf diesen nur das Nugget mit der Masse 2020g kommt, 
        welches ja in diesem Fall ausgeschlossen wurde). Das mit der Masse 2020g kommt auf $H_3$ und 
        offensichtlich auch nur auf $H_3$. \\
        Betrachte man nun die mit Rest 1: Sie kommen ausnahmslos auf $H_1$ und auf keinen anderen (nicht auf $H_2$, da sie 
        weder einen Rest von 0 noch von 3 haben, wenn man ihre Massen in g durch 4 teilt; nicht auf $H_3$, da auf 
        diesen nur das Nugget mit der Masse 2020g kommt). \\
        Betrachte man nun die mit Rest 2: Sie kommen alle auf $H_1$ und auf keinen anderen (nicht auf $H_2$, da sie weder 
        einen Rest von 0 noch von 3 haben; nicht auf $H_3$, da auf diesen nur das Nugget mit der Masse 2020g kommt). \\
        Und zu guter Letzt betrachte man nun die mit Rest 3: Sie kommen alle auf $H_2$ (nicht auf $H_1$, da sie weder 
        einen Rest von 1 noch von 2 haben; nicht auf $H_3$, da auf diesen nur das Nugget mit der Masse 2020g kommt).

        Im ersten Schritt wird Smilla also die 2020 Nuggets auf drei Haufen aufteilen, wobei jedes Nugget auf genau 
        einen Haufen kommt.
        
        Ähnlich zu dem Beweis, dass die Aufteilung tatsächlich möglich ist (also, dass jedes Nugget auf genau 
        einen Haufen kommt), werde ich hier noch beweisen, dass bei dem Verteilen der Nuggets auf die 
        Truhen nach dem in den Schritten 2 und 3 beschriebenen Schema stets eindeutig ist, welches Nugget Smilla in 
        die von Leo gewählte Truhe legen muss, und, dass dieses Nugget auch stets zur Verfügung steht.

        Dies ist jedoch ziemlich offensichtlich. Denn man kann sich die Wahl auch folgendermaßen vorstellen:\\
        Wenn Leo die rote Truhe wählt, geht Smilla im Kopf alle Nuggets (auch die, die schon gelegt wurden) durch, 
        allerdings nicht in irgendeiner Reihenfolge, sondern zunächst vom Leichtesten zum Schwersten von $H_1$, 
        dann das mit der Masse 2020g, dann vom Leichtesten zum Schwersten von $H_2$. Dabei nimmt sie dann das Erste, 
        das ihr noch zur Verfügung steht, und legt es in die rote Truhe. \\
        Wenn Leo die blaue Truhe wählt, geht sie auch alle Nuggets durch, allerdings zuerst vom Leichtesten zum 
        Schwersten von $H_2$, dann das mit der Masse 2020g, dann vom Leichtesten zum Schwersten von $H_1$.

        In beiden Fällen wird, da es keine zwei Nuggets mit der gleichen Masse gibt, eindeutig bestimmt sein, welches 
        Nugget Smilla in die von Leo gewählte Truhe legen muss.

        Es bleibt also zu beweisen, dass, wenn alle Nuggets verteilt sind, die Truhe, in der das Nugget mit der Masse 
        2020g liegt, Nuggets mit einer Gesamtmasse von mindestens 1.021.615g beinhaltet. Denn dann wird Smilla gemäß 
        Schritt 4 diese Truhe wählen und mindestens 1.021.615g Gold gewinnen.

        Zunächst berechne man hierzu folgende Werte: die Masse an Gold, die auf $H_1$ bzw. $H_2$ (bzw. $H_3$) liegt, 
        welche ich im Folgenden mit $m(H_1)$ bzw. $m(H_2)$ bzw. $m(H_3)$ bezeichnen werde.

        Offensichtlich befindet sich auf $H_3$ genau ein Nugget (das mit der Masse 2020g). Die Gesamtmasse der 
        Nuggets auf $H_3$ besteht also nur aus dem Nugget mit der Masse 2020g, also ist $m(H_3) = 2020 [\text{g}]$ 
        (von hier an werde ich bei $m(H_1), m(H_2), m(H_3)$ die Einheit g nicht mehr dazuschreiben). Betrachte man 
        nun die noch übrigen Nuggets, also die mit Massen von höchstens 2019g. Diese lassen sich folgendermaßen 
        schreiben (wobei ich "`das Nugget mit der Masse"' und die Einheit Gramm weggelassen habe): 
        $4\cdot 0+1, 4\cdot 0+2, 4\cdot 0+3, 4\cdot 1+0, 4\cdot 1+1, 4\cdot 1+2, 4\cdot 1+3, 4\cdot 2+0, 4\cdot 2+1, 
        \ldots, 4\cdot 504+0\hspace{5pt} (=2016), 4\cdot 504+1, 4\cdot 504+2, 4\cdot 504+3$. Offensichtlich sind 
        genau 505 von diesen von der Form $4k +1$ (für ein $k \in \N_0$), nämlich $4\cdot 0+1, 4\cdot 1+1, 4\cdot 2+1, 
        \ldots, 4\cdot 503+1, 4\cdot 504+1$. Gleichzeitig sind 505 von der Form $4k+2$ (für ein $k\in\N_0$), nämlich 
        $4\cdot 0+2, 4\cdot 1+2, 4\cdot 2+2, 4\cdot 3+2, \ldots, 4\cdot 503+2, 4\cdot 504+2$. Es gibt unter den 2019 
        Nuggets also 505 mit Massen (in g) der Form $4k+1$ und 505 mit Massen (in g) der Form $4k+2$. Nun sind dies 
        aber genau die Nuggets, die auf $H_1$ kommen.
        Also sind die 1010 Nuggets, die sich auf $H_1$ befinden, die mit folgenden Massen (in g): $4\cdot 0+1, 
        4\cdot 0+2, 4\cdot 1+1, 4\cdot 1+2, 4\cdot 2+1, 4\cdot 2+2, \ldots, 4\cdot 503+1, 4\cdot 503+2, 4\cdot 504+1, 
        4\cdot 504+2$. Die Gesamtmasse dieser Nuggets, also $m(H_1)$, ist also:
        \begin{align*}
            &m(H_1)\\
            =&(4\cdot 0+1)+(4\cdot 0+2)+(4\cdot 1+1)+(4\cdot 1+2)+\cdots+(4\cdot 504+1)+(4\cdot 504+2)\\
            = &(4\cdot 0+1)+(4\cdot1+1)+\cdots+(4\cdot 503+1)+(4\cdot 504+1)\\
            &+(4\cdot 0+2)+(4\cdot1+2)+\cdots+(4\cdot 503+2)+(4\cdot 504+2)\\
            = &(4\cdot0+4\cdot1+\cdots+4\cdot503+4\cdot504)+(1+1+\cdots+1)\\
            &+(4\cdot0+4\cdot1+\cdots+4\cdot503+4\cdot504)+(2+2+\cdots+2)\\
        \end{align*}
        Da $4\cdot0+1, 4\cdot1+1, \ldots, 4\cdot504+1$ insgesamt 505 Summanden sind, handelt sich bei der Summe
        $(1+1+\cdots+1)$ in der zweitletzten Zeile um 505 Einer (aus jedem Summanden der oberen Summe wurde ja eine 
        Eins genommen). Da auch $4\cdot0+2,5\cdot1+2,\ldots,4\cdot504+2$ insgesamt 505 Summanden sind, handelt es sich 
        mit gleicher Begründung bei der Summe $(2+2+\cdots+2)$ in der letzten Zeile um 505 Zweier. Es ist also:
        \begin{align*}
            &m(H_1)\\
            = &4\cdot(0+1+2+3+\cdots+504)+(505\cdot1)\\
            &+4\cdot(0+1+2+3+\cdots+504)+(505\cdot2)\\
            = &4\cdot\left(\frac{504\cdot(504+1)}{2}\right)+505
            +4\cdot\left(\frac{504\cdot(504+1)}{2}\right)+2\cdot 505\\
            =&2\cdot504\cdot505+505+2\cdot 504\cdot505+2\cdot505\\
            =&1008\cdot505+505+1008\cdot505+2\cdot505=(1008+1+1008+2)\cdot 505\\
            =&(2016+3)\cdot 505=2019\cdot505= 1.019.595
        \end{align*}
        (Berechnung von $2019\cdot 505$ ohne Taschenrechner\footnote{) $2019\cdot 505= 2019\cdot 1010 \cdot \frac12
        =(2019\cdot 1000+2019\cdot 10)\cdot \frac12=\frac12(2.019.000+20.190)=1.009.500+10.095=1.019.595$})\\
        Anm.: Zur Berechnung von $0+1+2+3+\cdots+504$ habe ich die aus der Schule bekannte Gaußsche Summenformel 
        verwendet, nach welcher $1+2+3+\cdots+(n-1)+n=\frac{n(n+1)}{2}$, also hier (mit $n=504$): 
        $0+1+2+3+\cdots+504=1+2+3+\cdots+504=\frac{504\cdot(504+1)}{2}$.

        Es ist also $m(H_1)= 1.019.595$ und $m(H_3)=2020$. Gleichzeitig ist aber, da jedes der Nuggets auf einem der 
        drei Haufen, allerdings keines auf zweien landet, die Gesamtmasse der drei Haufen gleich der Gesamtmasse 
        aller Nuggets; d.h.: $m(H_1)+m(H_2)+m(H_3)=1+2+3+\cdots+2019+2020=\frac{2020\cdot2021}{2}=2.041.210$ (nach der 
        Gaußschen Summenformel mit $n=2020$; Berechnung ohne Taschenrechner\footnote{) $\frac{2020\cdot2021}{2}=1010
        \cdot2021=1000\cdot2021+10\cdot2021=2.021.000+20.210=2.041.210$}). Es ist also $m(H_2)=2.041.210-m(H_1)-m(H_2) 
        = 2.041.210-1.019.595-2020=1.019.595$ (Überprüfung der Berechnung ohne Taschenrechner\footnote{) $2.041.210
        -1.019.595-2020=1.019.595\Leftrightarrow 2.041.210-2020=2\cdot 1.019.595=2.039.190\Leftrightarrow 2.041.210
        =2.039.190+2020=2.039.210+2000=2.041.210$})

        Die berechneten Werte sind also: $m(H_1)=1.019.595, m(H_2)=1.019.595, m(H_3)=2020$.

        Nun ist in den Regeln des Spiels festgelegt, dass erst aufgehört wird, wenn alle Nuggets in einer der beiden 
        Truhen liegen. Also muss auch das Nugget mit der Masse 2020g in einer der beiden Truhen liegen. Um nicht zu 
        allgemein formulieren zu müssen (was vermutlich unverständlicher wäre), kann man die beiden Fälle einzeln 
        betrachten:

        \textit{Fall 1: Das Nugget mit der Masse 2020g landet in der roten Truhe.} Offensichtlich können dann zu dem 
        Zeitpunkt, zu dem Smilla das Nugget mit der Masse 2020g in die rote Truhe legt, noch keine Nuggets von $H_1$ in 
        der blauen Truhe liegen. Denn wann immer sie eines hätte legen können (also wann immer Leo die blaue Truhe 
        gewählt hat), hätte sie auch noch das mit der Masse 2020g legen können (da sie es jetzt zur Verfügung hat, 
        hätte sie es zu diesem früheren Zeitpunkt auch zur Verfügung gehabt), was sie dann gemäß Schritt 3 zunächst 
        hätte tun müssen. Es liegt also kein Nugget von $H_1$ in der blauen Truhe. Da sie aber auch kein Nugget von 
        $H_1$ mehr zur Verfügung hat (sonst müsste sie das zuerst legen, wenn Leo die rote Truhe wählt), müssen alle 
        Nuggets von $H_1$ bereits in einer der beiden Truhen liegen. Weil diese Truhe nicht die blaue sein kann (in 
        ihr liegen ja keine Nuggets von $H_1$), müssen also alle Nuggets von $H_1$ in der roten Truhe liegen.

        \textit{Fall 2: Das Nugget mit der Masse 2020g landet in der blauen Truhe.} Offensichtlich können dann zu dem 
        Zeitpunkt, zu dem sie das Nugget mit der Masse 2020g in die blaue Truhe legt, noch keine Nuggets von $H_2$ in 
        der roten Truhe liegen. Denn wann immer sie eines hätte legen können (also wann immer Leo die rote Truhe 
        gewählt hat), hätte sie auch noch das mit der Masse 2020g legen können (da sie es jetzt zur Verfügung hat, 
        hätte sie es zu diesem früheren Zeitpunkt auch zur Verfügung gehabt), was sie dann gemäß Schritt 2 zunächst 
        hätte tun müssen. Es liegt also kein Nugget von $H_2$ in der roten Truhe. Da sie aber auch kein Nugget von 
        $H_2$ mehr zur Verfügung hat (sonst müsste sie das zuerst legen, wenn Leo die blaue Truhe wählt), müssen alle 
        Nuggets von $H_2$ bereits in einer der beiden Truhen liegen. Weil diese Truhe nicht die rote sein kann (in 
        ihr liegen ja keine Nuggets von $H_2$), müssen also alle Nuggets von $H_2$ in der blauen Truhe liegen.

        Wenn das Nugget mit der Masse 2020g in der roten Truhe landet, müssen in ihr also auch alle Nuggets von $H_1$ 
        sein, und wenn es in der blauen Truhe liegt, müssen in dieser auch alle Nuggets von $H_2$ liegen. In erstem 
        Fall hat also die rote Truhe eine Gesamtmasse von mindestens $m(H_1)+2020=1.021.615 [\text{g}]$, da in ihr 
        alle Nuggets von $H_1$ (die eine Gesamtmasse von $m(H_1)$ haben) und das mit der Masse 2020g liegen. In 
        letzterem Fall muss die blaue Truhe eine Gesamtmasse von mindestens $m(H_2)+2020=1.021.615 [\text{g}]$ haben,
        da in ihr alle Nuggets von $H_2$ (die eine Gesamtmasse von $m(H_2)$ haben) und das mit der Masse 2020g 
        liegen. (Berechnungen von $m(H_1)+2020$ und $m(H_2)+2020$ ohne Taschenrechner\footnote{) $m(H_1)+2020=
        1.019.595+2020=1.019.615+2000=1.021.615$ und $m(H_2)+2020=1.019.595+2020=m(H_1)+2020=1.021.615$})

        In beiden Fällen beinhaltet also die Truhe, in der das Nugget mir der Masse 2020g liegt, Gold mit einer 
        Gesamtmasse von mindestens 1.021.615g. Wenn Smilla dann gemäß Schritt 4 diese Truhe, in der das Nugget mit 
        der Masse 2020g liegt, wählt, wählt sie stets eine mit mindestens 1.021.615g Gold.

        Smilla kann also stets mit Sicherheit mindestens 1.021.615g Gold gewinnen!
    \end{proof}
    \begin{lem}\label{leo}
        Leo kann stets verhindern, dass Smilla mehr als 1.021.615g Gold bekommt.
    \end{lem}
    \begin{proof}[Beweis des HS]
        Dieser Beweis ist recht simpel, da Leo, um zu verhindern, dass Smilla mehr als 1.021.615g Gold bekommt, 
        einfach folgenden zweischrittigen Plan verfolgen kann:
        \begin{enumerate}
            \item Solange die blaue Truhe wählen, bis in ihr insgesamt mehr als 1.019.595g Gold sind.
            \item Sobald in der blauen Truhe mehr als 1.019.595g Gold sind, nur noch die rote Truhe wählen.
        \end{enumerate}
        Am Ende sind damit in der blauen Truhe auf jeden Fall mindestens 1.019.595g Gold. Wie im Beweis von HS 
        \ref{smilla} berechnet, ist die Gesamtmasse der 2020 Nuggets 2.041.210g. Wenn also in der blauen Truhe 
        mindestens 1.019.595g Gold sind, können in der roten Truhe nur noch höchstens $2.041.210\text{g}-1.019.595
        \text{g}=1.021.615\text{g}$ Gold sein(Überprüfung der Berechnung ohne Taschenrechner\footnote{) $2.041.210-
        1.019.595=1.021.615\Leftrightarrow 2.041.210=1.021.615+1.019.595=2.021.615+19.595=2.031.615+9.595=2.040.615
        +595=2.041.115+95=2.041.210$}).

        Offensichtlich kann Smilla in einem Zug nur maximal 2020g Gold  in eine der beiden Truhen legen. Wenn nun 
        irgendwann mehr als 1.021.615g Gold in der blauen Truhe wären, muss es einen Zug geben, in dem Smilla das 
        Nugget in die Truhe legt, das dafür sorgt, dass die Gesamtmasse der in der blauen Truhe befindlichen Nuggets 
        größer als 1.021.615g ist. Dieses Nugget kann höchstens 2020g wiegen. Bevor sie das Nugget hineinlegt, 
        müssten sich also schon Nuggets in der Truhe befinden, die eine Gesamtmasse von mehr als $1.021.615\text{g}-
        2020\text{g}=1.019.595\text{g}$ haben (Überprüfung ohne Taschenrechner\footnote{) $1.021.615-2020=
        1.019.5959 \Leftrightarrow 1.021.615=1.019.595+2020=m(H_1)+2020$, s. Fußnote 4}), 
        denn sonst würden die zusätzlichen maximal 2020g nicht genügen, um die Gesamtmasse über 1.021.615g zu 
        bringen. Wenn sich aber bereits mehr als 1.019.595g Gold in der blauen Truhe befunden hätten, hätte Leo -- 
        gemäß seinem Plan -- die rote Truhe gewählt; Smilla hätte also gar kein Nugget mehr in die blaue Truhe 
        legen können! Wenn Leo seinen simplen Plan verfolgt, kann es also nicht passieren, dass sich in der blauen 
        Truhe mehr als 1.021.615g Gold befinden. 

        Es können sich also weder in der blauen noch in der roten Truhe mehr als 1.021.615g Gold befinden. Demnach 
        kann Smilla natürlich auch keine Truhe wählen, in der sich mehr als 1.021.615g Gold befinden. Leo kann also 
        stets verhindern, dass Smilla mehr als 1.021.615g Gold bekommt!
    \end{proof}
    \renewcommand{\qedsymbol}{$\blacksquare$}
    Smilla kann also nach HS \ref{smilla} stets 1.021.615g Gold gewinnen. Allerdings kann sie keine größere Masse 
    Gold mit Sicherheit gewinnen, denn das kann Leo nach HS \ref{leo} stets verhindern. Die gröte Masse an Gold, 
    die Smilla mit Sicherheit gewinnen kann, ist also 1.021.615g.
\end{proof}

\pagebreak
\section{Aufgabe 2}

\subsection*{Hinweise zur Darstellung}

$\gcd(\alpha, \beta)$ steht für den größten gemeinsamen Teiler (engl.: greatest common divisor) der Zahlen $\alpha$ 
und $\beta$. Also ist z.B. $\gcd(\alpha, \beta) = 1$ äquivalent zu "`$\alpha$ und $\beta$ sind teilerfremd"'.

"`O.B.d.A."' (oder mitten im Satz "`o.B.d.A."') ist kurz für "`ohne Beschränkung der Allgemeinheit"'.

$\Q$ ist die Menge der rationalen Zahlen.

$\N$ ist die Menge der natürlichen Zahlen (ohne die 0) und $\N_0$ die Menge der natürlichen Zahlen mit der 0.

$\alpha \mid \beta$ heißt, dass $\alpha$ ein Teiler von $\beta$, also $\beta$ ein Vielfaches von $\alpha$ ist.

\subsection*{Lösung von Aufgabe 2}

Die in Aufgabe 2 zu zeigende Aussage ist nun die des folgenden Satzes:

\begin{thm}
    Es gibt keine drei Zahlen $x, y, z \in \Q$ mit $x + y + z = 0$ und $x^2 + y^2 + z^2 = 100$. \label{aufgabe_2}
\end{thm}
\begin{proof}[Beweis des Satzes durch Widerspruch]
    Nehme man an, es gäbe drei Zahlen $x, y, z \in \Q$, die beide Gleichungen erfüllen. Offenbar können nicht alle 
    drei Zahlen $x, y, z$ nicht-positiv und auch nicht alle drei Zahlen $x, y, z$ nicht-negativ sein, denn damit 
    dann $x+y+z = 0$ erfüllt wäre, müsste $x = y = z = 0$, was der zweiten Gleichung $x^2+y^2+z^2 = 100$ 
    widerspricht. Es sind also entweder zwei der drei Zahlen $x, y, z$ nicht-negativ und eine nicht-positiv oder es 
    sind zwei nicht-positiv und eine nicht-negativ. Unter der Annahme, dass es ein Lösungstripel $x, y, z$ gibt, 
    müsste es auch eines mit zwei nicht-negativen und einer nicht-positiven Zahl geben. Denn zu jedem Lösungstripel 
    $x, y, z$, dass zwei nicht-positive Zahlen und eine nicht-negative Zahl beinhaltet, gibt es ein Lösungstripel 
    $-x,-y,-z$ (das, wenn $x, y, z$ hinreichend für die beiden Gleichungen ist, dies offensichtlich ebenfalls ist), 
    das dann zwei nicht-negative Zahlen und eine nicht-positive Zahl beinhaltet. Wenn es also überhaupt 
    Lösungstripel gibt (wovon ja für den Widerspruchsbeweis ausgegangen wurde), so muss es also auch welche mit zwei 
    nicht-negativen Zahlen und einer nicht-positiven Zahl geben. Sei nun $x, y, z$ ein solches Tripel und sei 
    o.B.d.A. $z$ die nicht-positive Zahl (ansonsten benenne man einfach um, die Gleichungen ändern dadurch ihre Form 
    nicht). Es ist also $z \leq 0$ und $x, y \geq 0$. Da $x, y$ nicht-negative rationale Zahlen sind, muss es vier 
    Zahlen $a', b', c', d' \in \N_0$ ($b', d' \neq 0$) geben, für die $x = \frac{a'}{b'}$ und $y = \frac{c'}{d'}$. 
    Sei nun $\frac{a}{b}$ die vollständig gekürzte Form des Bruches $\frac{a'}{b'}$ und $\frac{c}{d}$ die des 
    Bruches $\frac{c'}{d'}$. D.h. es ist $x = \frac{a}{b}$ und $y = \frac{c}{d}$ mit $\gcd(a, b) = \gcd(c, d) = 1$. 
    Nun ist wegen $x+y+z=0$ offenbar $z = -x-y = -\frac{a}{b} - \frac{c}{d} = -\frac{a \cdot d}{b \cdot d} - 
    \frac{c \cdot b}{d \cdot b} = - \frac{ad+cb}{db}$. Damit gilt dann wegen $x^2+y^2+z^2 = 100$:
    \begin{align*}
        100 = x^2+y^2+z^2 &=\left(\frac{a}{b}\right)^2+\left(\frac{c}{d}\right)^2+\left(-\frac{ad+cb}{db}\right)^2\\
        &= \left( \frac{a}{b} \right)^2 + \left( \frac{c}{d} \right)^2 + \left( \frac{ad+cb}{db} \right)^2\\
        &= \left( \frac{a \cdot d}{b \cdot d} \right)^2 + \left( \frac{c \cdot b}{d \cdot b} \right)^2 + \left( 
        \frac{ad+cb}{db} \right)^2\\
        &= \frac{(ad)^2}{(db)^2} + \frac{(cb)^2}{(db)^2} + \frac{(ad)^2 + 2(ad)(cb) + (cb)^2}{(db)^2}\\
        &= \frac{(ad)^2 + (cb)^2 + (ad)^2 + 2abcd + (cb)^2}{(db)^2}\\
        &= \frac{2 \left( (ad)^2 + abcd + (cb)^2 \right)}{(db)^2} \quad \quad \quad \| \cdot \frac{(db)^2}{2}\\
        \iff 100\cdot \frac{(db)^2}{2}=  50 \cdot (db)^2 &= (ad)^2 + abcd + (cb)^2 \numberthis \label{frac_manip}
    \end{align*}\\
    Sei nun $\gcd(b, d) = t$ und seien die dann natürlichen Zahlen $\frac{b}{t}$ und $\frac{d}{t}$ mit $p$ bzw. $q$ 
    bezeichnet (sie sind natürlich, da, weil $t$ der größte gemeinsame Teiler von $b$ und $d$ ist, natürlich $t$ von 
    sowohl $b$ als auch $d$ Teiler sein muss). Die Zahlen $p$ und $q$ sind 
    dann teilerfremde natürliche Zahlen (sie sind teilerfremd, da nach dem Teilen durch den größten gemeinsamen 
    Teiler von $b$ und $d$ natürlich keine gemeinsamen Teiler mehr übrig bleiben können). Wenn man nun $b = pt$ und 
    $d = qt$ in \eqref{frac_manip} einsetzt, erhält man:
    \begin{align*}
        50 \cdot (db)^2 &= (ad)^2 + abcd + (cb)^2\\
        = 50 \cdot p^2t^2 \cdot q^2 t^2 &= a^2 \cdot q^2 t^2 + ac \cdot pt \cdot qt + c^2 \cdot p^2t^2 \quad \| 
        \cdot \frac{1}{t^2}\\
        \iff 50 \cdot t^2 \cdot  p^2 q^2 &= a^2 q^2 + acpq +c^2p^2 \numberthis \label{divisib}
    \end{align*}\\
    Nun ist in \eqref{divisib} offensichtlich die linke Seit durch $p$ und durch $q$ teilbar, also muss es die 
    rechte Seite auch sein. Es muss also $p \mid a^2 q^2 + acpq + c^2 p^2$ und $q \mid a^2 q^2 + acpq + c^2 p^2$. 
    Aus $p \mid a^2 q^2 + acpq + c^2 p^2$ folgt wegen $p \mid acpq + c^2 p^2$ offenbar, dass $p \mid a^2 q^2$, während 
    aus $q \mid a^2 q^2 + acpq + c^2 p^2$ wegen $q \mid a^2 q^2 + acpq$ folgt, dass $q \mid c^2 p^2$. Es gilt also:
    \[
        p \mid a^2 q^2 \text{ und } q \mid c^2 p^2
    \]
    Da aber $p$ und $q$ teilerfremd sind, folgt direkt:
    \[
        p \mid a^2 \text{ und } q \mid c^2 \iff \frac{b}{t} \mid a^2 \text{ und } \frac{d}{t} \mid c^2
    \]
    Nun sind aber $a$ und $b$ teilerfremd, d.h. kein Teiler von $b$ (was $\frac{b}{t}$ ja ist) außer 1 kann $a$ bzw. 
    $a^2$ teilen. Es muss also $\frac{b}{t} = 1$ und damit $b = t$. Da auch $d$ und $c$ teilerfremd sind, folgt mit 
    gleicher Begründung, dass $d = t$. Damit ist dann $b = d$. Setzt man dies in (\refeq{frac_manip}) ein, so erhält 
    man:
    \begin{align*}
        50 \cdot b^2 \cdot b^2 &= a^2 b^2 + ac \cdot b^2 + c^2 b^2 \quad \| \cdot \frac{1}{b^2}\\
        \iff 50b^2 &= a^2 + ac + c^2 \numberthis \label{even_odd}
    \end{align*}
    Nun ist die linke Seite von \eqref{even_odd} offensichtlich gerade und damit muss es die rechte auch sein. Wenn 
    nun aber beide Zahlen $a, c$ ungerade sind, so sind alle drei Summanden ungerade (das Produkt zweier ungerader 
    Zahlen ist stets eine ungerade Zahl) und damit dann die gesamte rechte Seite ungerade. Es muss also mindestens 
    eine der beiden Zahlen gerade sein. Wenn nun aber nur genau eine gerade ist, so ist auch das Quadrat dieser und 
    $ac$ gerade, jedoch nicht das Quadrat der anderen dann ungeraden Zahl. Die rechte Seite wäre die Summe zweier 
    gerader und  einer ungeraden Zahl, also auch ungerade. Da nicht beide und nicht genau eine der beiden Zahlen 
    ungerade sein können, müssen beide gerade sein. D.h. es gibt zwei Zahlen $r, s \in \N_0$, für die $a = 2r$ und 
    $c = 2s$ ist. Setzt man dies in \eqref{even_odd} ein, so erhält man:
    \begin{align*}
        50 b^2 &= (2r)^2 + (2r)(2s) + (2s)^2 = 4 \left( r^2 + rs + s^2 \right) \quad \| \cdot \frac12\\
        \iff 25b^2 &= 2 \left( r^2 + rs + s^2 \right)
    \end{align*}
    Da in dieser Gleichung die rechte Seite offensichtlich gerade ist, muss auch die linke, also $25b^2$, gerade sein. 
    Da aber 25 nicht gerade ist, muss $b^2$ gerade sein, wozu offensichtlich $b$ gerade sein muss. Nun sind also 
    sowohl $a$ also auch $b$ gerade, d.h. durch 2 teilbar. $a$ und $b$ haben also den gemeinsamen Teiler 2. Dies 
    stellt aber einen Widerspruch zu $\gcd(a, b) = 1$ dar! Denn offensichtlich ist ihr größter gemeinsamer Teiler 
    nicht 1, wenn 2 ein gemeinsamer Teiler von $a$ und $b$ ist. Da die Annahme, es gäbe drei Zahlen $x, y, z\in \Q$, 
    für die die beiden Gleichungen von Satz \ref{aufgabe_2} gelten, zu einem Widerspruch führt, muss sie falsch sein, 
    was dann den Widerspruchsbeweis von Satz \ref{aufgabe_2} vervollständigt.
\end{proof}

\pagebreak
\section{Aufgabe 3}

\subsection*{Anmerkungen zur Notation und einer häufig verwendeten Formel}

Im Folgenden wird $x_A$ die $x$- und $y_A$ die $y$-Koordinate des allgemeinen Punktes $A$ bezeichnen. D.h. hier, dass 
$P=(x_P, y_P), N=(x_N, y_N), M=(x_M, y_M), S=(x_S, y_S), Q=(x_Q, y_Q)$.\\
$\overline{AB}$ ist die Distanz von $A$ zu $B$ (stets positiv).

Sehr oft habe ich die Distanzformel 
\[\overline{AB}=\sqrt{(x_A-x_B)^2+(y_A-y_B)^2}=\sqrt{(x_B-x_A)^2+(y_B-y_A)^2}\text{    (für alle $A,B$)}\] 
verwendet. Ich habe bei den Gleichungen, bei denen ich sie verwendet habe, nicht dazugeschrieben, dass sich die 
Gleichungen mit der Distanzformel ergeben. Diese Entscheidung habe ich getroffen, da ich diese Formel so oft verwendet 
habe, dass, wenn ich sie jedes mal erwähnt hätte, dies einige Sätze deutlich komplizierter und teilweise sogar 
relativ unverständlich gemacht hätte. Ich denke, dass dies nicht problematisch ist; ich habe ja auch nicht dazugeschrieben, 
wenn ich z.B. einen Bruch gekürzt habe.

\subsection*{Algebraisierung der Aufgabe mit Hilfe eines kartesischen Koordinatensystems}

Um die Aufgabe von einer Geometrischen in eine eher Algebraische zu übersetzen, habe ich mich dazu entschieden, die 
in der Aufgabenstellung beschriebene Konstruktion mit Hilfe eines kartesischen Koordinatensystems zu beschreiben. 
Hierzu definiere man folgendermaßen ein kartesisches Koordinatensystem: $P$ sei der Urspung und $n$ die $x$-Achse; 
dabei sei die $x$-Achse so beschriftet, dass der Punkt $N$ eine positive $x$-Koordinate hat, bevor er $P$ passiert, 
und die $y$-Achse so, dass der Punkt $M$ eine positive $y$-Koordinate hat, bevor er $P$ passiert. Dies ist 
offensichtlich möglich, da $P$ auf $n$ liegt. Offensichtlich ist auch die Beschriftung der $y$-Achse (also der 
Orthogonalen zu $n$, die durch $P$ geht) möglich, denn die Geraden $m$ und $n$ liegen nicht aufeinander (sonst 
würden sie sich in mehr als einem Punkt schneiden); der Punkt $M$ muss also, bevor er $P$ passiert, auf einer Seite 
der $x$-Achse liegen.

Der Punkt $N$ hat, da er auf $n$, also der $x$-Achse, liegt, natürlich eine $y$-Koordinate von 0. Es ist also: 
$N=(x_N, 0)$. Sei nun $d_M$ die Distanz $\overline{PM}$, allerdings mit negativem Vorzeichen, wenn $y_M$ ein 
negatives Vorzeichen hat, also nachdem $M$ den Punkt $P$ passiert hat. Nun ist $x_N$ wegen $y_N=0$ die Distanz 
$\overline{PN}$, allerdings mit negativem Vorzeichen, nachdem $N$ den Punkt $P$ passiert hat. Offensichtlich ist 
dann, da $N$ und $M$ sich mit der gleichen, konstanten Geschwindigkeit von $P$ weg/zu $P$ hin bewegen, $d_M-x_N$ 
konstant. Sei diese Konstante mit $c$ bezeichnet. Dann ist $d_M-x_N=c\Leftrightarrow d_M=x_N+c$ und damit 
$\overline{PM}=|x_N+c|$. Nun ist $|x_N+c|=\overline{PM}=\sqrt{(x_M-x_P)^2+(y_M-y_P)^2}=\sqrt{x_M^2+y_M^2}$, da 
$y_P=x_P=0$. Die Gerade $m$ ist, da sie durch $P$, den Urspung des kartesischen Koordinatensystems, geht, natürlich 
eine Urspungsgerade. Im Folgenden werde ich nur die Fälle betrachten, in denen $m$ nicht auf der $y$-Achse liegt, 
also nicht orthogonal zu $n$ ist. Dann hat $m$ eine reele Steigung, die nicht Null ist, da $m$ sonst auf $n$ liegen 
würde, was einen Widerspruch dazu darstellt, dass $m$ und $n$ sich nur in einem Punkt $P$ schneiden. Diese Steigung 
wird im Folgenden $r$ genannt. Die Geradengleichung von $m$ ist also: $y=r\cdot x$ bzw. $x=\frac{y}{r}$. Da $M$ auf 
$m$ liegt, muss $x_M=\frac{y_M}{r}$ gelten. Zusammen mit $|x_N+c|=\sqrt{x_M^2+y_M^2}$ ergibt das dann: $|x_N+c|
=\sqrt{x_M^2+y_M^2}=\sqrt{(\frac{y_M}{r})^2+y_M^2}=\sqrt{\frac{1}{r^2}\cdot y_M^2+y_M^2}=\sqrt{(1+\frac{1}{r^2})
\cdot y_m^2}=|y_M|\cdot\sqrt{1+\frac{1}{r^2}}$. Da aber $y_M$ das gleiche Vorzeichen wie $d_M=x_N+c$ hat, sind die 
Betragsstriche nicht notwendig:
\begin{alignat*}{2}
    x_N+c&=y_M\cdot\sqrt{1+\frac{1}{r^2}}&&\\
    \Leftrightarrow y_M&=\frac{x_N+c}{\sqrt{1+\frac{1}{r^2}}}&&=\frac{r\cdot (x_N+c)}{r\cdot\sqrt{1+\frac{1}{r^2}}}\\
    &=\frac{r(x_N+c)}{\sqrt{r^2\left(1+\frac{1}{r^2}\right)}}&&=\frac{r(x_N+c)}{\sqrt{r^2+1}}
\end{alignat*}
Es ist also: $y_M=\frac{r(x_N+c)}{\sqrt{r^2+1}}$.\\
Nun ist $x_M=\frac{y_M}{r}$, also mit $y_M=\frac{r(x_N+c)}{\sqrt{r^2+1}}$:
\[
x_M=\frac{1}{r}\cdot y_M=\frac{1}{r}\cdot \frac{r(x_N+c)}{\sqrt{r^2+1}}=\frac{x_N+c}{\sqrt{r^2+1}}
\]
Es ist also:
\[
M=\left( \frac{x_N+c}{\sqrt{r^2+1}} , \frac{r(x_N+c)}{\sqrt{r^2+1}} \right)
\]
Die Punkte $P, N, M$ haben also folgende Koordinaten, wenn $m$ nicht orthogonal zu $n$ ist:
\[
P=(0, 0); N=(x_N, 0); M=\left( \frac{x_N+c}{\sqrt{r^2+1}} , \frac{r(x_N+c)}{\sqrt{r^2+1}} \right)
\]

Wenn $m$ orthogonal zu $n$ ist, also $m$ auf der $y$-Achse liegt, hat $M$ natürlich eine $x$-Koordinate von 0. 
Es gilt aber immer noch $|x_N+c|=\sqrt{x_M^2+y_M^2}$, was mit $x_M=0$ dann $|x_N+c|=\sqrt{y_M^2}=|y_M|$ ergibt. Da 
$y_M$ das gleiche Vorzeichen wie $d_M=x_N+c$ hat, gilt die Gleichung auch ohne die Betragsstriche; es ist also $y_M
=x_N+c$. Zusammen mit $x_M=0$ ist dann also, wenn $m$ orthogonal zu $n$ ist, $M=(0, x_N+c)$.

\subsection*{Beweis der zu zeigenden Aussage}

Nun muss nur noch bewiesen werden, dass es einen Punkt $Q$ gibt, sodass $P, M, N, Q$ stets auf einem Kreis liegen. 
Es muss also stets einen Punkt $S$ geben, von dem $P, M, N, Q$ den Gleichen Abstand haben.

Folgender Satz wird später hilfreich sein:
\begin{thm}\label{cor_hilfe}
        $x_A^2+y_A^2=2x_Ax_S+2y_Ay_S$ ist äquivalent zu $\overline{AS}=\sqrt{x_S^2+y_S^2}$ für alle 
        Punkte $A=(x_A, y_A), S=(x_S, y_S)$.
\end{thm}
\begin{proof}
    Die Behauptung ergibt sich direkt durch Umformen:
    \begin{alignat*}{3}
        &\overline{AS}=\sqrt{(x_A-x_S)^2+(y_A-y_S)^2}&&=\sqrt{x_S^2+y_S^2}\quad&&\|(\ldots)^2\\
        &\Leftrightarrow (x_A-x_S)^2+(y_A-y_S)^2&&=x_S^2+y_S^2\quad &&\|\text{2. Binomische Formel}\\
        &\Leftrightarrow (x_A^2-2x_Ax_S+x_S^2)+(y_A^2-2y_Ay_S+y_S^2)&&=x_S^2+y_S^2\quad &&\|-x_S^2-y_S^2\\
        &\Leftrightarrow x_A^2-2x_Ax_S+y_A^2-2y_Ay_S&&=0\quad &&\|+2x_Ax_S+2y_Ay_S\\
        &\Leftrightarrow x_A^2+y_A^2&&=2x_Ax_S+2y_Ay_S&&
    \end{alignat*}
\end{proof}
Der folgende Satz beschäftigt sich nur mit dem Fall "`$m$ orthogonal zu $n$"':
\begin{thm}\label{aufgabe_3}
    Die Punkte $P, M, N$ haben stets den gleichen Abstand zu einem Punkt $S$ wie der Punkt $Q\neq P$, wenn $m$ 
    orthogonal zu $n$ ist. Hierbei ist $S=\left(\frac{x_N}{2}, \frac{x_N+c}{2}\right)$ und $Q=\left(-\frac{c}{2}, 
    \frac{c}{2}\right)$.
\end{thm}
\begin{proof}
    \renewcommand{\qedsymbol}{$\square$}
    Wenn $P=Q$ wäre, wäre offensichtlich auch $x_P=x_Q$, also $x_Q=0$. Dann wäre aber auch $c=0$ und damit $d_M
    =x_N+c=x_N$. Wenn $N$ den Punkt $P$ passiert, also $x_N=0$ ist, wäre also auch $d_M=0$ und damit auch $|d_M|
    =\overline{PM}=0$. Damit  würden aber $M$ und $N$ beide auf $P$ liegen, also würden $M$ und $N$ gleichzeitig 
    den Punkt $P$ passieren, was ja nicht sein kann. Also muss $P\neq Q$ sein. Auch bewegt sich $Q$ offensichtlich 
    nicht, da $c$ und damit auch $\frac{c}{2}$ und $-\frac{c}{2}$ konstant sind.
    
    Zunächst werde ich ein paar Gleichungen in den folgenden Lemmata beweisen, aus denen dann mit Satz \ref{cor_hilfe}
    direkt die Korollare folgen:
    \begin{lem}\label{no_P}
        Es ist $x_P^2+y_P^2=2x_Px_S+2y_Py_S$.
    \end{lem}
    \begin{proof}
        Die Behauptung ergibt sich direkt durch Umformen:
        \begin{align*}
            x_P^2+y_P^2&=0^2+0^2\\
            =0=2\cdot 0\cdot x_S+2\cdot 0\cdot y_S&=2x_Px_S+2y_Py_S
        \end{align*}
    \end{proof}
    \begin{cor}\label{noc_P}
        Mit Satz \ref{cor_hilfe} folgt aus Lemma \ref{no_P} direkt: $\overline{PS}=\sqrt{x_S^2+y_S^2}$.
    \end{cor}
    \begin{lem}\label{no_M}
        Es ist $x_M^2+y_M^2=2x_Mx_S+2y_My_S$.
    \end{lem}
    \begin{proof}
        Die Behauptung ergibt sich direkt durch Umformen:
        \begin{align*}
            x_M^2+y_M^2&=0^2+(x_N+c)^2=0+2\cdot \frac{(x_N+c)^2}{2}=0+2\cdot (x_N+c)\cdot\frac{x_N+c}{2}\\
            &=2\cdot 0\cdot \frac{x_N}{2}+2\cdot y_M\cdot y_S=2x_Mx_S+2y_My_S
        \end{align*}
    \end{proof}
    \begin{cor}\label{noc_M}
        Mit Satz \ref{cor_hilfe} folgt aus Lemma \ref{no_M} direkt: $\overline{MS}=\sqrt{x_S^2+y_S^2}$
    \end{cor}
    \begin{lem}\label{no_N}
        Es ist $x_N^2+y_N^2=2x_Nx_S+2y_Ny_S$.
    \end{lem}
    \begin{proof}
        Die Behauptung ergibt sich direkt durch Umformen:
        \begin{align*}
            x_N^2+y_N^2&=x_N^2+0^2=x_N^2+0=2\cdot \frac{x_N^2}{2}+0=2x_N\frac{x_N}{2}+0\\
            &=2x_Nx_S+2\cdot 0\cdot \frac{x_N+c}{2}=2x_Nx_S+2y_Ny_S
        \end{align*}
    \end{proof}
    \begin{cor}\label{noc_N}
        Mit Satz \ref{cor_hilfe} folgt aus Lemma \ref{no_N} direkt: $\overline{NS}=\sqrt{x_S^2+y_S^2}$.
    \end{cor}
    \begin{lem}\label{no_Q}
        Es ist $x_Q^2+y_Q^2=2x_Qx_S+2y_Qy_S$.
    \end{lem}
    \begin{proof}
        Die Behauptung ergibt sich direkt durch Umformen:
        \begin{align*}
            x_Q^2+y_Q^2&=\left(-\frac{c}{2}\right)^2+\left(\frac{c}{2}\right)^2=\frac{c^2}{4}+\frac{c^2}{4}\\
            &=2\cdot \frac{c^2}{4}=\frac{c^2}{2}=0\cdot \frac{c}{2}+\frac{c^2}{2}=(-x_N+x_N)\frac{c}{2}+\frac{c^2}{2}\\
            &=-x_N\frac{c}{2}+x_N\frac{c}{2}+c\cdot\frac{c}{2}=x_N\left(-\frac{c}{2}\right)+(x_N+c)\frac{c}{2}\\
            &=2\frac{x_N}{2}\left(-\frac{c}{2}\right)+2\frac{x_N+c}{2}\frac{c}{2}=2x_Sx_Q+2y_Sy_Q\\
            &=2x_Qx_S+2y_Qy_S
        \end{align*}
    \end{proof}
    \begin{cor}\label{noc_Q}
        Mit Satz \ref{cor_hilfe}  folgt aus Lemma \ref{no_Q} direkt: $\overline{QS}=\sqrt{x_S^2+y_S^2}$
    \end{cor}
    Aus den Korollaren \ref{noc_P}, \ref{noc_M}, \ref{noc_N} und \ref{noc_Q} folgt direkt: 
    $\overline{PS}=\overline{MS}=\overline{NS}=\overline{QS}$, da sie alle gleich $\sqrt{x_S^2+y_S^2}$ sind. Die 
    Punkte $P, M, N, Q$ haben also alle den gleichen Abstand zu $S$, wenn $m$ orthogonal zu $n$ ist.
    \renewcommand{\qedsymbol}{$\blacksquare$}
\end{proof}

Wenn $n$ orthogonal zu $m$ ist, gibt es also eine Punkt $Q$, der sich nicht bewegt, sodass $P, M, N, Q$ den gleichen 
Abstand zu einem Punkt $S$ haben, also auf einem Kreis liegen.

Nun muss noch bewiesen werden, dass es auch einen solchen Punkt $Q$ gibt, wenn $m$ nicht orthogonal zu $m$ ist:

\begin{thm}
    Wenn $m$ nicht orthogonal zu $n$ ist, haben die Punkte $P, M, N$ den gleichen Abstand zu einem Punkt $S$ wie 
    ein Punkt $Q\neq P$
\end{thm}

\pagebreak
\section{Aufgabe 4}

\subsection*{Einführung hilfreicher Begriffe}

Seien zunächst folgende Begriffe eingeführt, die den Beweis hoffentlich kürzer, aber auch besser lesbar machen:

[Im Folgenden ist die $k$-te Spalte immer die $k$-te Spalte von links und die $k$-Zeile immer die $k$-te Zeile von oben.]
\begin{definition}
    Mit einer "`fast-positiv-reell-wertigen"' (kurz f.p.r.w.) Tabelle ist eine Tabelle gemeint, für die gilt, dass 
    in jedem ihrer Felder eine nicht-negative reelle Zahl steht, und dabei in jeder Spalte der Tabelle mindestens 
    eine positive.
\end{definition}
\begin{definition}
    Das Feld $(i, j)$ einer Tabelle ist das Feld in der $i$-ten Zeile und der $j$-ten Spalte dieser Tabelle.
\end{definition}
\begin{definition}
    Wenn $'$ die Feldwertfunktion einer Tabelle ist, ist $(i, j)'$ der Wert, der im Feld $(i, j)$ dieser Tabelle 
    steht.
\end{definition}
\begin{definition}\label{rfunktion}
    Wenn $R$ die Rechtecksfunktion und $'$ die Feldwertfunktion einer Tabelle ist, ist $R(a_1, b_1; a_2, b_2)$ kurz 
    für die folgende Summe, wobei $a_1\leq b_1$ und $a_2\leq b_2$ gelten muss:
    \begin{alignat*}{6}
        R(a_1, a_2; b_1, b_2) &= (a_1, &&a_2)' + &&(a_1, &&a_2+1)' + \quad \cdots \quad + &&(a_1, &&b_2)'\\
        &+(a_1+1, &&a_2)' + &&(a_1+1, &&a_2+1)' +\quad \cdots \quad + &&(a_1+1, &&b_2)'\\
        &+(a_1+2, &&a_2)' + &&(a_1+2, &&a_2+1)' +\quad \cdots \quad + &&(a_1+2, &&b_2)'\\
        & &&\vdotswithin{a} && &&\vdotswithin{a}\quad\quad\quad\quad\quad \ddots && &&\vdotswithin{b} \\
        &+(b_1-1, &&a_2)' + &&(b_1-1, &&a_2+1)' +\quad \cdots \quad + &&(b_1-1, &&b_2)'\\
        &+(b_1, &&a_2)' + &&(b_1, &&a_2+1)' + \quad \cdots \quad + &&(b_1, &&b_2)'
    \end{alignat*}    
\end{definition}
\begin{definition}
    Mit einer "`spaltenweise geordneten"' Tabelle ist eine Tabelle gemeint, für die stets $R(s, 1; s, m)\geq 
    R(s+1, 1; s+1, m)$ gilt, wobei $m$ die Anzahl an Zeilen dieser Tabelle ist. Äquivalent ist mit einer 
    "`zeilenweise geordneten"' Tabelle eine gemeint, für die stets $R(z, 1; z, n)\geq R(z+1, 1; z+1, n)$ gilt, 
    wobei $n$ die Anzahl an Spalten dieser Tabelle ist. Ist eine Tabelle zeilen- und spaltenweise geordneten, heißt 
    sie kurz auch "`wohlgeordnet"'.
\end{definition}

\subsection*{Hilfssätze über die eingeführten Begriffe}

\renewcommand{\qedsymbol}{$\square$}

\begin{lem}\label{r_summe}
    Wenn $R$ die Rechtecksfunktion einer Tabelle ist und $a_1\leq b_1\leq c_1$ und $a_2\leq b_2\leq c_2$ ist, gilt: 
    $R(a_1, a_2; b_1, b_2)+R(b_1+1, a_2; c_1, b_2)=R(a_1, a_2; c_1, b_2)$ und 
    $R(a_1, a_2; b_1, b_2)+R(a_1, b_2+1; b_1, c_2)=R(a_1, a_2; b_1, c_2)$.
\end{lem}
\begin{proof}[Beweis des HS]
    Erstere Aussage ergibt sich direkt aus der Definition der Rechtecksfunktion:
    \begin{alignat*}{3}
        &R(a_1, a_2; b_1, b_2)&& +R(b_1+1, a_2; c_1, b_2)&&\\
        &=(a_1, a_2)' &&+ (a_1, a_2+1)' &&+\cdots+ (a_1, b_2)'\\
        &+(a_1+1, a_2)' &&+ (a_1+1, a_2+1)' &&+\cdots+ (a_1+1, b_2)'\\
        & &&\vdots\quad\quad\quad\quad&&\vdots\\
        &+(b_1-1, a_2)' &&+ (b_1-1, a_2+1)' &&+\cdots + (b_1-1, b_2)'\\
        &+(b_1, a_2)' &&+ (b_1, a_2+1)' &&+ \cdots + (b_1, b_2)'\\
        &+R(b_1+1, a_2; &&c_1, b_2) &&\\
        &=(a_1, a_2)' &&+ (a_1, a_2+1)' &&+\cdots+ (a_1, b_2)'\\
        &+(a_1+1, a_2)' &&+ (a_1+1, a_2+1)' &&+\cdots+ (a_1+1, b_2)'\\
        & &&\vdots\quad\quad\quad\quad&&\vdots\\
        &+(b_1-1, a_2)' &&+ (b_1-1, a_2+1)' &&+\cdots + (b_1-1, b_2)'\\
        &+(b_1, a_2)' &&+ (b_1, a_2+1)' &&+ \cdots + (b_1, b_2)'\\
        &+(b_1+1, a_2)' &&+ (b_1+1, a_2+1)' &&+\cdots+ (b_1+1, b_2)'\\
        &+(b_1+2, a_2)' &&+ (b_1+2, a_2+1)' &&+\cdots+ (b_1+2, b_2)'\\
        & &&\vdots\quad\quad\quad\quad&&\vdots\\
        &+(c_1-1, a_2)' &&+ (c_1-1, a_2+1)' &&+\cdots + (c_1-1, b_2)'\\
        &+(c_1, a_2)' &&+ (c_1, a_2+1)' &&+ \cdots + (c_1, b_2)'\\
        &=R(a_1, a_2; c_1, &&b_2)&&
    \end{alignat*}
    Auch letztere Aussage ergibt sich direkt durch Einsetzen der Definition der Rechtecksfunktion:
    \begin{alignat*}{4}
        &R(a_1, a_2; b_1, b_2)&& &&+R(a_1, b_2+1; b_1, &&c_2)\\
        &=(a_1, a_2)' &&+\cdots+ (a_1, b_2)'
        &&+(a_1, b_2+1)'&&+\cdots+(a_1, c_2)'\\
        &+(a_1+1, a_2)'&&+\cdots+ (a_1+1, b_2)'
        &&+(a_1+1, b_2+1)'&&+\cdots+(a_1+1, c_2)'\\
        & \quad\vdots&&\quad\vdots&&+\quad\vdots&&\quad\vdots\\
        &+(b_1-1, a_2)'&&+\cdots + (b_1-1, b_2)'
        &&+(b_1-1, b_2+1)'&&+\cdots+(b_1-1, c_2)'\\
        &+(b_1, a_2)'&&+ \cdots + (b_1, b_2)'
        &&+(b_1, b_2+1)'&&+\cdots+(b_1, c_2)'\\
        &=R(a_1, a_2; b_1, &&c_2)
    \end{alignat*}
\end{proof}
Anm.  zu HS \ref{r_summe}: Wenn die für die Gültigkeit dieses HS nötigen Ungleichungen offensichtlich wahr waren, 
habe ich sie oft gar nicht erwähnt. Da sie so offensichtlich sind, wird hierdurch die mathematische Vollständigkeit 
nicht eingeschränkt.
\begin{lem}\label{r_zeile_spalte}
    Wenn $R$ die Rechtecksfunktion einer Tabelle mit $m$ Zeilen und $n$ Spalten ist, ist $R(i, 1; i, n)$ die Summe 
    der Werte in den Feldern der $i$-ten Zeile dieser Tabelle und $R(1, j; m, j)$ die Summe der Werte in den Feldern 
    der $j$-ten Spalte dieser Tabelle.
\end{lem}
\begin{proof}[Beweis des HS]
    Per Defintion ist $R(i, 1; i, n)=(i, 1)'+(i, 2)'+\cdots+(i, n-1)'+(i, n)$. Die Felder der $i$-ten Zeile sind die 
    der Form $(i, a)$, wobei $0<a\leq n$. Offensichtlich sind diese genau die Felder, deren Feldwerte aufsummiert 
    werden, um $R(i, 1; i, n)$ zu berechnen. $R(i, 1; i, n)$ ist also die Summe der Werte in den Feldern der $i$-ten 
    Zeile.

    Ähnlich ist per Definition $R(1, j; m, j)=(1, j)'+(2, j)'+\cdots+(m-1, j)'+(m, j)'$. Die Felder der $j$-ten Spalte 
    sind die der Form $(a, j)$, wobei $0<a\leq m$. Offensichtlich sind diese genau die Felder, deren Feldwerte 
    aufsummiert werden, um $R(1, j; m, j)$ zu berechnen. $R(1, j; m, j)$ ist also die Summe der Werte in den Feldern 
    der $j$-ten Spalte.
\end{proof}
\begin{lem}\label{mehr_nuller}
    Wenn in einer wohlgeordneten Tabelle mit Rechtecksfunktion $R$ für zwei Zahlen $i, j$ die Ungleichungen 
    $R(i, 1; i, n)>R(1, j; m, j)$ gilt, so gilt auch $R(i-a, 1; i-a, n)>R(1, j+b; m, j+b)$ für alle $a, b\geq0$, für 
    die es eine $(i-a)$-te Zeile und eine $(j+b)$-te Spalte gibt.
\end{lem}
\begin{proof}[Beweis des HS]
    Zunächst gilt, da die Tabelle wohlgeordnet ist:
    \begin{align*}
        R(i, 1; i, n)&\leq R(i-1, 1; i-1, n)\\
        R(i-1, 1; i-1, n)&\leq R(i-2, 1; i-2, n)\\
        &\quad\vdots\\
        R(i-(a-1), 1; i-(a-1), n)&\leq R(i-a, 1; i-a, n)
    \end{align*}
    Diese Ungleichungen lassen sich folgendermaßen zusammenfassen:
    \[
        R(i, 1; i, n)\leq R(i-1, 1; i-1, n)\leq R(i-2, 1; i-2, n)\leq\cdots\leq R(i-a, 1; i-a, n)
    \]
    Insbesondere ist also $R(i-a, 1; i-a, n)\geq R(i, 1; i, n)$. Erneut aufgrund der Wohlgeordnethait der Tabelle 
    gilt auch:
    \begin{align*}
        R(1, j; m, j)&\geq R(1, j+1; m, j+1)\\
        R(1, j+1; m, j+1)&\geq R(1, j+2; m, j+2)\\
        &\quad\vdots\\
        R(1, j+(b-1); m, j+(b-1))&\geq R(1, j+b; m, j+b)
    \end{align*}
    Auch diese Gleichungen lassen sich in einer Zeile zusammenfassen:
    \[
        R(1, j; m, j)\geq R(1, j+1; m, j+1)\geq R(1, j+2; m, j+2)\geq\cdots\geq R(1, j+b; m, j+b)
    \]
    Insbesondere ist also $R(1, j; m, j)\geq R(1, j+b; m, j+b)$. Es gelten also die folgenden drei Ungleichungen:
    \begin{align*}
        R(i-a, 1; i-a, n)&\geq R(i, 1; i, n)\\
        R(i, 1; i, n)&>R(1, j; m, j)\\
        R(1, j; m, j)&\geq R(1, j+b; m, j+b)
    \end{align*}
    In einer Zeile zusammengefasst ergibt das:
    \[
        R(i-a, 1; i-a, n)\geq R(i, 1; i, n)>R(1, j; m, j)\geq R(1, j+b; m, j+b)
    \]
    Insbesondere ist also $R(i-a, 1; i-a, n)>R(1, j+b; m, j+b)$.
\end{proof}

\subsection*{Lösung von Aufgabe 4}

In Aufgabe 4 werden Tabellen betrachtet, bei denen in jedem Feld eine nicht-negative reelle Zahl steht und dabei in 
jeder Spalte mindestens eine positive. Es handelt sich also um f.p.r.w. Tabellen. Auch müssen die Tabellen mehr 
Spalten als Zeilen haben. Es werden also f.p.r.w. Tabellen mit $m$ Zeilen und $n$ Spalten betrachtet, wobei $m<n$. 
Es soll nun bewiesen werden, dass es ein Feld mit einer positiven Zahl derart gibt, dass die Summe der Zahlen in der 
Zeile dieses Feldes größer als die Summe der Zahlen in der Spalte dieses Feldes ist. Nenne man dieses Feld $(i, j)$, 
die Feldwertfunktion de betrachteten Tabelle $'$ und ihre Rechtecksfunktion $R$. Dann muss also der Feldwert von $(i, j)$ 
positiv sein; es muss also $(i, j)'>0$ gelten. Auch muss die Summe der Zahlen in der Zeile dieses Feldes (welche 
gemäß HS \ref{r_zeile_spalte} durch $R(i, 1; i, n)$ gegeben ist) größer als die Summe der Zahlen in der Spalte 
dieses Feldes (welche gemäß HS \ref{r_zeile_spalte} durch $R(1, j; m, j)$ gegeben ist) sein; es muss also 
$R(i, 1; i, n) > R(1, j; m, j)$. Insgesamt ist die in Aufgabe 4 zu zeigende Aussage also äquivalent zu folgendem 
Satz:

\begin{thm}\label{haupt4}
    In jeder f.p.r.w. Tabelle mit $m$ Zeilen und $n$ Spalten, wobei $m<n$, gibt es ein Feld $(i, j)$, sodass 
    $(i, j)'>0$ und $R(i, 1; i, n) > R(1, j; m, j)$, wobei $'$ die Feld- und $R$ die Rechtecksfunktion dieser 
    Tabelle ist.
\end{thm}

\begin{proof}[Beweis des Satzes durch Widerspruch]
    Nehme man zunächst an, es gäbe eine Tabelle, für die Satz \ref{haupt4} nicht gilt. Nach folgendem HS kann man 
    davon ausgehen, dass diese Tabelle wohlgeordnet ist.
    \begin{lem}\label{fallbeschrank}
        Wenn es keine wohlgeordneten Tabellen gibt, für die Satz \ref{haupt4} nicht gilt, so gibt es überhaupt keine 
        Tabellen, für die Satz \ref{haupt4} nicht gilt.
    \end{lem}
    \begin{proof}[Beweis des HS durchh Widerspruch]
        Nehme man für den Widerspruchsbeweis an, es gäbe eine f.p.r.w. Tabelle $T$ mit mehr Spalten als Zeilen, für 
        die Satz \ref{haupt4} nicht gilt, aber keine solchen Tabellen, die wohlgeordnet sind. 

        Sei zunächst $\dot T$ die gleiche Tabelle wie $T$, allerdings mit den Zeilen so umgeordnet, dass sie 
        zeilenweise geordnet ist. Da sich nur die Reihenfolge der Zeilen ändert, gibt es in jeder Spalte immer noch 
        mindestens ein Feld mit einer positiven Zahl. Natürlich gibt es auch immer noch nur Felder mit nicht-negativen 
        Zahlen, also ist $\dot T$ insgesamt f.p.r.w. Außerdem hat $\dot T$ genau wie $T$ mehr Spalten als Zeilen. 
        Wenn für $\dot T$ Satz \ref{haupt4} gelten würde, müsste es ein Feld $(i, j)$ geben, sodass $(i, j)'>0$ 
        und $\dot R(i, 1; i, \dot n) > \dot R(1, j; \dot m, j)$, wobei $'$ die Feld-, $\dot R$ die Rechtecksfunktion, 
        $\dot m$ die Anzahl an Zeilen und $\dot n$ die Anzahl an Spalten von $\dot T$ ist. Sei nun die $l$-te Zeile 
        von $T$ die Zeile, die nach der Umordnung die $i$-te Zeile von $\dot T$ bildet. Der Wert des Feldes $(l, j)$ 
        von $T$ steht also nach der Umordnung im Feld $(i, j)$ von $\dot T$. Da $(i, j)'>0$, ist also:
        \[
            (l, j)^*>0,
        \]
        wobei $^*$ die Feldwertfunktion von $T$ ist. In der $j$-ten Spalte von $\dot T$ befinden sich offensichtlich 
        die gleichen Zahlen wie in der $j$-ten Spalte von $T$; lediglich ihre Reihenfolge ändert sich durch die 
        Umordnung der Zeilen. Ihre Summe (gemäß HS \ref{r_zeile_spalte} durch $R(1, j; m, j)$ bzw. $\dot R(1, j; 
        \dot m, j)$ gegeben) ist also auch die gleiche:
        \[
            R(1, j; m, j)=\dot R(1, j; \dot m, j),
        \]
        wobei $R$ die Rechtecksfunktion $m$ die Anzahl an Zeilen und $n$ die Anzahl an Spalten von $T$ ist. 
        Sämtliche Zahlen der $i$-ten Zeile von $\dot T$ standen vor der Umordnung in der $l$-ten Zeile von $T$. 
        Die Summe der Zahlen in diesen beiden Zeilen muss also die gleiche sein. Es ist also (erneut ergeben sich die 
        folgenden Werte direkt aus HS \ref{r_zeile_spalte}):
        \[
            R(i, 1; i, n)=\dot R(i, 1; i, \dot n)
        \]
        Insgesamt ist also $(l, j)^*>0$, $R(1, j; m, j)=\dot R(1, j; \dot m, j)$ und $R(i, 1; i, n)=\dot R(i, 1; i, 
        \dot n)$. Da $\dot R(i, 1; i, \dot n) > \dot R(1, j; \dot m, j)$, muss also auch $R(i, 1; i, n) > \dot 
        R(1, j; m, j)$. Damit gibt es dann aber das Feld $(i, j)$ der Tabelle $T$, für das $(l, j)^*>0$ und $R(i, 
        1; i, n) > R(1, j; m, j)$. Für die Tabelle $T$ gilt also Satz \ref{haupt4}. Da das nicht sein kann, 
        muss die Annahme, für $\dot T$ würde Satz \ref{haupt4} gelten, falsch sein! Für $\dot T$ gilt Satz 
        \ref{haupt4} also nicht!

        Sei nun $\ddot T$ die gleiche Tabelle wie $\dot T$, allerdings mit den Spalten so umgeordnet, dass $\ddot T$ 
        spaltenweise geordnet ist. Da sich nur die Reihenfolge der Spalten ändert, gibt es in jeder Spalte immer noch 
        mindestens ein Feld mit einer positiven Zahl. Natürlich gibt es auch immer noch nur Felder mit nicht-negativen 
        Zahlen, also ist $\ddot T$ insgesamt f.p.r.w. Außerdem hat $\ddot T$ genau wie $\dot T$ mehr Spalten als Zeilen. 
        Wenn für $\ddot T$ Satz \ref{haupt4} gelten würde, müsste es ein Feld $(i, j)$ geben, sodass $(i, j)^{**}>0$ 
        und $\ddot R(i, 1; i, \ddot n) > \ddot R(1, j; \ddot m, j)$, wobei $^{**}$ die Feld-, $\ddot R$ die 
        Rechtecksfunktion, $\ddot m$ die Anzahl an Zeilen und $\ddot n$ die Anzahl an Spalten von $\ddot T$ ist. 
        Sei nun die $k$-te Spalte von $\dot T$ die Spalte, die nach der Umordnung die $j$-te Spalte von $\ddot T$ 
        bildet. Der Wert des Feldes $(i, k)$ von $\dot T$ steht also nach der Umordnung im Feld $(i, j)$ von 
        $\ddot T$. Da $(i, j)^{**}>0$, ist also:
        \[
            (i, k)^*>0
        \]
        In der $i$-ten Zeile von $\ddot T$ befinden sich offensichtlich die gleichen Zahlen wie in der $i$-ten Zeile 
        von $\dot T$; lediglich ihre Reihenfolge ändert sich durch die Umordnung der Spalten. Ihre Summe (gemäß HS 
        \ref{r_zeile_spalte} durch $\dot R(i, 1; i, \dot n)$ bzw. $\ddot R(i, 1; i, \ddot n)$ gegeben) ist also auch 
        die gleiche:
        \[
            \dot R(i, 1; i, \dot n)=\ddot R(i, 1; i, \ddot n),
        \]
        wobei $\ddot R$ die Rechtecksfunktion von $\ddot T$ ist.
        Sämtliche Zahlen der $j$-ten Spalte von $\ddot T$ standen vor der Umordnung in der $k$-ten Spalte von $\dot T$. 
        Die Summe der Zahlen in diesen beiden Zeilen muss also die gleiche sein. Es ist also (erneut ergeben sich die 
        folgenden Werte direkt aus HS \ref{r_zeile_spalte}):
        \[
            \dot R(1, k; \dot m, k)=\ddot R(1, j; \ddot m, j)
        \]
        Insgesamt ist also $(i, k)^*>0$, $\dot R(i, 1; i, \dot n)=\ddot R(i, 1; i, \ddot n)$ und $\dot R(1, k; 
        \dot m, k)=\ddot R(1, j; \ddot m, j)$. Da $\ddot R(i, 1; i, \ddot n)>\ddot R(1, j; \ddot m, j)$, muss auch 
        $\dot R(i, 1; i, \dot n)>\dot R(1, k; \dot m, k)$. Damit gibt es aber ein Feld $(i, k)$ mit $(i, k)^*>0$ und 
        $\dot R(i, 1; i, \dot n)>\dot R(1, k; \dot m, k)$. Für $\dot T$ gilt also Satz \ref{haupt4}. Das kann aber--wie 
        oben gezeigt--nicht sein. Die Annahme, für $\ddot T$ würde Satz \eqref{haupt4} gelten, führt also zu einem 
        Widerspruch, d.h. sie muss falsch sein! Für $\ddot T$ kann Satz \ref{haupt4} also nicht gelten!

        Während der Umordnung der Spalten von $\dot T$, um $\ddot T$ zu erhalten, wurden offensichtlich die Summen der 
        Zahlen in den Spalten der Tabelle nicht geändert; jedes Feld blieb ja in der gleichen Zeile, da nur die 
        Spalten entlang der Zeilen umgeordnet wurden. Also ist $\ddot T$, da $\dot T$ zeilenweise geordnet ist, 
        ebenfalls zeilenweise geordnet. Nun ist $\ddot T$ aber auch spaltenweise geordnet, also ist $\ddot T$ 
        wohlgeordnet!

        $\ddot T$ ist also eine wohlgeordnete Tabelle, für die Satz \ref{haupt4} nicht gilt. Nun wurde am Anfang dieses 
        Widerspruchsbeweises angenommen, es gäbe zwar Tabellen, für die Satz \ref{haupt4} nicht gilt, allerdings keine 
        solchen wohlgeordneten. Da diese Annahme zu einem Widerspruch--nämlich einer wohlgeordneten Tabelle, für die 
        Satz \ref{haupt4} nicht gilt--geführt hat, muss sie falsch sein! Wenn es also keine wohlgeordneten Tabellen 
        gibt, für die Satz \eqref{haupt4} nicht gilt, kann es überhaupt keine Tabellen geben, für die Satz \ref{haupt4} 
        nicht gilt!
    \end{proof}
    Da es nach HS \ref{fallbeschrank} überhaupt keine Tabellen gibt, für die Satz \ref{haupt4} 
    nicht gilt, wenn es keine wohlgeordneten gibt, genügt es, zu beweisen, dass es keine wohlgeordneten Tabellen 
    gibt, für die Satz \ref{haupt4} nicht gilt, um zu beweisen, dass es überhaupt keine gibt. Es folgt also ein 
    Widerspruchsbeweis dafür, dass es keine wohlgeordneten Tabellen gibt, für die Satz \ref{haupt4} nicht gilt:

    Nehme man für den Widerspruchsbeweis an, es gäbe eine wohlgeordnete, f.p.r.w. Tabelle $T^*$ mit $m$ Zeilen und $n$ 
    Spalten, wobei $m<n$, für die Satz \ref{haupt4} nicht gilt; in der es also kein Feld $(i, j)$ mit $(i, j)'>0$ 
    und $R(i, 1; i, n)>R(1, j; m, j)$ gibt, wobei $'$ die Feldwert- und $R$ die Rechtecksfunktion der Tabelle $T^*$ 
    ist. Dann folgt mit vollständiger Induktion, dass $R(m-t, 1; m, n)\leq R(1, n-t; m, n)$ für alle $t$ mit $0\leq 
    t<m$:

    \textit{Induktionsanfang ($t=0$)}: Es folgt ein Beweis durch Widerspruch. Nehme man hierzu an, es sei nicht 
    $R(m-t, 1; m, n)\leq R(1, n-t; m, n)$ für $t=0$, also $R(m, 1; m, n)>R(1, n; m, n)$. Nach 
    HS \ref{mehr_nuller} (mit $i=m$, $j=n$, $b=0$) muss dann auch $R(m-a, 1; m-a, n) > R(1, n; m, n)$ für alle 
    $a$ mit $0\leq a<m$, denn nur für diese gibt es eine $(m-a)$-te Zeile. Nun sind die Felder $(m-a, n)$ für $0\leq 
    a<m$ genau die Felder der $n$-ten Spalte (ein beliebiges Feld der $n$-ten Spalte $(j, n)$ ergibt sich mit $a=m-j$, 
    was wegen $j>0$ kleiner als $m$ und wegen $j\leq m$ nicht negativ sein kann). Natürlich muss in mindestens einem 
    der Felder der $n$-ten Spalte eine positive Zahl stehen, da $T^*$ f.p.r.w. ist. Es gibt also ein Feld $(j, n)$ 
    mit $(j, n)'>0$ und $R(j, 1; j, n)>R(1, n; m, n)$. Das stellt aber einen Widerspruch dazu dar, dass für $T^*$ nicht 
    Satz \ref{haupt4} gilt! Da aus $R(m, 1; m, n)>R(1, n; m, n)$ ein Widerspruch folgt, kann nicht $R(m, 1; m, n)>
    R(1, n; m, n)$ sein; es muss also $R(m, 1; m, n)\leq R(1, n; m, n)$. Das entspricht $R(m-t, 1; m, n)\leq 
    R(1, n-t; m, n)$ mit $t=0$, die Aussage gilt also für $t=0$.

    \textit{Induktionsschritt}: (Beweis, dass aus der Induktionsannahme, $R(m-t, 1; m, n)\leq R(1, n-t; m, n)$ für 
    ein $t$ mit $0\leq t<m-1$, folgt, dass $R(m-(t+1), 1; m, n)\leq R(1, n-(t+1); m, n)$) Nehme man hierzu an, es sei 
    $R(m-(t+1), 1; m-(t+1), n)>R(1, n-(t+1); m, n-(t+1))$. Dann muss gemäß HS \ref{mehr_nuller} auch 
    $R()$

    % TODO


    Mit dem Induktionsprinzip (Gültigkeit der Aussage für $t=0$ wurde im Induktionsanfang bewiesen, mit dem 
    Induktionsschritt folgt dann Gültigkeit für $t=1$, damit dann auch für $t=2$, \dots, für $t=m$) folgt also, dass 
    $R(m-t, 1; m, n)\leq R(1, n-t; m, n)$ für alle $t$ mit $0\leq t<m$. Insbesondere ist also auch $R(m-t, 1; m, n)
    \leq R(1, n-t; m, n)$ für $t=m-1$, also $R(m-(m-1), 1; m, n)\leq R(1, n-(m-1); m, n)\Leftrightarrow R(1, 1; 
    m, n)\leq R(1, n-m+1; m, n)$. Nun ist nach HS \ref{r_summe} (mit $a_1=1, a_2=1, b_1=n-m, b_2=n$  und 
    $c_2=m$) aber (offensichtlich sind die für die Gültigkeit von HS \ref{r_summe} Ungleichungen mit diesen 
    Werten gegeben):
    \[
    R(1, 1; m, n)=R(1, 1; m, n-m)+R(1, n-m+1; m, n)
    \]
    Also ist, da $R(1, 1; m, n)\leq R(1, n-m+1; m, n)$, auch:
    \begin{alignat*}{2}
        &R(1, 1; m, n-m)+R(1, n-m+1; m, n)&&\leq R(1, n-m+1; m, n)\\
        \Leftrightarrow &R(1, 1; m, n-m)&&\leq 0
    \end{alignat*}
    Nun kann nicht $n<m+1$, denn wegen $m<n$ wäre dann auch $m+1<n+1$ und somit $n<m+1<n+1$, also 
    $n<n+1$, was nicht sein kann. Es ist demnach $n\geq m+1$, also entweder $n=m+1$ oder $n>m+1$. Nehme man zunächst 
    an, es sei $n>m+1$. Dann ist gemäß HS \ref{r_summe} (mit $a_1=1, a_2=1, b_1=m, b_2=1$ und $c_2=n-m$) 
    offenbar $R(1, 1; m, n-m)=R(1, 1; m, 1)+R(1, 2; m, n-m)$. Hierbei ist wegen $n>m+1$ natürlich $n-m>1$, also 
    $n-m\geq 2$; das ist ja notwendige Bedingung für $R(1, 2; n, n-m)$. Da nun $R(1, 1; m, n-m)\leq 0$, ist auch 
    $R(1, 1; m, 1)+R(1, 2; m, n-m)\leq 0$ und damit $R(1, 1; m, 1)\leq -R(1, 2; m, n-m)$. Weil die Tabelle $T^*$ 
    f.p.r.w. ist, stehen in ihren Feldern nur nicht-negative Zahlen. Eine Summe von nicht-negativen Zahlen kann 
    natürlich nicht negativ sein. Da $R(1, 2; m, n-m)$ der Wert einer solchen Summe ist, muss also $R(1, 2; m, n-m)
    \geq 0$ und damit $-R(1, 2; m, n-m)\leq 0$ gelten. Es ist also:
    \[
    R(1, 1; m, 1)\leq -R(1, 2; m, n-m)\leq 0
    \]
    Insbesondere also $R(1, 1; m, 1)\leq 0$. Wegen $R(1, 1; m, n-m)\leq 0$ gilt, wenn $n=m+1$, $R(1, 1; m, (m+1)-m)
    \leq 0$, also $R(1, 1; m, 1)\leq 0$. Es gilt also stets $R(1, 1; m, 1)\leq 0$, egal, ob $n>m+1$ oder $n=m+1$. 
    Nun ist $R(1, 1; m, 1)$ nach HS \ref{r_zeile_spalte} die Summe der Werte in den Feldern der ersten Spalte. Da 
    all diese Werte nicht-negativ sind, muss auch $R(1, 1; m, 1)$ nicht-negativ sein. Weil gleichzeitig aber auch 
    $R(1, 1; m, 1)\leq 0$ ist, kann $R(1, 1; m, 1)$ nicht positiv sein. Es muss also:
    \[
    R(1, 1; m, 1)= 0
    \]
    Dafür müssen dann all die Werte, die in den Feldern der ersten Spalte stehen, 0 sein, da unter ihnen keine 
    negativen sind und gleichzeitig ihre Summe 0 ist. Nun ist aber $T^*$ eine f.p.r.w. Tabelle, also muss es in jeder 
    Spalte von $T^*$ mindestens ein Feld geben, in dem eine positive Zahl steht. Aber ein solches Feld gibt es in 
    der ersten Spalte nicht! Die Annahme, es gäbe eine Tabelle, für die Satz \ref{haupt4} nicht gilt, führt also zu 
    einem Widerspruch. Eine solche Tabelle gibt es also nicht, was den Widerspruchsbeweis von Satz \ref{haupt4} vollendet.
    \renewcommand{\qedsymbol}{$\blacksquare$}
\end{proof}

\end{document}