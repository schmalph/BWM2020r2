\section{Aufgabe 4}

\subsection*{Einführung hilfreicher Begriffe}

Seien zunächst folgende Begriffe eingeführt, die den Beweis hoffentlich kürzer, aber auch leichter lesbar machen:
\begin{definition}
    Mit einer "`proportionierten"' Tabelle ist eine Tabelle gemeint, die mehr Spalten als Zeilen hat.
\end{definition}
\begin{definition}
    Mit einer "`fast-positiv-reell-wertigen"' (kurz f.p.r.w.) Tabelle ist eine Tabelle gemeint, für die gilt, dass 
    in jedem ihrer Felder eine nicht-negative reelle Zahl steht, und dabei in jeder Spalte der Tabelle mindestens 
    eine positive.
\end{definition}
\begin{definition}
    Das Feld $(i, j)$ einer Tabelle ist das Feld in der $i$-ten Zeile und der $j$-ten Spalte dieser Tabelle.
\end{definition}
\begin{definition}
    Wenn $'$ die Feldwertfunktion einer Tabelle ist, ist $(i, j)'$ der Wert, der im Feld $(i, j)$ dieser Tabelle 
    steht.
\end{definition}
\begin{definition}\label{rfunktion}
    Wenn $R$ die Rechtecksfunktion und $'$ die Feldwertfunktion einer Tabelle ist, ist $R(a_1, b_1; a_2, b_2)$ kurz 
    für die folgende Summe, wobei $a_1\leq b_1$ und $a_2\leq b_2$ gelten muss:
    \begin{alignat*}{6}
        R(a_1, a_2; b_1, b_2) &= (a_1, &&a_2)' + &&(a_1, &&a_2+1)' + \quad \cdots \quad + &&(a_1, &&b_2)'\\
        &+(a_1+1, &&a_2)' + &&(a_1+1, &&a_2+1)' +\quad \cdots \quad + &&(a_1+1, &&b_2)'\\
        &+(a_1+2, &&a_2)' + &&(a_1+2, &&a_2+1)' +\quad \cdots \quad + &&(a_1+2, &&b_2)'\\
        & &&\vdotswithin{a} && &&\vdotswithin{a}\quad\quad\quad\quad\quad \ddots && &&\vdotswithin{b} \\
        &+(b_1-1, &&a_2)' + &&(b_1-1, &&a_2+1)' +\quad \cdots \quad + &&(b_1-1, &&b_2)'\\
        &+(b_1, &&a_2)' + &&(b_1, &&a_2+1)' + \quad \cdots \quad + &&(b_1, &&b_2)
    \end{alignat*}    
\end{definition}
\begin{definition}
    Mit einer "`spaltenweise geordneten"' Tabelle ist eine Tabelle gemeint, für die stets $R(s, 1; s, m)\geq 
    R(s+1, 1; s+1, m)$ gilt, wobei $m$ die Anzahl an Zeilen dieser Tabelle ist. Äquivalent ist mit einer 
    "`zeilenweise geordneten"' Tabelle eine gemeint, für die stets $R(z, 1; z, n)\geq R(z+1, 1; z+1, n)$ gilt, 
    wobei $n$ die Anzahl an Spalten dieser Tabelle ist. Ist eine Tabelle zeilen- und spaltenweise geordneten, heißt 
    sie kurz auch "`wohlgeordnet"'.
\end{definition}

\subsection*{Sätze über die eingeführten Begriffe (insb. Rechtecksfunktion)}

\begin{thm}\label{r_summe}
    Wenn $R$ die Rechtecksfunktion einer Tabelle ist, gilt, wobei $a_1\leq b_1\leq c_1$ und $a_2\leq b_2\leq c_2$: 
    $R(a_1, a_2; b_1, b_2)+R(b_1+1, a_2; c_1, b_2)=R(a_1, a_2; c_1, b_2)$ und 
    $R(a_1, a_2; b_1, b_2)+R(a_1, b_2+1; b_1, c_2)=R(a_1, a_2; b_1, c_2)$.
\end{thm}
\begin{proof}
    beweis
\end{proof}

\subsection*{Beweis der zu zeigenden Ausssage von Aufgabe 4}

\begin{thm}\label{haupt4}
    In jeder proportionierten und f.p.r.w. Tabelle gibt es ein Feld $(i, j)$, sodass $(i, j)'>0$ und $Z(i) > S(j)$, 
    wobei $'$ die Feld- und $R$ die Rechtecksfunktion dieser Tabelle ist.
\end{thm}

\begin{proof}
    Um den Kern des Beweises besser erkennbar zu machen, lohnt es sich, zunächst ein paar Lemmata zu beweisen:
    \renewcommand{\qedsymbol}{$\square$}
    \begin{lem}\label{fallbeschrank}
        Wenn es keine wohlgeordneten Tabellen gibt, für die Satz \ref{haupt4} nicht gilt, so gibt es überhaupt keine 
        Tabellen, für die Satz \ref{haupt4} nicht gilt.
    \end{lem}
    \begin{proof}
        beweis von diesem lemma (zuerst zeilen umordnen, dann spalten umordnen und gültigkeit von satz 3 ändert sich 
        dabei nicht) %TODO
    \end{proof}
    \begin{lem}\label{rechteckkke}
        Wenn in einer wohlgeordneten Tabelle $T$ für zwei Zahlen $i, j$, wobei $1 \leq i \leq m_T$ und $1 \leq j \leq 
        n_T$, $Z(i) > S(j)$ gilt, so gilt auch $Z(i-a) > S(j+b)$ für alle $a, b$ mit $0 \leq a < i$ und $0 \leq b 
        \leq n_T-j$. Umgekehrt folgt aus $Z(i) \leq S_T(j)$, dass für alle $c,d$ mit $0 \leq c \leq m_T$ und
        $0 \leq d < j$ gilt: $Z(i+c) \leq S(j-d)$. 
    \end{lem}
    \begin{proof}
        beweis für dieses lemma (folgt direkt aus wohlgeordnetheit von $T$)%TODO
    \end{proof}
    Nun folgt das Herzstück dieses Beweises, denn sobald dieses bewiesen ist, folgt Satz \ref{haupt4} quasi von selbst 
    mit vollständiger Induktion:
    \begin{lem}
        Wenn es eine wohlgeordnete fast-positiv-reelwertige Tabelle gibt, für die Satz \ref{haupt4} nicht gilt, so 
        
    \end{lem}
    \begin{proof}
        Nehme man zunächst an, $T^*$ sei eine solche Tabelle mit $m_{T^*}=m^*$ Zeilen und $n_{T^*}n^*$ Spalten.\\
        \underline{Induktionsanfang ($t=0$)}: Wenn $S^*(n^*) < Z^*(m^*)$ wäre, müsste nach Lemma \ref{rechteckkke} 
        auch $S^*(n^*) < Z^*(m^*-a)$ für alle $a$ mit $0 \leq a \leq m$ gelten. Da $T^*$ aber eine Tabelle 
        sein soll, für die Satz \ref{haupt4} nicht gilt, müssten dann alle Felder $T^*(m^*-a, n^*)$ den Wert 0 haben. 
        Da dies aber alle Felder der $n^*$-ten Spalte sind, müssten sie gleichzeitig auch mindestens ein Feld 
        mit einem positiven Wert haben ($T^*$ ist ja fast-positiv-reell-wertig). Da aus der Annahme $S^*(n^*) < Z^*(m)$ 
        also ein Widerspruch folgt, muss gelten: $S^*(n^*-t) \leq Z^*(m^*-t)$ für $t=0$.\\
        \underline{Induktionsschritt (von $t$ zu $t+1$)}: (Unter der Induktionsannahme, dass $S^*(n^*-r) \geq 
        Z^*(m^*-r)$ für alle $r$ mit $0 \leq r \leq t$) Nehme man nun an, dass $S^*(n^*-(t+1)) < Z^*(m^*-(t+1))$. 
        Da $T^*$ wohlgeordnet und fast-positiv-reell-wertig ist, folgt dann aus Lemma \ref{rechteckkke}, 
        dass auch $S^*(n^*-(t+1)+b) < Z^*(m^*-(t+1)-a)$ für alle $a, b$ mit $0\leq a\leq m^*-(t+1)$ und 
        $0\leq b\leq t+1$. Nun würde aber für $T^*$ Satz \ref{haupt4} gelten, wenn es zwei Zahlen $a', b'$ mit 
        $0\leq a'\leq m^*-(t+1)$ und $0\leq b'\leq t+1$ geben würde, für die $f^*(m^*-(t+1)-a', n^*-(t+1)+b') >0$, 
        denn für diese wäre dann ja auch $S^*(n^*-(t+1)+b) < Z^*(m^*-(t+1)-a)$. Da die Tabelle $T^*$ aber ein 
        Gegenbeispiel für Satz \ref{haupt4} darstellen soll, kann das nicht sein und es muss damit für alle $a, b$ 
        mit $0\leq a\leq m^*-(t+1)$ und $0\leq b\leq t+1$ gelten: $f^*(m^*-(t+1)-a, n^*-(t+1)+b)=0$, wobei $f^*$ die 
        Feldwertfunktion von $T^*$ ist. Sei nun $A=m^*-(t+1)-a$ und $B=(t+1)-b$, dann muss wegen $0\leq a\leq m^*-(t+1)$ 
        für $A=m^*-(t+1)-a$ gelten: $0\leq A\leq m^*-(t+1)$. Und wegen $0\leq b\leq t+1$ muss für $B=(t+1)-b$ gelten: 
        $0 \leq B\leq t+1$. Es ist also $f^*(A, n^*-B) =0$ für alle $A,B$ mit $0\leq A\leq m^*-(t+1)$ und $0 \leq B
        \leq t+1$.\\
        Nun lässt sich die Induktionsannahme ($S^*(n^*-r) \geq Z^*(m^*-r)$ für alle $r$ mit $0 \leq r \leq t$) wegen 
        $S^*(n^*-r)=f^*(1,n^*-r) + f^*(2,n^*-r)+\cdots +f^*(m^*,n^*-r)$ und $Z^*(m^*-r)=f^*(m^*-r, 1) + f^*(m^*-r, 2) 
        + \cdots + f^*(m^*-r, n^*)$ zu Folgender Umformen:
        \begin{align*}
            &f^*(1,n^*-r)+f^*(2,n^*-r)+\cdots+f^*(m^*,n^*-r)\\ 
            \geq &f^*(m^*-r, 1)+f^*(m^*-r, 2)+\cdots +f^*(m^*-r, n^*)\text{, wobei $0 \leq r \leq t$}
        \end{align*}

    \end{proof}
    noch mehr beweissssssss
    \renewcommand{\qedsymbol}{$\blacksquare$}
\end{proof}