\section{Aufgabe 4}

\subsection*{Einführung hilfreicher Begriffe}

Seien zunächst folgende Begriffe eingeführt, die den Beweis hoffentlich kürzer, aber auch leichter lesbar machen:

[Im Folgenden ist die $k$-te Spalte immer die $k$-te Spalte von links und die $k$-Zeile immer die $k$-te Zeile von oben.]
\begin{definition}
    Mit einer "`fast-positiv-reell-wertigen"' (kurz f.p.r.w.) Tabelle ist eine Tabelle gemeint, für die gilt, dass 
    in jedem ihrer Felder eine nicht-negative reelle Zahl steht, und dabei in jeder Spalte der Tabelle mindestens 
    eine positive.
\end{definition}
\begin{definition}
    Das Feld $(i, j)$ einer Tabelle ist das Feld in der $i$-ten Zeile und der $j$-ten Spalte dieser Tabelle.
\end{definition}
\begin{definition}
    Wenn $'$ die Feldwertfunktion einer Tabelle ist, ist $(i, j)'$ der Wert, der im Feld $(i, j)$ dieser Tabelle 
    steht.
\end{definition}
\begin{definition}\label{rfunktion}
    Wenn $R$ die Rechtecksfunktion und $'$ die Feldwertfunktion einer Tabelle ist, ist $R(a_1, b_1; a_2, b_2)$ kurz 
    für die folgende Summe, wobei $a_1\leq b_1$ und $a_2\leq b_2$ gelten muss:
    \begin{alignat*}{6}
        R(a_1, a_2; b_1, b_2) &= (a_1, &&a_2)' + &&(a_1, &&a_2+1)' + \quad \cdots \quad + &&(a_1, &&b_2)'\\
        &+(a_1+1, &&a_2)' + &&(a_1+1, &&a_2+1)' +\quad \cdots \quad + &&(a_1+1, &&b_2)'\\
        &+(a_1+2, &&a_2)' + &&(a_1+2, &&a_2+1)' +\quad \cdots \quad + &&(a_1+2, &&b_2)'\\
        & &&\vdotswithin{a} && &&\vdotswithin{a}\quad\quad\quad\quad\quad \ddots && &&\vdotswithin{b} \\
        &+(b_1-1, &&a_2)' + &&(b_1-1, &&a_2+1)' +\quad \cdots \quad + &&(b_1-1, &&b_2)'\\
        &+(b_1, &&a_2)' + &&(b_1, &&a_2+1)' + \quad \cdots \quad + &&(b_1, &&b_2)
    \end{alignat*}    
\end{definition}
\begin{definition}
    Mit einer "`spaltenweise geordneten"' Tabelle ist eine Tabelle gemeint, für die stets $R(s, 1; s, m)\geq 
    R(s+1, 1; s+1, m)$ gilt, wobei $m$ die Anzahl an Zeilen dieser Tabelle ist. Äquivalent ist mit einer 
    "`zeilenweise geordneten"' Tabelle eine gemeint, für die stets $R(z, 1; z, n)\geq R(z+1, 1; z+1, n)$ gilt, 
    wobei $n$ die Anzahl an Spalten dieser Tabelle ist. Ist eine Tabelle zeilen- und spaltenweise geordneten, heißt 
    sie kurz auch "`wohlgeordnet"'.
\end{definition}

\subsection*{Sätze über die eingeführten Begriffe (insb. Rechtecksfunktion)}

\renewcommand{\qedsymbol}{$\square$}

\begin{lem}\label{r_summe}
    Wenn $R$ die Rechtecksfunktion einer Tabelle ist und $a_1\leq b_1\leq c_1$ und $a_2\leq b_2\leq c_2$ ist, gilt: 
    $R(a_1, a_2; b_1, b_2)+R(b_1+1, a_2; c_1, b_2)=R(a_1, a_2; c_1, b_2)$ und 
    $R(a_1, a_2; b_1, b_2)+R(a_1, b_2+1; b_1, c_2)=R(a_1, a_2; b_1, c_2)$.
\end{lem}
\begin{proof}[Beweis des Hilfssatzes]
    beweis
    % TODO: einfach für beide R-FUnktionen die Definiiton einsetzen
\end{proof}
\begin{lem}\label{r_zeile_spalte}
    Wenn $R$ die Rechtecksfunktion einer Tabelle mit $m$ Zeilen und $n$ Spalten ist, ist $R(i, 1; i, n)$ die Summe 
    der Werte in den Feldern der $i$-ten Zeile und $R(1, j; m, j)$ die Summe der Werte in den Feldern der $j$-ten 
    Spalte.
\end{lem}
\begin{proof}[Beweis des Hilfssatzes]
    beweis
    % TODO: Felder in R mit Feldern der Zeile/Spalte vergleichen  (werden die gleichen sein)
\end{proof}
\begin{lem}\label{mehr_nuller}
    Wenn in einer wohlgeordneten Tabelle $T$ mit Rechtecksfunktion $R$ für zwei Zahlen $i, j$, $R(i, 1; i, n)>
    R(1, j; m, j)$ gilt, so gilt auch $R(i-a, 1; i-a, n)>R(1, j+b; m, j+b)$ für alle $a, b\geq0$, für die es eine 
    $(i-a)$-te Zeile und eine $(j+b)$-te Spalte gibt. Umgekehrt folgt aus $R(i, 1; i, n)\leq R(1, j; m, j)$, dass für 
    alle $c,d\geq0$, für die es eine $(i+c)$-te Zeile und  eine $(j-d)$-te Spalte gibt, gilt: $R(i+c, 1; i+c, n)\leq 
    R(1, j-d; m, j-d)$. 
\end{lem}
\begin{proof}[Beweis des Hilfssatzes]
    folgt aus wohlgeordnetheit % TODO
\end{proof}

\subsection*{Beweis der zu zeigenden Ausssage von Aufgabe 4}

\begin{thm}\label{haupt4}
    In jeder f.p.r.w. Tabelle mit $m$ Zeilen und $n$ Spalten, wobei $m>n$, gibt es ein Feld $(i, j)$, sodass 
    $(i, j)'>0$ und $R(i, 1; i, n) > R(1, j; m, j)$, wobei $'$ die Feld- und $R$ die Rechtecksfunktion dieser 
    Tabelle ist.
\end{thm}

\begin{proof}[Beweis des Satzes durch Widerspruch]
    Nehme man zunächst an, es gäbe eine Tabelle, für die Satz \ref{haupt4} nicht gilt. Nach folgendem Lemma kann man 
    davon ausgehen, dass diese Tabelle wohlgeordnet ist.
    \begin{lem}\label{fallbeschrank}
        Wenn es keine wohlgeordneten Tabellen gibt, für die Satz \ref{haupt4} nicht gilt, so gibt es überhaupt keine 
        Tabellen, für die Satz \ref{haupt4} nicht gilt.
    \end{lem}
    \begin{proof}[Beweis des Hilfssatzes]
        beweis von diesem lemma (zuerst zeilen umordnen, dann spalten umordnen und gültigkeit von satz 3 ändert sich 
        dabei nicht) %TODO
    \end{proof}
    Da es nach Lemma \ref{fallbeschrank} überhaupt keine Tabellen gibt, für die Satz \ref{haupt4} 
    nicht gilt, wenn es keine wohlgeordneten gibt, genügt es, zu beweisen, dass es keine wohlgeordneten Tabellen 
    gibt, für die Satz \ref{haupt4} nicht gilt, um zu beweisen, dass es überheupt keine gibt. Es folgt also ein 
    Widerspruchsbeweis dafür, dass es keine wohlgeordneten Tabellen gibt, für die Satz \ref{haupt4} nicht gilt:

    Nehme man für den Widerspruchsbeweis an, es gäbe eine wohlgeordnete, f.p.r.w. Tabelle $T^*$ mit $m$ Zeilen und $n$ 
    Spalten, wobei $m>n$, für die Satz \ref{haupt4} nicht gilt; in der es also kein Feld $(i, j)$ mit $(i, j)'>0$ 
    und $R(i, 1; i, n)>R(1, j; m, j)$ gibt, wobei $'$ die Feldwert- und $R$ die Rechtecksfunktion der Tabelle $T^*$ 
    ist. Dann folgt mit vollständiger Induktion, dass $R(m-t, 1; m, n)\leq R(1, n-t; m, n)$ für alle $t$ mit $0\leq 
    t<n$:

    \textit{Induktionsanfang ($t=0$)}: Es folgt ein beweis durch Widerspruch. Nehme man hierzu an, es sei nicht 
    $R(m-t, 1; m, n)\leq R(1, n-t; m, n)$ für $t=0$, also $R(m, 1; m, n)>R(1, n; m, n)$. Nach dem ersten Satz von 
    Hilfssatz \ref{mehr_nuller} (mit $i=m$, $j=n$, $b=0$) muss dann auch $R(m-a, 1; m-a, n) > R(1, n; m, n)$ für alle 
    $a$ mit $0\leq a<m$, denn nur für diese gibt es eine $(m-a)$-te Zeile. Nun sind die Felder $(m-a, n)$ für $0\leq 
    a<m$ genau die Felder der $n$-ten Spalte (ein beliebiges Feld der $n$-ten Spalte $(j, n)$ ergibt sich mit $a=m-j$, 
    was wegen $j>0$ kleiner als $m$ und wegen $j\leq m$ nicht negativ sein kann). Natürlich muss in mindestens einem 
    der Felder der $n$-ten Spalte eine positive Zahl stehen, da $T^*$ f.p.r.w. ist. Es gibt also ein Feld $(j, n)$ 
    mit $(j, n)'>0$ und $R(j, 1; j, n)>R(1, n; m, n)$. Das stellt aber einen Widerspruch dazu dar, dass für $T^*$ nicht 
    Satz \ref{haupt4} gilt! Da aus $R(m, 1; m, n)>R(1, n; m, n)$ ein Widerspruch folgt, kann nicht $R(m, 1; m, n)>
    R(1, n; m, n)$ sein; es muss also $R(m, 1; m, n)\leq R(1, n; m, n)$. Das entspricht $R(m-t, 1; m, n)\leq 
    R(1, n-t; m, n)$ mit $t=0$, die Aussage gilt also für $t=0$.


    \textit{Induktionsschritt}: (Beweis, dass aus der Induktionsannahme, $R(m-t, 1; m, n)\leq R(1, n-t; m, n)$ für 
    ein $t$ mit $0\leq t<n-1$, folgt, dass $R(m-(t+1), 1; m, n)\leq R(1, n-(t+1); m, n)$) Nehme man nun an, es sei 






    noch mehr beweissssssss
    \renewcommand{\qedsymbol}{$\blacksquare$}
\end{proof}