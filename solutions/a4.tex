\section{Aufgabe 4}

\subsection*{Einführung hilfreicher Begriffe}

Seien zunächst folgende Begriffe eingeführt, die den Beweis hoffentlich kürzer, aber auch besser lesbar machen:

[Im Folgenden ist die $k$-te Spalte immer die $k$-te Spalte von links und die $k$-Zeile immer die $k$-te Zeile von oben.]
\begin{definition}
    Mit einer "`fast-positiv-reell-wertigen"' (kurz f.p.r.w.) Tabelle ist eine Tabelle gemeint, für die gilt, dass 
    in jedem ihrer Felder eine nicht-negative reelle Zahl steht, und dabei in jeder Spalte der Tabelle mindestens 
    eine positive.
\end{definition}
\begin{definition}
    Das Feld $(i, j)$ einer Tabelle ist das Feld in der $i$-ten Zeile und der $j$-ten Spalte dieser Tabelle.
\end{definition}
\begin{definition}
    Wenn $'$ die Feldwertfunktion einer Tabelle ist, ist $(i, j)'$ der Wert, der im Feld $(i, j)$ dieser Tabelle 
    steht.
\end{definition}
\begin{definition}\label{rfunktion}
    Wenn $R$ die Rechtecksfunktion und $'$ die Feldwertfunktion einer Tabelle ist, ist $R(a_1, b_1; a_2, b_2)$ kurz 
    für die folgende Summe, wobei $a_1\leq b_1$ und $a_2\leq b_2$ gelten muss:
    \begin{alignat*}{6}
        R(a_1, a_2; b_1, b_2) &= (a_1, &&a_2)' + &&(a_1, &&a_2+1)' + \quad \cdots \quad + &&(a_1, &&b_2)'\\
        &+(a_1+1, &&a_2)' + &&(a_1+1, &&a_2+1)' +\quad \cdots \quad + &&(a_1+1, &&b_2)'\\
        &+(a_1+2, &&a_2)' + &&(a_1+2, &&a_2+1)' +\quad \cdots \quad + &&(a_1+2, &&b_2)'\\
        & &&\vdotswithin{a} && &&\vdotswithin{a}\quad\quad\quad\quad\quad \ddots && &&\vdotswithin{b} \\
        &+(b_1-1, &&a_2)' + &&(b_1-1, &&a_2+1)' +\quad \cdots \quad + &&(b_1-1, &&b_2)'\\
        &+(b_1, &&a_2)' + &&(b_1, &&a_2+1)' + \quad \cdots \quad + &&(b_1, &&b_2)'
    \end{alignat*}    
\end{definition}
\begin{definition}
    Mit einer "`spaltenweise geordneten"' Tabelle ist eine Tabelle gemeint, für die stets $R(s, 1; s, m)\geq 
    R(s+1, 1; s+1, m)$ gilt, wobei $m$ die Anzahl an Zeilen dieser Tabelle ist. Äquivalent ist mit einer 
    "`zeilenweise geordneten"' Tabelle eine gemeint, für die stets $R(z, 1; z, n)\geq R(z+1, 1; z+1, n)$ gilt, 
    wobei $n$ die Anzahl an Spalten dieser Tabelle ist. Ist eine Tabelle zeilen- und spaltenweise geordneten, heißt 
    sie kurz auch "`wohlgeordnet"'.
\end{definition}

\subsection*{Hilfssätze über die eingeführten Begriffe}

\renewcommand{\qedsymbol}{$\square$}

\begin{lem}\label{r_summe}
    Wenn $R$ die Rechtecksfunktion einer Tabelle ist und $a_1\leq b_1\leq c_1$ und $a_2\leq b_2\leq c_2$ ist, gilt: 
    $R(a_1, a_2; b_1, b_2)+R(b_1+1, a_2; c_1, b_2)=R(a_1, a_2; c_1, b_2)$ und 
    $R(a_1, a_2; b_1, b_2)+R(a_1, b_2+1; b_1, c_2)=R(a_1, a_2; b_1, c_2)$.
\end{lem}
\begin{proof}[Beweis des HS]
    Erstere Aussage ergibt sich direkt aus der Definition der Rechtecksfunktion:
    \begin{alignat*}{3}
        &R(a_1, a_2; b_1, b_2)&& +R(b_1+1, a_2; c_1, b_2)&&\\
        &=(a_1, a_2)' &&+ (a_1, a_2+1)' &&+\cdots+ (a_1, b_2)'\\
        &+(a_1+1, a_2)' &&+ (a_1+1, a_2+1)' &&+\cdots+ (a_1+1, b_2)'\\
        & &&\vdots\quad\quad\quad\quad&&\vdots\\
        &+(b_1-1, a_2)' &&+ (b_1-1, a_2+1)' &&+\cdots + (b_1-1, b_2)'\\
        &+(b_1, a_2)' &&+ (b_1, a_2+1)' &&+ \cdots + (b_1, b_2)'\\
        &+R(b_1+1, a_2; &&c_1, b_2) &&\\
        &=(a_1, a_2)' &&+ (a_1, a_2+1)' &&+\cdots+ (a_1, b_2)'\\
        &+(a_1+1, a_2)' &&+ (a_1+1, a_2+1)' &&+\cdots+ (a_1+1, b_2)'\\
        & &&\vdots\quad\quad\quad\quad&&\vdots\\
        &+(b_1-1, a_2)' &&+ (b_1-1, a_2+1)' &&+\cdots + (b_1-1, b_2)'\\
        &+(b_1, a_2)' &&+ (b_1, a_2+1)' &&+ \cdots + (b_1, b_2)'\\
        &+(b_1+1, a_2)' &&+ (b_1+1, a_2+1)' &&+\cdots+ (b_1+1, b_2)'\\
        &+(b_1+2, a_2)' &&+ (b_1+2, a_2+1)' &&+\cdots+ (b_1+2, b_2)'\\
        & &&\vdots\quad\quad\quad\quad&&\vdots\\
        &+(c_1-1, a_2)' &&+ (c_1-1, a_2+1)' &&+\cdots + (c_1-1, b_2)'\\
        &+(c_1, a_2)' &&+ (c_1, a_2+1)' &&+ \cdots + (c_1, b_2)'\\
        &=R(a_1, a_2; c_1, &&b_2)&&
    \end{alignat*}
    Auch letztere Aussage ergibt sich direkt durch Einsetzen der Definition der Rechtecksfunktion:
    \begin{alignat*}{4}
        &R(a_1, a_2; b_1, b_2)&& &&+R(a_1, b_2+1; b_1, &&c_2)\\
        &=(a_1, a_2)' &&+\cdots+ (a_1, b_2)'
        &&+(a_1, b_2+1)'&&+\cdots+(a_1, c_2)'\\
        &+(a_1+1, a_2)'&&+\cdots+ (a_1+1, b_2)'
        &&+(a_1+1, b_2+1)'&&+\cdots+(a_1+1, c_2)'\\
        & \quad\vdots&&\quad\vdots&&+\quad\vdots&&\quad\vdots\\
        &+(b_1-1, a_2)'&&+\cdots + (b_1-1, b_2)'
        &&+(b_1-1, b_2+1)'&&+\cdots+(b_1-1, c_2)'\\
        &+(b_1, a_2)'&&+ \cdots + (b_1, b_2)'
        &&+(b_1, b_2+1)'&&+\cdots+(b_1, c_2)'\\
        &=R(a_1, a_2; b_1, &&c_2)
    \end{alignat*}
\end{proof}
Anm.  zu HS \ref{r_summe}: Wenn die für die Gültigkeit dieses HS nötigen Ungleichungen offensichtlich wahr waren, 
habe ich sie oft gar nicht erwähnt. Da sie so offensichtlich sind, wird hierdurch die mathematische Vollständigkeit 
nicht eingeschränkt.
\begin{lem}\label{r_zeile_spalte}
    Wenn $R$ die Rechtecksfunktion einer Tabelle mit $m$ Zeilen und $n$ Spalten ist, ist $R(i, 1; i, n)$ die Summe 
    der Werte in den Feldern der $i$-ten Zeile dieser Tabelle und $R(1, j; m, j)$ die Summe der Werte in den Feldern 
    der $j$-ten Spalte dieser Tabelle.
\end{lem}
\begin{proof}[Beweis des HS]
    Per Defintion ist $R(i, 1; i, n)=(i, 1)'+(i, 2)'+\cdots+(i, n-1)'+(i, n)$. Die Felder der $i$-ten Zeile sind die 
    der Form $(i, a)$, wobei $0<a\leq n$. Offensichtlich sind diese genau die Felder, deren Feldwerte aufsummiert 
    werden, um $R(i, 1; i, n)$ zu berechnen. $R(i, 1; i, n)$ ist also die Summe der Werte in den Feldern der $i$-ten 
    Zeile.

    Ähnlich ist per Definition $R(1, j; m, j)=(1, j)'+(2, j)'+\cdots+(m-1, j)'+(m, j)'$. Die Felder der $j$-ten Spalte 
    sind die der Form $(a, j)$, wobei $0<a\leq m$. Offensichtlich sind diese genau die Felder, deren Feldwerte 
    aufsummiert werden, um $R(1, j; m, j)$ zu berechnen. $R(1, j; m, j)$ ist also die Summe der Werte in den Feldern 
    der $j$-ten Spalte.
\end{proof}
\begin{lem}\label{mehr_nuller}
    Wenn in einer wohlgeordneten Tabelle mit Rechtecksfunktion $R$ für zwei Zahlen $i, j$ die Ungleichungen 
    $R(i, 1; i, n)>R(1, j; m, j)$ gilt, so gilt auch $R(i-a, 1; i-a, n)>R(1, j+b; m, j+b)$ für alle $a, b\geq0$, für 
    die es eine $(i-a)$-te Zeile und eine $(j+b)$-te Spalte gibt. Umgekehrt folgt aus $R(i, 1; i, n)\leq R(1, j; m, j)$, 
    dass für alle $c,d\geq0$, für die es eine $(i+c)$-te Zeile und  eine $(j-d)$-te Spalte gibt, gilt: 
    $R(i+c, 1; i+c, n)\leq R(1, j-d; m, j-d)$. 
\end{lem}
\begin{proof}[Beweis des HS]
    folgt aus wohlgeordnetheit % TODO
\end{proof}

\subsection*{Lösung von Aufgabe 4}

\begin{thm}\label{haupt4}
    In jeder f.p.r.w. Tabelle mit $m$ Zeilen und $n$ Spalten, wobei $m>n$, gibt es ein Feld $(i, j)$, sodass 
    $(i, j)'>0$ und $R(i, 1; i, n) > R(1, j; m, j)$, wobei $'$ die Feld- und $R$ die Rechtecksfunktion dieser 
    Tabelle ist.
\end{thm}

\begin{proof}[Beweis des Satzes durch Widerspruch]
    Nehme man zunächst an, es gäbe eine Tabelle, für die Satz \ref{haupt4} nicht gilt. Nach folgendem HS kann man 
    davon ausgehen, dass diese Tabelle wohlgeordnet ist.
    \begin{lem}\label{fallbeschrank}
        Wenn es keine wohlgeordneten Tabellen gibt, für die Satz \ref{haupt4} nicht gilt, so gibt es überhaupt keine 
        Tabellen, für die Satz \ref{haupt4} nicht gilt.
    \end{lem}
    \begin{proof}[Beweis des HS]
        beweis von diesem lemma (zuerst zeilen umordnen, dann spalten umordnen und gültigkeit von satz 3 ändert sich 
        dabei nicht) %TODO
    \end{proof}
    Da es nach HS \ref{fallbeschrank} überhaupt keine Tabellen gibt, für die Satz \ref{haupt4} 
    nicht gilt, wenn es keine wohlgeordneten gibt, genügt es, zu beweisen, dass es keine wohlgeordneten Tabellen 
    gibt, für die Satz \ref{haupt4} nicht gilt, um zu beweisen, dass es überheupt keine gibt. Es folgt also ein 
    Widerspruchsbeweis dafür, dass es keine wohlgeordneten Tabellen gibt, für die Satz \ref{haupt4} nicht gilt:

    Nehme man für den Widerspruchsbeweis an, es gäbe eine wohlgeordnete, f.p.r.w. Tabelle $T^*$ mit $m$ Zeilen und $n$ 
    Spalten, wobei $m<n$, für die Satz \ref{haupt4} nicht gilt; in der es also kein Feld $(i, j)$ mit $(i, j)'>0$ 
    und $R(i, 1; i, n)>R(1, j; m, j)$ gibt, wobei $'$ die Feldwert- und $R$ die Rechtecksfunktion der Tabelle $T^*$ 
    ist. Dann folgt mit vollständiger Induktion, dass $R(m-t, 1; m, n)\leq R(1, n-t; m, n)$ für alle $t$ mit $0\leq 
    t<m$:

    \textit{Induktionsanfang ($t=0$)}: Es folgt ein Beweis durch Widerspruch. Nehme man hierzu an, es sei nicht 
    $R(m-t, 1; m, n)\leq R(1, n-t; m, n)$ für $t=0$, also $R(m, 1; m, n)>R(1, n; m, n)$. Nach dem ersten Satz von 
    HS \ref{mehr_nuller} (mit $i=m$, $j=n$, $b=0$) muss dann auch $R(m-a, 1; m-a, n) > R(1, n; m, n)$ für alle 
    $a$ mit $0\leq a<m$, denn nur für diese gibt es eine $(m-a)$-te Zeile. Nun sind die Felder $(m-a, n)$ für $0\leq 
    a<m$ genau die Felder der $n$-ten Spalte (ein beliebiges Feld der $n$-ten Spalte $(j, n)$ ergibt sich mit $a=m-j$, 
    was wegen $j>0$ kleiner als $m$ und wegen $j\leq m$ nicht negativ sein kann). Natürlich muss in mindestens einem 
    der Felder der $n$-ten Spalte eine positive Zahl stehen, da $T^*$ f.p.r.w. ist. Es gibt also ein Feld $(j, n)$ 
    mit $(j, n)'>0$ und $R(j, 1; j, n)>R(1, n; m, n)$. Das stellt aber einen Widerspruch dazu dar, dass für $T^*$ nicht 
    Satz \ref{haupt4} gilt! Da aus $R(m, 1; m, n)>R(1, n; m, n)$ ein Widerspruch folgt, kann nicht $R(m, 1; m, n)>
    R(1, n; m, n)$ sein; es muss also $R(m, 1; m, n)\leq R(1, n; m, n)$. Das entspricht $R(m-t, 1; m, n)\leq 
    R(1, n-t; m, n)$ mit $t=0$, die Aussage gilt also für $t=0$.

    \textit{Induktionsschritt}: (Beweis, dass aus der Induktionsannahme, $R(m-t, 1; m, n)\leq R(1, n-t; m, n)$ für 
    ein $t$ mit $0\leq t<m-1$, folgt, dass $R(m-(t+1), 1; m, n)\leq R(1, n-(t+1); m, n)$) Nehme man hierzu an, es sei 
    $R(m-(t+1), 1; m-(t+1), n)>R(1, n-(t+1); m, n-(t+1))$. Dann muss gemäß HS \ref{mehr_nuller} auch 
    $R()$




    Mit dem Induktionsprinzip (Gültigkeit der Aussage für $t=0$ wurde im Induktionsanfang bewiesen, mit dem 
    Induktionsschritt folgt dann Gültigkeit für $t=1$, damit dann auch für $t=2$, \dots, für $t=m$) folgt also, dass 
    $R(m-t, 1; m, n)\leq R(1, n-t; m, n)$ für alle $t$ mit $0\leq t<m$. Insbesondere ist also auch $R(m-t, 1; m, n)
    \leq R(1, n-t; m, n)$ für $t=m-1$, also $R(m-(m-1), 1; m, n)\leq R(1, n-(m-1); m, n)\Leftrightarrow R(1, 1; 
    m, n)\leq R(1, n-m+1; m, n)$. Nun ist nach HS \ref{r_summe} (mit $a_1=1, a_2=1, b_1=n-m, b_2=n$  und 
    $c_2=m$) aber (offensichtlich sind die für die Gültigkeit von HS \ref{r_summe} Ungleichungen mit diesen 
    Werten gegeben):
    \[
    R(1, 1; m, n)=R(1, 1; m, n-m)+R(1, n-m+1; m, n)
    \]
    Also ist, da $R(1, 1; m, n)\leq R(1, n-m+1; m, n)$, auch:
    \begin{alignat*}{2}
        &R(1, 1; m, n-m)+R(1, n-m+1; m, n)&&\leq R(1, n-m+1; m, n)\\
        \Leftrightarrow &R(1, 1; m, n-m)&&\leq 0
    \end{alignat*}
    Nun kann nicht $n<m+1$, denn wegen $m<n$ wäre dann auch $m+1<n+1$ und somit $n<m+1<n+1$, also 
    $n<n+1$, was nicht sein kann. Es ist demnach $n\geq m+1$, also entweder $n=m+1$ oder $n>m+1$. Nehme man zunächst 
    an, es sei $n>m+1$. Dann ist gemäß HS \ref{r_summe} (mit $a_1=1, a_2=1, b_1=m, b_2=1$ und $c_2=n-m$) 
    offenbar $R(1, 1; m, n-m)=R(1, 1; m, 1)+R(1, 2; m, n-m)$. Hierbei ist wegen $n>m+1$ natürlich $n-m>1$, also 
    $n-m\geq 2$; das ist ja notwendige Bedingung für $R(1, 2; n, n-m)$. Da nun $R(1, 1; m, n-m)\leq 0$, ist auch 
    $R(1, 1; m, 1)+R(1, 2; m, n-m)\leq 0$ und damit $R(1, 1; m, 1)\leq -R(1, 2; m, n-m)$. Weil die Tabelle $T^*$ 
    f.p.r.w. ist, stehen in ihren Feldern nur nicht-negative Zahlen. Eine Summe von nicht-negativen Zahlen kann 
    natürlich nicht negativ sein. Da $R(1, 2; m, n-m)$ der Wert einer solchen Summe ist, muss also $R(1, 2; m, n-m)
    \geq 0$ und damit $-R(1, 2; m, n-m)\leq 0$ gelten. Es ist also:
    \[
    R(1, 1; m, 1)\leq -R(1, 2; m, n-m)\leq 0
    \]
    Insbesondere also $R(1, 1; m, 1)\leq 0$. Wegen $R(1, 1; m, n-m)\leq 0$ gilt, wenn $n=m+1$, $R(1, 1; m, (m+1)-m)
    \leq 0$, also $R(1, 1; m, 1)\leq 0$. Es gilt also stets $R(1, 1; m, 1)\leq 0$, egal, ob $n>m+1$ oder $n=m+1$. 
    Nun ist $R(1, 1; m, 1)$ nach HS \ref{r_zeile_spalte} die Summe der Werte in den Feldern der ersten Spalte. Da 
    all diese Werte nicht-negativ sind, muss auch $R(1, 1; m, 1)$ nicht-negativ sein. Weil gleichzeitig aber auch 
    $R(1, 1; m, 1)\leq 0$ ist, kann $R(1, 1; m, 1)$ nicht positiv sein. Es muss also:
    \[
    R(1, 1; m, 1)= 0
    \]
    Dafür müssen dann all die Werte, die in den Feldern der ersten Spalte stehen, 0 sein, da unter ihnen keine 
    negativen sind und gleichzeitig ihre Summe 0 ist. Nun ist aber $T^*$ eine f.p.r.w. Tabelle, also muss es in jeder 
    Spalte von $T^*$ mindestens ein Feld geben, in dem eine positive Zahl steht. Aber ein solches Feld gibt es in 
    der ersten Spalte nicht! Die Annahme, es gäbe eine Tabelle, für die Satz \ref{haupt4} nicht gilt, führt also zu 
    einem Widerspruch. Eine solche Tabelle gibt es also nicht, was den Widerspruchsbeweis von Satz \ref{haupt4} vollendet.
    \renewcommand{\qedsymbol}{$\blacksquare$}
\end{proof}