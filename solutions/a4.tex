\section{Aufgabe 4}

\subsection*{Einführung hilfreicher Begriffe}

Seien zunächst folgende Begriffe eingeführt, die den Beweis hoffentlich kürzer, aber auch besser lesbar machen:

[Im Folgenden ist die $k$-te Spalte immer die $k$-te Spalte von links und die $k$-Zeile immer die $k$-te Zeile von oben.]
\begin{definition}
    Mit einer "`fast-positiv-reell-wertigen"' (kurz f.p.r.w.) Tabelle ist eine Tabelle gemeint, in deren Feldern nur 
    nicht-negative reelle Zahlen stehen und dabei in jeder Spalte der Tabelle mindestens eine positive steht.
\end{definition}
\begin{definition}
    Das Feld $(i, j)$ einer Tabelle ist das Feld in der $i$-ten Zeile und der $j$-ten Spalte dieser Tabelle.
\end{definition}
\begin{definition}
    Wenn $'$ die Feldwertfunktion einer Tabelle ist, ist $(i, j)'$ der Wert, der im Feld $(i, j)$ dieser Tabelle 
    steht.
\end{definition}
\begin{definition}\label{rfunktion}
    Wenn $R$ die Rechtecksfunktion und $'$ die Feldwertfunktion einer Tabelle ist, ist $R(a_1, a_2; b_1, b_2)$ kurz 
    für die folgende Summe, wobei $a_1\leq b_1$ und $a_2\leq b_2$ gelten muss:
    \begin{alignat*}{6}
        R(a_1, a_2; b_1, b_2) &= (a_1, &&a_2)' + &&(a_1, &&a_2+1)' + \quad \cdots \quad + &&(a_1, &&b_2)'\\
        &+(a_1+1, &&a_2)' + &&(a_1+1, &&a_2+1)' +\quad \cdots \quad + &&(a_1+1, &&b_2)'\\
        &+(a_1+2, &&a_2)' + &&(a_1+2, &&a_2+1)' +\quad \cdots \quad + &&(a_1+2, &&b_2)'\\
        & &&\vdotswithin{a} && &&\vdotswithin{a}\quad\quad\quad\quad\quad \ddots && &&\vdotswithin{b} \\
        &+(b_1-1, &&a_2)' + &&(b_1-1, &&a_2+1)' +\quad \cdots \quad + &&(b_1-1, &&b_2)'\\
        &+(b_1, &&a_2)' + &&(b_1, &&a_2+1)' + \quad \cdots \quad + &&(b_1, &&b_2)'
    \end{alignat*}    
\end{definition}
\begin{definition}
    Mit einer "`spaltenweise geordneten"' Tabelle ist eine Tabelle gemeint, für die stets $R(s, 1; s, m)\leq 
    R(s+1, 1; s+1, m)$ gilt, wobei $m$ die Anzahl an Zeilen dieser Tabelle ist und $1\leq s\leq m-1$. Ähnlich ist 
    mit einer "`zeilenweise geordneten"' Tabelle eine gemeint, für die stets $R(z, 1; z, n)\leq R(z+1, 1; z+1, n)$ 
    gilt, wobei $n$ die Anzahl an Spalten dieser Tabelle ist und $1\leq z\leq n-1$. Ist eine Tabelle zeilen- und 
    spaltenweise geordneten, heißt sie kurz auch "`wohlgeordnet"'.
\end{definition}

\subsection*{Hilfssätze über die eingeführten Begriffe}

\renewcommand{\qedsymbol}{$\square$}

\begin{lem}\label{r_summe}
    Wenn $R$ die Rechtecksfunktion einer Tabelle ist und $a_1\leq b_1\leq c_1$ und $a_2\leq b_2\leq c_2$ ist, gilt: 
    $R(a_1, a_2; b_1, b_2)+R(b_1+1, a_2; c_1, b_2)=R(a_1, a_2; c_1, b_2)$ und 
    $R(a_1, a_2; b_1, b_2)+R(a_1, b_2+1; b_1, c_2)=R(a_1, a_2; b_1, c_2)$.
\end{lem}
\begin{proof}[Beweis des HS]
    Erstere Aussage ergibt sich direkt aus der Definition der Rechtecksfunktion:
    \begin{alignat*}{3}
        &R(a_1, a_2; b_1, b_2)&& +R(b_1+1, a_2; c_1, b_2)&&\\
        &=(a_1, a_2)' &&+ (a_1, a_2+1)' &&+\cdots+ (a_1, b_2)'\\
        &+(a_1+1, a_2)' &&+ (a_1+1, a_2+1)' &&+\cdots+ (a_1+1, b_2)'\\
        & &&\vdots\quad\quad\quad\quad&&\vdots\\
        &+(b_1-1, a_2)' &&+ (b_1-1, a_2+1)' &&+\cdots + (b_1-1, b_2)'\\
        &+(b_1, a_2)' &&+ (b_1, a_2+1)' &&+ \cdots + (b_1, b_2)'\\
        &+R(b_1+1, a_2; &&c_1, b_2) &&\\
        &=(a_1, a_2)' &&+ (a_1, a_2+1)' &&+\cdots+ (a_1, b_2)'\\
        &+(a_1+1, a_2)' &&+ (a_1+1, a_2+1)' &&+\cdots+ (a_1+1, b_2)'\\
        & &&\vdots\quad\quad\quad\quad&&\vdots\\
        &+(b_1-1, a_2)' &&+ (b_1-1, a_2+1)' &&+\cdots + (b_1-1, b_2)'\\
        &+(b_1, a_2)' &&+ (b_1, a_2+1)' &&+ \cdots + (b_1, b_2)'\\
        &+(b_1+1, a_2)' &&+ (b_1+1, a_2+1)' &&+\cdots+ (b_1+1, b_2)'\\
        &+(b_1+2, a_2)' &&+ (b_1+2, a_2+1)' &&+\cdots+ (b_1+2, b_2)'\\
        & &&\vdots\quad\quad\quad\quad&&\vdots\\
        &+(c_1-1, a_2)' &&+ (c_1-1, a_2+1)' &&+\cdots + (c_1-1, b_2)'\\
        &+(c_1, a_2)' &&+ (c_1, a_2+1)' &&+ \cdots + (c_1, b_2)'\\
        &=R(a_1, a_2; c_1, &&b_2)&&
    \end{alignat*}
    Auch letztere Aussage ergibt sich direkt durch Einsetzen der Definition der Rechtecksfunktion:
    \begin{alignat*}{4}
        &R(a_1, a_2; b_1, b_2)&& &&+R(a_1, b_2+1; b_1, &&c_2)\\
        &=(a_1, a_2)' &&+\cdots+ (a_1, b_2)'
        &&+(a_1, b_2+1)'&&+\cdots+(a_1, c_2)'\\
        &+(a_1+1, a_2)'&&+\cdots+ (a_1+1, b_2)'
        &&+(a_1+1, b_2+1)'&&+\cdots+(a_1+1, c_2)'\\
        & \quad\vdots&&\quad\vdots&&+\quad\vdots&&\quad\vdots\\
        &+(b_1-1, a_2)'&&+\cdots + (b_1-1, b_2)'
        &&+(b_1-1, b_2+1)'&&+\cdots+(b_1-1, c_2)'\\
        &+(b_1, a_2)'&&+ \cdots + (b_1, b_2)'
        &&+(b_1, b_2+1)'&&+\cdots+(b_1, c_2)'\\
        &=R(a_1, a_2; b_1, &&c_2)
    \end{alignat*}
\end{proof}
Anm.  zu HS \ref{r_summe}: Wenn die für die Gültigkeit dieses HS nötigen Ungleichungen offensichtlich wahr waren, 
habe ich sie oft gar nicht erwähnt. Da sie so offensichtlich sind, wird hierdurch die mathematische Vollständigkeit 
nicht eingeschränkt.
\begin{lem}\label{r_zeile_spalte}
    Wenn $R$ die Rechtecksfunktion einer Tabelle mit $m$ Zeilen und $n$ Spalten ist, ist $R(i, 1; i, n)$ die Summe 
    der Werte in den Feldern der $i$-ten Zeile dieser Tabelle und $R(1, j; m, j)$ die Summe der Werte in den Feldern 
    der $j$-ten Spalte dieser Tabelle.
\end{lem}
\begin{proof}[Beweis des HS]
    Per Defintion ist $R(i, 1; i, n)=(i, 1)'+(i, 2)'+\cdots+(i, n-1)'+(i, n)$. Die Felder der $i$-ten Zeile sind die 
    der Form $(i, a)$, wobei $0<a\leq n$. Offensichtlich sind diese genau die Felder, deren Feldwerte aufaddiert 
    werden, um $R(i, 1; i, n)$ zu berechnen. $R(i, 1; i, n)$ ist also die Summe der Werte in den Feldern der $i$-ten 
    Zeile.

    Ähnlich ist per Definition $R(1, j; m, j)=(1, j)'+(2, j)'+\cdots+(m-1, j)'+(m, j)'$. Die Felder der $j$-ten Spalte 
    sind die der Form $(a, j)$, wobei $0<a\leq m$. Offensichtlich sind diese genau die Felder, deren Feldwerte 
    aufaddiert werden, um $R(1, j; m, j)$ zu berechnen. $R(1, j; m, j)$ ist also die Summe der Werte in den Feldern 
    der $j$-ten Spalte.
\end{proof}
\begin{lem}\label{mehr_nuller}
    Wenn in einer wohlgeordneten Tabelle mit Rechtecksfunktion $R$ für zwei Zahlen $i, j$ die Ungleichung 
    $R(i, 1; i, n)>R(1, j; m, j)$ gilt, so gilt auch $R(i+a, 1; i+a, n)>R(1, j-b; m, j-b)$ für alle $a, b\geq0$, für 
    die es eine $(i+a)$-te Zeile und eine $(j-b)$-te Spalte gibt.
\end{lem}
\begin{proof}[Beweis des HS]
    Zunächst gilt, da die Tabelle wohlgeordnet ist:
    \begin{align*}
        R(i, 1; i, n)&\leq R(i+1, 1; i+1, n)\\
        R(i+1, 1; i+1, n)&\leq R(i+2, 1; i+2, n)\\
        &\quad\vdots\\
        R(i+(a-1), 1; i+(a-1), n)&\leq R(i+a, 1; i+a, n)
    \end{align*}
    Diese Ungleichungen lassen sich folgendermaßen zusammenfassen:
    \[
        R(i, 1; i, n)\leq R(i+1, 1; i+1, n)\leq R(i+2, 1; i+2, n)\leq\cdots\leq R(i+a, 1; i+a, n)
    \]
    Insbesondere ist also $R(i+a, 1; i+a, n)\geq R(i, 1; i, n)$. Erneut aufgrund der Wohlgeordnetheit der Tabelle 
    gilt auch:
    \begin{align*}
        R(1, j; m, j)&\geq R(1, j-1; m, j-1)\\
        R(1, j-1; m, j-1)&\geq R(1, j-2; m, j-2)\\
        &\quad\vdots\\
        R(1, j-(b-1); m, j-(b-1))&\geq R(1, j-b; m, j-b)
    \end{align*}
    Auch diese Gleichungen lassen sich in einer Zeile zusammenfassen:
    \[
        R(1, j; m, j)\geq R(1, j-1; m, j-1)\geq R(1, j-2; m, j-2)\geq\cdots\geq R(1, j-b; m, j-b)
    \]
    Insbesondere ist also $R(1, j; m, j)\geq R(1, j-b; m, j-b)$. Es gelten also die folgenden drei Ungleichungen:
    \begin{align*}
        R(i+a, 1; i+a, n)&\geq R(i, 1; i, n)\\
        R(i, 1; i, n)&>R(1, j; m, j)\\
        R(1, j; m, j)&\geq R(1, j-b; m, j-b)
    \end{align*}
    In einer Zeile zusammengefasst ergibt das:
    \[
        R(i+a, 1; i+a, n)\geq R(i, 1; i, n)>R(1, j; m, j)\geq R(1, j-b; m, j-b)
    \]
    Insbesondere ist also $R(i+a, 1; i+a, n)>R(1, j-b; m, j-b)$.
\end{proof}

\subsection*{Lösung von Aufgabe 4}

In Aufgabe 4 werden Tabellen betrachtet, bei denen in jedem Feld eine nicht-negative reelle Zahl steht und dabei in 
jeder Spalte mindestens eine positive. Es handelt sich also um f.p.r.w. Tabellen. Auch müssen die Tabellen mehr 
Spalten als Zeilen haben. Es werden also f.p.r.w. Tabellen mit $m$ Zeilen und $n$ Spalten betrachtet, wobei $m<n$. 
Es soll nun bewiesen werden, dass es ein Feld mit einer positiven Zahl derart gibt, dass die Summe der Zahlen in der 
Zeile dieses Feldes größer als die Summe der Zahlen in der Spalte dieses Feldes ist. Nenne man dieses Feld $(i, j)$, 
die Feldwertfunktion der betrachteten Tabelle $'$ und ihre Rechtecksfunktion $R$. Dann muss also der Feldwert von $(i, j)$ 
positiv sein; es muss also $(i, j)'>0$ gelten. Auch muss die Summe der Zahlen in der Zeile dieses Feldes (welche 
gemäß HS \ref{r_zeile_spalte} durch $R(i, 1; i, n)$ gegeben ist) größer als die Summe der Zahlen in der Spalte 
dieses Feldes (welche gemäß HS \ref{r_zeile_spalte} durch $R(1, j; m, j)$ gegeben ist) sein; es muss also 
$R(i, 1; i, n) > R(1, j; m, j)$. Insgesamt ist die in Aufgabe 4 zu zeigende Aussage also äquivalent zu folgendem 
Satz:

\begin{thm}\label{haupt4}
    In jeder f.p.r.w. Tabelle mit $m$ Zeilen und $n$ Spalten, wobei $m<n$, gibt es ein Feld $(i, j)$, sodass 
    $(i, j)'>0$ und $R(i, 1; i, n) > R(1, j; m, j)$, wobei $'$ die Feld- und $R$ die Rechtecksfunktion dieser 
    Tabelle ist.
\end{thm}

\begin{proof}[Beweis des Satzes durch Widerspruch]
    Nehme man zunächst an, es gäbe eine Tabelle, für die Satz \ref{haupt4} nicht gilt. Nach folgendem HS kann man 
    davon ausgehen, dass diese Tabelle wohlgeordnet ist.
    \begin{lem}\label{fallbeschrank}
        Wenn es keine wohlgeordneten Tabellen gibt, für die Satz \ref{haupt4} nicht gilt, so gibt es überhaupt keine 
        Tabellen, für die Satz \ref{haupt4} nicht gilt.
    \end{lem}
    \begin{proof}[Beweis des HS durchh Widerspruch]
        Nehme man für den Widerspruchsbeweis an, es gäbe eine f.p.r.w. Tabelle $T$ mit mehr Spalten als Zeilen, für 
        die Satz \ref{haupt4} nicht gilt, aber keine solchen Tabellen, die wohlgeordnet sind. 

        Sei zunächst $\dot T$ die gleiche Tabelle wie $T$, allerdings mit den Zeilen so umgeordnet, dass sie 
        zeilenweise geordnet ist. Da sich nur die Reihenfolge der Zeilen ändert, gibt es in jeder Spalte immer noch 
        mindestens ein Feld mit einer positiven Zahl. Natürlich gibt es auch immer noch nur Felder mit nicht-negativen 
        Zahlen, also ist $\dot T$ insgesamt f.p.r.w. Außerdem hat $\dot T$ genau wie $T$ mehr Spalten als Zeilen. 
        Wenn für $\dot T$ Satz \ref{haupt4} gelten würde, müsste es ein Feld $(i, j)$ geben, sodass $(i, j)'>0$ 
        und $\dot R(i, 1; i, \dot n) > \dot R(1, j; \dot m, j)$, wobei $'$ die Feld-, $\dot R$ die Rechtecksfunktion, 
        $\dot m$ die Anzahl an Zeilen und $\dot n$ die Anzahl an Spalten von $\dot T$ ist. Sei nun die $l$-te Zeile 
        von $T$ die Zeile, die nach der Umordnung die $i$-te Zeile von $\dot T$ bildet. Der Wert des Feldes $(l, j)$ 
        von $T$ steht also nach der Umordnung im Feld $(i, j)$ von $\dot T$. Da $(i, j)'>0$, ist also:
        \[
            (l, j)^*>0,
        \]
        wobei $^*$ die Feldwertfunktion von $T$ ist. In der $j$-ten Spalte von $\dot T$ befinden sich offensichtlich 
        die gleichen Zahlen wie in der $j$-ten Spalte von $T$; lediglich ihre Reihenfolge ändert sich durch die 
        Umordnung der Zeilen. Ihre Summe (gemäß HS \ref{r_zeile_spalte} durch $R(1, j; m, j)$ bzw. $\dot R(1, j; 
        \dot m, j)$ gegeben) ist also auch die gleiche:
        \[
            R(1, j; m, j)=\dot R(1, j; \dot m, j),
        \]
        wobei $R$ die Rechtecksfunktion $m$ die Anzahl an Zeilen und $n$ die Anzahl an Spalten von $T$ ist. 
        Sämtliche Zahlen der $i$-ten Zeile von $\dot T$ standen vor der Umordnung in der $l$-ten Zeile von $T$. 
        Die Summe der Zahlen in diesen beiden Zeilen muss also die gleiche sein. Es ist also (erneut ergeben sich die 
        folgenden Werte direkt aus HS \ref{r_zeile_spalte}):
        \[
            R(i, 1; i, n)=\dot R(i, 1; i, \dot n)
        \]
        Insgesamt ist also $(l, j)^*>0$, $R(1, j; m, j)=\dot R(1, j; \dot m, j)$ und $R(i, 1; i, n)=\dot R(i, 1; i, 
        \dot n)$. Da $\dot R(i, 1; i, \dot n) > \dot R(1, j; \dot m, j)$, muss also auch $R(i, 1; i, n) > 
        R(1, j; m, j)$. Damit gibt es dann aber das Feld $(i, j)$ der Tabelle $T$, für das $(l, j)^*>0$ und $R(i, 
        1; i, n) > R(1, j; m, j)$. Für die Tabelle $T$ gilt also Satz \ref{haupt4}. Da das nicht sein kann, 
        muss die Annahme, für $\dot T$ würde Satz \ref{haupt4} gelten, falsch sein! Für $\dot T$ gilt Satz 
        \ref{haupt4} also nicht!

        Sei nun $\ddot T$ die gleiche Tabelle wie $\dot T$, allerdings mit den Spalten so umgeordnet, dass $\ddot T$ 
        spaltenweise geordnet ist. Da sich nur die Reihenfolge der Spalten ändert, gibt es in jeder Spalte immer noch 
        mindestens ein Feld mit einer positiven Zahl. Natürlich gibt es auch immer noch nur Felder mit nicht-negativen 
        Zahlen, also ist $\ddot T$ insgesamt f.p.r.w. Außerdem hat $\ddot T$ genau wie $\dot T$ mehr Spalten als Zeilen. 
        Wenn für $\ddot T$ Satz \ref{haupt4} gelten würde, müsste es ein Feld $(i, j)$ geben, sodass $(i, j)^{**}>0$ 
        und $\ddot R(i, 1; i, \ddot n) > \ddot R(1, j; \ddot m, j)$, wobei $^{**}$ die Feld-, $\ddot R$ die 
        Rechtecksfunktion, $\ddot m$ die Anzahl an Zeilen und $\ddot n$ die Anzahl an Spalten von $\ddot T$ ist. 
        Sei nun die $k$-te Spalte von $\dot T$ die Spalte, die nach der Umordnung die $j$-te Spalte von $\ddot T$ 
        bildet. Der Wert des Feldes $(i, k)$ von $\dot T$ steht also nach der Umordnung im Feld $(i, j)$ von 
        $\ddot T$. Da $(i, j)^{**}>0$, ist also:
        \[
            (i, k)^*>0
        \]
        In der $i$-ten Zeile von $\ddot T$ befinden sich offensichtlich die gleichen Zahlen wie in der $i$-ten Zeile 
        von $\dot T$; lediglich ihre Reihenfolge ändert sich durch die Umordnung der Spalten. Ihre Summe (gemäß HS 
        \ref{r_zeile_spalte} durch $\dot R(i, 1; i, \dot n)$ bzw. $\ddot R(i, 1; i, \ddot n)$ gegeben) ist also auch 
        die gleiche:
        \[
            \dot R(i, 1; i, \dot n)=\ddot R(i, 1; i, \ddot n),
        \]
        wobei $\ddot R$ die Rechtecksfunktion von $\ddot T$ ist.
        Sämtliche Zahlen der $j$-ten Spalte von $\ddot T$ standen vor der Umordnung in der $k$-ten Spalte von $\dot T$. 
        Die Summe der Zahlen in diesen beiden Zeilen muss also die gleiche sein. Es ist also (erneut ergeben sich die 
        folgenden Werte direkt aus HS \ref{r_zeile_spalte}):
        \[
            \dot R(1, k; \dot m, k)=\ddot R(1, j; \ddot m, j)
        \]
        Insgesamt ist also $(i, k)^*>0$, $\dot R(i, 1; i, \dot n)=\ddot R(i, 1; i, \ddot n)$ und $\dot R(1, k; 
        \dot m, k)=\ddot R(1, j; \ddot m, j)$. Da $\ddot R(i, 1; i, \ddot n)>\ddot R(1, j; \ddot m, j)$, muss auch 
        $\dot R(i, 1; i, \dot n)>\dot R(1, k; \dot m, k)$. Damit gibt es aber ein Feld $(i, k)$ mit $(i, k)^*>0$ und 
        $\dot R(i, 1; i, \dot n)>\dot R(1, k; \dot m, k)$. Für $\dot T$ gilt also Satz \ref{haupt4}. Das kann aber--wie 
        oben gezeigt--nicht sein. Die Annahme, für $\ddot T$ würde Satz \eqref{haupt4} gelten, führt also zu einem 
        Widerspruch, d.h. sie muss falsch sein! Für $\ddot T$ kann Satz \ref{haupt4} also nicht gelten!

        Während der Umordnung der Spalten von $\dot T$, um $\ddot T$ zu erhalten, wurden offensichtlich die Summen der 
        Zahlen in den Spalten der Tabelle nicht geändert; jedes Feld blieb ja in der gleichen Zeile, da nur die 
        Spalten entlang der Zeilen umgeordnet wurden. Also ist $\ddot T$, da $\dot T$ zeilenweise geordnet ist, 
        ebenfalls zeilenweise geordnet. Nun ist $\ddot T$ aber auch spaltenweise geordnet, also ist $\ddot T$ 
        wohlgeordnet!

        $\ddot T$ ist also eine wohlgeordnete Tabelle, für die Satz \ref{haupt4} nicht gilt. Nun wurde am Anfang dieses 
        Widerspruchsbeweises angenommen, es gäbe zwar Tabellen, für die Satz \ref{haupt4} nicht gilt, allerdings keine 
        solchen wohlgeordneten. Da diese Annahme zu einem Widerspruch--nämlich einer wohlgeordneten Tabelle, für die 
        Satz \ref{haupt4} nicht gilt--geführt hat, muss sie falsch sein! Wenn es also keine wohlgeordneten Tabellen 
        gibt, für die Satz \eqref{haupt4} nicht gilt, kann es überhaupt keine Tabellen geben, für die Satz \ref{haupt4} 
        nicht gilt!
    \end{proof}
    Da es nach HS \ref{fallbeschrank} überhaupt keine Tabellen gibt, für die Satz \ref{haupt4} 
    nicht gilt, wenn es keine wohlgeordneten gibt, genügt es, zu beweisen, dass es keine wohlgeordneten Tabellen 
    gibt, für die Satz \ref{haupt4} nicht gilt, um zu beweisen, dass es überhaupt keine gibt. Es folgt also ein 
    Widerspruchsbeweis dafür, dass es keine wohlgeordneten Tabellen gibt, für die Satz \ref{haupt4} nicht gilt:

    Nehme man für den Widerspruchsbeweis an, es gäbe eine wohlgeordnete, f.p.r.w. Tabelle $T^*$ mit $m$ Zeilen und $n$ 
    Spalten, wobei $m<n$, für die Satz \ref{haupt4} nicht gilt; in der es also kein Feld $(i, j)$ mit $(i, j)'>0$ 
    und $R(i, 1; i, n)>R(1, j; m, j)$ gibt, wobei $'$ die Feldwert- und $R$ die Rechtecksfunktion der Tabelle $T^*$ 
    ist. Dann folgt mit vollständiger Induktion, dass $R(1, 1; t, n)\leq R(1, 1; m, t)$ für alle $t$ mit $1\leq 
    t\leq m$:

    \textit{Induktionsanfang ($t=1$)}: Es folgt ein Beweis durch Widerspruch. Nehme man hierzu an, es sei nicht 
    $R(1, 1; t, n)\leq R(1, 1; m, t)$ für $t=1$, also:
    \[
        R(1, 1; 1, n)>R(1, 1; m, 1)
    \]
    Nach HS \ref{mehr_nuller} (mit $i=1$, $j=1$, $b=0$) muss dann auch $R(1+a, 1; 1+a, n) > R(1, 1; m, 1)$ für alle 
    $a$ mit $0\leq a<m$, denn nur für diese gibt es eine $(1+a)$-te Zeile. Sei nun $\dot a=a+1$, also $1\leq \dot 
    a\leq m$. Dann sind die Felder $(\dot a, 1)$ offensichtlich genau die Felder der ersten Spalte. Natürlich muss 
    in mindestens einem der Felder der ersten Spalte eine positive Zahl stehen, da $T^*$ f.p.r.w. ist. Es gibt also 
    ein Feld $(\dot a, 1)$ mit $(\dot a, 1)'>0$ und $R(\dot a, 1; \dot a, n)>R(1, 1; m, 1)$. Wenn für $T^*$ nicht 
    Satz \ref{haupt4} gelten würde, dürfte es ein solches Feld aber nicht geben. Dessen Exisenz stellt also einen 
    Widerspruch dazu dar, dass für $T^*$ nicht Satz \ref{haupt4} gilt! Da aus $R(1, 1; 1, n)>R(1, 1; m, 1)$ ein 
    Widerspruch folgt, kann nicht $R(1, 1; 1, n)>R(1, 1; m, 1)$ sein; es muss also:
    \[
        R(1, 1; 1, n)\leq R(1, 1; m, 1)
    \]
    Das entspricht $R(1, 1; t, n)\leq R(1, 1; m, t)$ mit $t=1$; die Aussage gilt also für $t=1$.

    \textit{Induktionsschritt}: (Beweis, dass aus der Induktionsannahme, $R(1, 1; t, n)\leq R(1, 1; m, t)$ für 
    ein $t$ mit $1\leq t\leq m-1$, folgt, dass $R(1, 1; t+1, n)\leq R(1, 1; m, t+1)$) Nehme man hierzu an, es sei:
    \[
        R(t+1, 1; t+1, n)>R(1, t+1; m, t+1)
    \]
    Dann muss gemäß HS \ref{mehr_nuller} auch:
    \[
        R(t+1+a, 1; t+1+a, n)>R(1, t+1-b; m, t+1-b), \numberthis \label{or_ecke}
    \]
    wobei $0\leq a\leq m-(t+1)$ und $0\leq b\leq t$. Offensichtlich ist mit $0\leq a\leq m-(t+1)$ dann $t+1\leq t+1+a
    \leq m$, woraus zusammen mit $0<t+1$ folgt, dass es eine $(t+1+a)$-te Zeile gibt. Ähnlich ist mit $0\leq b$ 
    zunächst $t+1-b\leq t+1$ und mit $b\leq t$, also $-t\leq-b$, auch $t+1-t\leq t+1-b\Leftrightarrow 1\leq t+1-b$, 
    also insgesamt $1\leq t+1-b\leq t+1$, woraus zusammen mit $t\leq m-1$, also $t+1\leq m$, folgt, dass es eine 
    $(t+1-b)$-te Spalte gibt. Alle nötigen Bedingungen für HS \ref{mehr_nuller} sind also gegeben, weshalb 
    \eqref{or_ecke} gilt. Die Ungleichung \eqref{or_ecke} lässt sich wegen $t+1\leq t+1+a\leq m$ und $1\leq t+1-b
    \leq t+1$ kürzer schreiben, indem man $c=t+1+a$ und $d=t+1-b$ setzt:
    \[
        R(c, 1; c, n)>R(1, d; m, d), \numberthis \label{or_eck}
    \]
    wobei $t+1\leq c\leq m$ und $1\leq d\leq t+1$.

    Nehme man nun an, es gäbe ein Feld $(i, j)$ mit $t+1\leq i\leq m$, $1\leq j\leq t+1$ und $(i, j)'>0$. Dann 
    ist wegen \eqref{or_eck} $R(i, 1; i, n)>R(1, j; m, j)$. Wenn es ein solches Feld $(i, j)$ gäbe, würde also für 
    $T^*$ Satz \ref{haupt4} gelten. Das kann aber unter der Annahme des Widerspruchsbeweises nicht sein; es kann 
    also kein Feld $(i, j)$ mit $t+1\leq i\leq m$, $1\leq j\leq t+1$ und $(i, j)'>0$ geben. D.h. es muss $(i, j)'
    =0$ für alle $(i, j)$ mit $t+1\leq i\leq m$ und $1\leq j\leq t+1$ gelten. Diese lassen sich in zwei Gruppen 
    unterteilen:
    \begin{enumerate}
        \item Die mit $t+1\leq i\leq m$ und $1\leq j<t+1$, also $1\leq j\leq t$
        \item Die mit $t+1\leq i\leq m$ und $j=t+1$
    \end{enumerate}
    Erstere sind die Felder, deren Feldwerte aufaddiert werden, um $R(t+1, 1; m, t)$ zu berechnen. 
    % TODO: kurze begründung
    Letztere sind genau die Felder, deren Feldwerte aufaddiert werden, um $R(t+1, t+1; m, t+1)$ zu berechnen. 
    Beide sind also eine Summe von Nullern, also selbst 0:
    \[
        R(t+1, 1; m, t)=0 \numberthis \label{das_ne_null}
    \]
    \[
        R(t+1, t+1; m, t+1)=0 \numberthis \label{nuller_spalte}
    \]
    Nun ist nach HS \ref{r_summe} (mit $a_1=1, a_2=1, b_1=t, b_2=t$ und $c_1=m$):
    \[
        R(1, 1; m, t)=R(1, 1; t, t)+R(t+1, 1; m, t)
    \]
    Mit \eqref{das_ne_null} ergibt sich daraus direkt:
    \[
        R(1, 1; m, t)=R(1, 1; t, t)\numberthis \label{rechts_is_unrecht}
    \]
    Auch ergibt sich mit HS \ref{r_summe} (mit $a_1=1, a_2=1, b_1=t, b_2=t$ und $c_2=n$):
    \[
        R(1, 1; t, n)=R(1, 1; t, t)+R(1, t+1; t, n)
    \]
    Wenn man in diese Gleichung \eqref{rechts_is_unrecht} einsetzt, erhält man:
    \[
        R(1, 1; t, n)=R(1, 1; m, t)+R(1, t+1; t, n) \numberthis \label{2.summ_is_0}
    \]
    Nun ist $R(1, t+1; t, n)$ die Summe nicht-negativer Zahlen, also selbst nicht-negativ. Allerdings kann $R(1, t+1; 
    t, n)$ auch nicht positiv sein, denn dann wäre wegen \eqref{2.summ_is_0}:
    \[
        R(1, 1; t, n)=R(1, 1; m, t)+R(1, t+1; t, n)>R(1, 1; m, t)
    \]
    Also wäre insbesondere:
    \[
        R(1, 1; t, n)>R(1, 1; m, t)
    \]
    Das widerpsicht aber der Induktionsannahme, nach welcher $R(1, 1; t, n)\leq R(1, 1; m, t)$. Also ist $R(1, t+1; 
    t, n)$ nicht-negativ und nicht positiv, also Null. Es ist also:
    \[
        R(1, t+1; t, n)=0 \numberthis \label{rest_auch_0}
    \]
    Nun ist nach HS \ref{r_summe} (mit $a_1=1, a_2=t+1, b_1=t, b_2=t+1$ und $c_2=n$):
    \[
        R(1, t+1; t, n)=R(1, t+1; t, t+1)+R(1, t+2; t, n)
    \]
    Da die linke Seite nach \eqref{rest_auch_0} Null ist, muss es natürlich auch die rechte sein. Nun sind die beiden 
    Summanden auf der rechten Seite nicht-negativ, da sie Summen von nicht-negativen Zahlen sind. Damit ihre Summe 
    also Null sein kann, müssen sie beide Null sein. Es ist also insbesondere:
    \[
        R(1, t+1; t, t+1)=0 \numberthis \label{rest_spalte_0}
    \]
    Gemäß HS \ref{r_summe} (mit $a_1=1, a_2=t+1, b_1=t, b_2=t+1$ und $c_1=m$) ist nun:
    \[
        R(1, t+1; m, t+1)=R(1, t+1; t, t+1)+R(t+1, t+1; m, t+1)
    \]
    Wegen \eqref{rest_spalte_0} ist der erste Summand gleich 0. Wegen \eqref{nuller_spalte} ist aber auch der zweite 
    Summand gleich Null! Es muss also auch ihre Summe gleich Null sein:
    \[
        R(1, t+1; m, t+1)=0
    \]
    Nun ist $R(1, t+1; m, t+1)$ aber gemäß HS \ref{r_zeile_spalte} die Summe der Zahlen in der $(t+1)$-ten Spalte. 
    Da all diese Zahlen nicht-negativ sind, müssen sie, damit ihre Summe gleich Null sein kann, alle gleich Null sein. 
    Dies stellt aber einen Widerspruch dazu dar, dass $T^*$ eine f.p.r.w. Tabelle ist, denn in einer f.p.r.w. Tabelle 
    gibt es in jeder Spalte mindestens ein Feld mit einer positiven Zahl! Die Annahme, es sei $R(t+1, 1; t+1, n)>
    R(1, t+1; m, t+1)$ führt also zu einem Widerspruch und muss dementsprechend falsch sein. Es ist also:
    \[
        R(t+1, 1; t+1, n)\leq R(1, t+1; m, t+1)
    \]
    Außerdem gilt natürlich--das ist die Induktionsannahme--folgende Ungleichung:
    \[
        R(1, 1; t, n)\leq R(1, 1; m, t)
    \]
    Zusammen ergeben diese beiden Ungleichungen direkt:
    \[
        R(1, 1; t, n)+R(t+1, 1; t+1, n)\leq R(1, 1; m, t)+R(1, t+1; m, t+1) \numberthis \label{fast_schritt}
    \]
    Nun gelten nach HS \ref{r_summe} (mit bzw. ) folgende Gleichungen:
    \begin{align*}
        R(1, 1; t, n)+R(t+1, 1; t+1, n)&=R(1, 1; t+1, n)\\
        R(1, 1; m, t)+R(1, t+1; m, t+1)&=R(1, 1; m, t+1)
    \end{align*}
    Setzt man diese in \eqref{fast_schritt} ein, erhält man:
    \[
        R(1, 1; t+1, n)\leq R(1, 1; m, t+1)
    \]
    Insgesamt folgt also aus $R(1, 1; t, n)\leq R(1, 1; m, t)$ (für ein $t$ mit $0\leq t\leq m-1$) die gleiche 
    Ungleichung, allerdings mit $t+1$ statt $t$, also $R(1, 1; t+1, n)\leq R(1, 1; m, t+1)$.

    Die Ungleichung $R(1, 1; t, n)\leq R(1, 1; m, t)$ gilt also gemäß dem Induktionsanfang für $t=1$ und damit gilt sie 
    dann gemäß dem Induktionsschritt auch für $t=2$ (natürlich nur sofern $m\geq 2$) usw. usw. bis $t=m$. Für $t=m$ 
    lautet die Ungleichung dann:
    \[
        R(1, 1; m, n)\leq R(1, 1; m, m) \numberthis \label{widerspruch}
    \]
    Nun ist nach HS \ref{r_summe} (mit $a_1=1, a_2=1, b_1=m, b_2=m$ und $c_2=n$) aber:
    \[
        R(1, 1; m, n)=R(1, 1; m, m)+R(1, m+1; m, n)
    \]
    Setzt man dies in \eqref{widerspruch} ein, erhält man:
    \begin{align*}
        R(1, 1; m, m)+R(1, m+1; m, n)&\leq R(1, 1; m, m)\\
        \Leftrightarrow R(1, m+1; m, n)&\leq 0 \numberthis \label{ulti}
    \end{align*}
    Wegen $m<n$ ist offensichtlich $n\geq m+1$, also entweder $n=m+1$ oder $n>m+1$. Nehme man zunächst 
    an, es sei $n>m+1$, also $n=m+1+k$ mit $1\leq k$. Es ist also:
    \[
        R(1, m+1; m, n)=R(1, m+1; m, m+1+k)
    \]
    Dann ist unter Verwendung dieser Gleichung gemäß HS \ref{r_summe} (mit $a_1=1, a_2=1, b_1=m, b_2=m$ und $c_2=m+1$) 
    offenbar:
    \[
        R(1, m+1; m, n)=R(1, m+1; m, m+1)+R(1, m+2; m, m+1+k)
    \]
    Da gemäß \eqref{ulti} die linke Seite nicht-positiv ist, muss es auch die rechte Seite sein. Da die beiden 
    Summanden auf der rechten Seite Summen nicht-negativer Zahlen sind, also selbst nicht-negativ sind, müssen sie, 
    damit ihre Summe nicht positiv wird, beide gleich Null sein. Insbesondere ist also:
    \[
        R(1, m+1; m, m+1)=0
    \]
    Wenn $n=m+1$ ist, erhält man, indem man das in \eqref{ulti} einsetzt:
    \[
        R(1, m+1; m, n)=R(1, m+1; m, m+1)\leq 0
    \]
    Da $R(1, m+1; m, m+1)$ die Summe nicht-negativer Zahlen ist, ist $R(1, m+1; m, m+1)$ selbst nicht-negativ. Damit 
    $R(1, m+1; m, m+1)$ gleichzeitig auch nicht-positiv sein kann, muss also:
    \[
        R(1, m+1; m, m+1)=0
    \]
    Egal, ob $n=m+1$ oder $n>m+1$, es gilt also stets diese Gleichung.

    AB HIER NOCH NICHT ANGEPASST
    % TODO: ab hier anpassen
    Dafür müssen dann all die Werte, die in den Feldern der ersten Spalte stehen, 0 sein, da unter ihnen keine 
    negativen sind und gleichzeitig ihre Summe 0 ist. Nun ist aber $T^*$ eine f.p.r.w. Tabelle, also muss es in jeder 
    Spalte von $T^*$ mindestens ein Feld geben, in dem eine positive Zahl steht. Aber ein solches Feld gibt es in 
    der ersten Spalte nicht! Die Annahme, es gäbe eine Tabelle, für die Satz \ref{haupt4} nicht gilt, führt also zu 
    einem Widerspruch. Eine solche Tabelle gibt es also nicht, was den Widerspruchsbeweis von Satz \ref{haupt4} vollendet.
    \renewcommand{\qedsymbol}{$\blacksquare$}
\end{proof}