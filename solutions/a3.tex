\section{Aufgabe 3}

\subsection*{Anmerkungen zur Notation und einer häufig verwendeten Formel}

Im Folgenden wird $x_A$ die $x$- und $y_A$ die $y$-Koordinate des allgemeinen Punktes $A$ bezeichnen. D.h. hier, dass 
$P=(x_P, y_P), N=(x_N, y_N), M=(x_M, y_M), S=(x_S, y_S), Q=(x_Q, y_Q)$.\\
$\overline{AB}$ ist die Distanz von $A$ zu $B$ (stets positiv).

Sehr oft habe ich die Distanzformel 
\[\overline{AB}=\sqrt{(x_A-x_B)^2+(y_A-y_B)^2}=\sqrt{(x_B-x_A)^2+(y_B-y_A)^2}\text{    (für alle $A,B$)}\] 
verwendet. Ich habe bei den Gleichungen, bei denen ich sie verwendet habe, nicht dazugeschrieben, dass sich die 
Gleichungen mit der Distanzformel ergeben. Diese Entscheidung habe ich getroffen, da ich diese Formel so oft verwendet 
habe, dass, wenn ich sie jedes mal erwähnt hätte, dies einige Sätze deutlich komplizierter und teilweise sogar 
relativ unverständlich gemacht hätte. Ich denke, dass dies nicht problematisch ist; ich habe ja auch nicht dazugeschrieben, 
wenn ich z.B. einen Bruch gekürzt habe.

\subsection*{Algebraisierung der Aufgabe mit Hilfe eines kartesischen Koordinatensystems}

Um die Aufgabe von einer Geometrischen in eine eher Algebraische zu übersetzen, habe ich mich dazu entschieden, die 
in der Aufgabenstellung beschriebene Konstruktion mit Hilfe eines kartesischen Koordinatensystems zu beschreiben. 
Hierzu definiere man folgendermaßen ein kartesisches Koordinatensystem: $P$ sei der Urspung und $n$ die $x$-Achse; 
dabei sei die $x$-Achse so beschriftet, dass der Punkt $N$ eine positive $x$-Koordinate hat, bevor er $P$ passiert, 
und die $y$-Achse so, dass der Punkt $M$ eine positive $y$-Koordinate hat, bevor er $P$ passiert. Dies ist 
offensichtlich möglich, da $P$ auf $n$ liegt. Offensichtlich ist auch die Beschriftung der $y$-Achse (also der 
Orthogonalen zu $n$, die durch $P$ geht) möglich, denn die Geraden $m$ und $n$ liegen nicht aufeinander (sonst 
würden sie sich in mehr als einem Punkt schneiden); der Punkt $M$ muss also, bevor er $P$ passiert, auf einer Seite 
der $x$-Achse liegen.

Der Punkt $N$ hat, da er auf $n$, also der $x$-Achse, liegt, natürlich eine $y$-Koordinate von 0. Es ist also: 
$N=(x_N, 0)$. Sei nun $d_M$ die Distanz $\overline{PM}$, allerdings mit negativem Vorzeichen, wenn $y_M$ ein 
negatives Vorzeichen hat, also nachdem $M$ den Punkt $P$ passiert hat. Nun ist $x_N$ wegen $y_N=0$ die Distanz 
$\overline{PN}$, allerdings mit negativem Vorzeichen, nachdem $N$ den Punkt $P$ passiert hat. Offensichtlich ist 
dann, da $N$ und $M$ sich mit der gleichen, konstanten Geschwindigkeit von $P$ weg/zu $P$ hin bewegen, $d_M-x_N$ 
konstant. Sei diese Konstante mit $c$ bezeichnet. Dann ist $d_M-x_N=c\Leftrightarrow d_M=x_N+c$ und damit 
$\overline{PM}=|x_N+c|$. Nun ist $|x_N+c|=\overline{PM}=\sqrt{(x_M-x_P)^2+(y_M-y_P)^2}=\sqrt{x_M^2+y_M^2}$, da 
$y_P=x_P=0$. Die Gerade $m$ ist, da sie durch $P$, den Urspung des kartesischen Koordinatensystems, geht, natürlich 
eine Urspungsgerade. Im Folgenden werde ich nur die Fälle betrachten, in denen $m$ nicht auf der $y$-Achse liegt, 
also nicht orthogonal zu $n$ ist. Dann hat $m$ eine reele Steigung, die nicht Null ist, da $m$ sonst auf $n$ liegen 
würde, was einen Widerspruch dazu darstellt, dass $m$ und $n$ sich nur in einem Punkt $P$ schneiden. Diese Steigung 
wird im Folgenden $r$ genannt. Die Geradengleichung von $m$ ist also: $y=r\cdot x$ bzw. $x=\frac{y}{r}$. Da $M$ auf 
$m$ liegt, muss $x_M=\frac{y_M}{r}$ gelten. Zusammen mit $|x_N+c|=\sqrt{x_M^2+y_M^2}$ ergibt das dann: $|x_N+c|
=\sqrt{x_M^2+y_M^2}=\sqrt{(\frac{y_M}{r})^2+y_M^2}=\sqrt{\frac{1}{r^2}\cdot y_M^2+y_M^2}=\sqrt{(1+\frac{1}{r^2})
\cdot y_m^2}=|y_M|\cdot\sqrt{1+\frac{1}{r^2}}$. Da aber $y_M$ das gleiche Vorzeichen wie $d_M=x_N+c$ hat, sind die 
Betragsstriche nicht notwendig:
\begin{alignat*}{2}
    x_N+c&=y_M\cdot\sqrt{1+\frac{1}{r^2}}&&\\
    \Leftrightarrow y_M&=\frac{x_N+c}{\sqrt{1+\frac{1}{r^2}}}&&=\frac{r\cdot (x_N+c)}{r\cdot\sqrt{1+\frac{1}{r^2}}}\\
    &=\frac{r(x_N+c)}{\sqrt{r^2\left(1+\frac{1}{r^2}\right)}}&&=\frac{r(x_N+c)}{\sqrt{r^2+1}}
\end{alignat*}
Es ist also: $y_M=\frac{r(x_N+c)}{\sqrt{r^2+1}}$.\\
Nun ist $x_M=\frac{y_M}{r}$, also mit $y_M=\frac{r(x_N+c)}{\sqrt{r^2+1}}$:
\[
x_M=\frac{1}{r}\cdot y_M=\frac{1}{r}\cdot \frac{r(x_N+c)}{\sqrt{r^2+1}}=\frac{x_N+c}{\sqrt{r^2+1}}
\]
Es ist also:
\[
M=\left( \frac{x_N+c}{\sqrt{r^2+1}} , \frac{r(x_N+c)}{\sqrt{r^2+1}} \right)
\]
Die Punkte $P, N, M$ haben also folgende Koordinaten, wenn $m$ nicht orthogonal zu $n$ ist:
\[
P=(0, 0); N=(x_N, 0); M=\left( \frac{x_N+c}{\sqrt{r^2+1}} , \frac{r(x_N+c)}{\sqrt{r^2+1}} \right)
\]

Wenn $m$ orthogonal zu $n$ ist, also $m$ auf der $y$-Achse liegt, hat $M$ natürlich eine $x$-Koordinate von 0. 
Es gilt aber immer noch $|x_N+c|=\sqrt{x_M^2+y_M^2}$, was mit $x_M=0$ dann $|x_N+c|=\sqrt{y_M^2}=|y_M|$ ergibt. Da 
$y_M$ das gleiche Vorzeichen wie $d_M=x_N+c$ hat, gilt die Gleichung auch ohne die Betragsstriche; es ist also $y_M
=x_N+c$. Zusammen mit $x_M=0$ ist dann also, wenn $m$ orthogonal zu $n$ ist, $M=(0, x_N+c)$.

\subsection*{Beweis der zu zeigenden Aussage}

Nun muss nur noch bewiesen werden, dass es stets einen festen Punkt $Q\neq P$ gibt, sodass $P, M, N, Q$ zu jedem 
Zeitpunkt auf einem Kreis gemeinsamen liegen. Es muss also stets einen Punkt $S$ geben, von dem $P, M, N, Q$ den 
Gleichen Abstand haben.

Zunächst beweise ich hierzu den folgenden Hilfssatz:
\begin{lem}\label{cor_hilfe}
    $x_A^2+y_A^2=2x_Ax_S+2y_Ay_S$ ist äquivalent zu $\overline{AS}=\sqrt{x_S^2+y_S^2}$ für alle 
    Punkte $A=(x_A, y_A), S=(x_S, y_S)$.
\end{lem}
\begin{proof}[Beweis des Hilfssatzes]
    \renewcommand{\qedsymbol}{$\square$}
    Die Behauptung ergibt sich direkt durch Umformen:
    \begin{alignat*}{3}
        &\overline{AS}=\sqrt{(x_A-x_S)^2+(y_A-y_S)^2}&&=\sqrt{x_S^2+y_S^2}\quad&&\|(\ldots)^2\\
        &\Leftrightarrow (x_A-x_S)^2+(y_A-y_S)^2&&=x_S^2+y_S^2\quad &&\|\text{2. Binomische Formel}\\
        &\Leftrightarrow (x_A^2-2x_Ax_S+x_S^2)+(y_A^2-2y_Ay_S+y_S^2)&&=x_S^2+y_S^2\quad &&\|-x_S^2-y_S^2\\
        &\Leftrightarrow x_A^2-2x_Ax_S+y_A^2-2y_Ay_S&&=0\quad &&\|+2x_Ax_S+2y_Ay_S\\
        &\Leftrightarrow x_A^2+y_A^2&&=2x_Ax_S+2y_Ay_S&&
    \end{alignat*}
\end{proof}
Der folgende Satz beschäftigt sich nur mit dem Fall "`$m$ orthogonal zu $n$"':
\begin{thm}\label{aufgabe_3}
    Wenn $m$ nicht orthogonal zu $n$ ist, gibt es einen festen Punkt $Q\neq P$ so, dass $P, M, N, Q$ 
    den gleichen Abstand zu einem Punkt $S$ haben, also stets auf einem gemeinsamen Kreis liegen. 
    Hierbei sind $S$ und $Q$ durch $S=\left(\frac{x_N}{2}, \frac{x_N+c}{2}\right)$ und $Q=\left(-\frac{c}{2}, 
    \frac{c}{2}\right)$ gegeben.
\end{thm}
\begin{proof}[Beweis des Satzes]
    \renewcommand{\qedsymbol}{$\square$}
    Anm.: Im folgenden Beweis ist natürlich $M=(0, x_N+c)$, da in Satz \ref{aufgabe_3} davon ausgegangen wird, dass 
    $m$ orthogonal zu $n$ ist.

    Wenn $P=Q$ wäre, wäre offensichtlich auch $x_P=x_Q$, also $x_Q=0$. Dann wäre aber auch $c=0$ und damit $d_M
    =x_N+c=x_N$. Wenn $N$ den Punkt $P$ passiert, also $x_N=0$ ist, wäre also auch $d_M=0$ und damit auch $|d_M|
    =\overline{PM}=0$. Damit  würden aber $M$ und $N$ beide auf $P$ liegen, also würden $M$ und $N$ gleichzeitig 
    den Punkt $P$ passieren, was ja nicht sein kann. Also muss $P\neq Q$ sein. Auch bewegt sich $Q$ offensichtlich 
    nicht, da $c$ und damit auch $\frac{c}{2}$ und $-\frac{c}{2}$ konstant sind.

    Offensichtlich bewegt sich der Punkt $Q$ nicht, denn seine Koordinaten sind nur von der Konstanten $c$ abhängig. 
    $Q$ ist also ein fester, von $P$ verschiedener Punkt.
    
    Zunächst werde ich ein paar Gleichungen in den folgenden Hilfssätzen beweisen, aus denen dann mit Hilfssatz 
    \ref{cor_hilfe}
    direkt die Korollare folgen:
    \begin{lem}\label{no_P}
        Es ist $\overline{PS}=\sqrt{x_S^2+y_S^2}$.
    \end{lem}
    \begin{proof}[Beweis des Hilfssatzes]
        Die Behauptung ergibt sich direkt mit der Distanzformel:
        \[
        \overline{PS}=\sqrt{(x_S-x_P)^2+(y_S-y_P)^2}=\sqrt{(x_S-0)^2+(y_S-0)^2}=\sqrt{x_S^2+y_S^2}    
        \]
    \end{proof}
    \begin{lem}\label{no_M}
        Es ist $\overline{MS}=\sqrt{x_S^2+y_S^2}$.
    \end{lem}
    \begin{proof}[Beweis des Hilfssatzes]
        Durch Umformen ergibt sich direkt die Gleichung $x_M^2+y_M^2=2x_Mx_S+2y_My_S$:
        \begin{align*}
            x_M^2+y_M^2&=0^2+(x_N+c)^2=0+2\cdot \frac{(x_N+c)^2}{2}=0+2\cdot (x_N+c)\cdot\frac{x_N+c}{2}\\
            &=2\cdot 0\cdot \frac{x_N}{2}+2\cdot y_M\cdot y_S=2x_Mx_S+2y_My_S
        \end{align*}
        Aus der Gleichung $x_M^2+y_M^2=2x_Mx_S+2y_My_S$ ergibt sich mit Hilfssatz \ref{cor_hilfe} direkt die Behauptung.
    \end{proof}
    \begin{lem}\label{no_N}
        Es ist $\overline{NS}=\sqrt{x_S^2+y_S^2}$.
    \end{lem}
    \begin{proof}[Beweis des Hilfssatzes]
        Durch Umformen ergibt sich direkt die Gleichung $x_N^2+y_N^2=2x_Nx_S+2y_Ny_S$:
        \begin{align*}
            x_N^2+y_N^2&=x_N^2+0^2=x_N^2+0=2\cdot \frac{x_N^2}{2}+0=2x_N\frac{x_N}{2}+0\\
            &=2x_Nx_S+2\cdot 0\cdot \frac{x_N+c}{2}=2x_Nx_S+2y_Ny_S
        \end{align*}
        Aus der Gleichung $x_N^2+y_N^2=2x_Nx_S+2y_Ny_S$ ergibt sich mit Hilfssatz \ref{cor_hilfe} direkt die Behauptung.
    \end{proof}
    \begin{lem}\label{no_Q}
        Es ist $\overline{QS}=\sqrt{x_S^2+y_S^2}$.
    \end{lem}
    \begin{proof}[Beweis des Hilfssatzes]
        Durch Umformen ergibt sich direkt die Gleichung $x_Q^2+y_Q^2=2x_Qx_S+2y_Qy_S$:
        \begin{align*}
            x_Q^2+y_Q^2&=\left(-\frac{c}{2}\right)^2+\left(\frac{c}{2}\right)^2=\frac{c^2}{4}+\frac{c^2}{4}
            =2\cdot \frac{c^2}{4}=\frac{c^2}{2}=0\cdot \frac{c}{2}+\frac{c^2}{2}\\
            &=(-x_N+x_N)\frac{c}{2}+\frac{c^2}{2}=-x_N\frac{c}{2}+x_N\frac{c}{2}+c\cdot\frac{c}{2}\\
            &=x_N\left(-\frac{c}{2}\right)+(x_N+c)\frac{c}{2}=2\frac{x_N}{2}
            \left(-\frac{c}{2}\right)+2\frac{x_N+c}{2}\frac{c}{2}\\
            &=2x_Sx_Q+2y_Sy_Q=2x_Qx_S+2y_Qy_S
        \end{align*}
        Aus der Gleichung $x_Q^2+y_Q^2=2x_Qx_S+2y_Qy_S$ ergibt sich mit Hilfssatz \ref{cor_hilfe} direkt die Behauptung.
    \end{proof}
    Aus den Hilfssätzen \ref{no_P}, \ref{no_M}, \ref{no_N} und \ref{no_Q} folgt direkt: 
    $\overline{PS}=\overline{MS}=\overline{NS}=\overline{QS}=\sqrt{x_S^2+y_S^2}$. Wenn $n$ orthogonal zu $m$ ist, 
    gibt es also einen festen Punkt $Q\neq P$ so, dass $P, M, N, Q$ den gleichen Abstand zu einem Punkt $S$ haben, 
    also stets auf einem gemeinsamen Kreis liegen.
    \renewcommand{\qedsymbol}{$\blacksquare$}
\end{proof}

Nun muss noch bewiesen werden, dass es einen solchen Punkt $Q$ auch gibt, wenn $m$ nicht orthogonal zu $m$ ist:

\begin{thm}\label{augabe_3_dumm}
    Wenn $m$ nicht orthogonal zu $n$ ist, gibt es einen festen Punkt $Q\neq P$ so, dass $P, M, N, Q$ 
    den gleichen Abstand zu einem Punkt $S$ haben, also stets auf einem gemeinsamen Kreis liegen. 
    Hierbei sind $S$ und $Q$ durch
    \[
    S=\left(\frac{x_N}{2},\frac{(x_N+c)\sqrt{r^2+1}-x_N}{2r}\right)\text{ und}
    \]
    \[
    Q=\left(\frac{c\sqrt{r^2+1}\left(1-\sqrt{r^2+1}\right)}{\left(1-\sqrt{r^2+1}\right)^2+r^2}, 
    \frac{cr\sqrt{r^2+1}}{\left(1-\sqrt{r^2+1}\right)^2+r^2}\right)
    \]
    gegeben.
\end{thm}
\begin{proof}[Beweis des Satzes]
    \renewcommand{\qedsymbol}{$\square$}
    Anm.: Im folgenden Beweis ist natürlich $M=\left( \frac{x_N+c}{\sqrt{r^2+1}} , \frac{r(x_N+c)}{\sqrt{r^2+1}} 
    \right)$, da in Satz \ref{augabe_3_dumm} davon ausgegangen wird, dass $m$ nicht orthogonal zu $n$ ist.

    Wenn $P=Q$ wäre, müsste auch $y_Q=y_P=0$ seinn. Es müsste also $\frac{cr\sqrt{r^2+1}}{(1-\sqrt{r^2+1})^2+r^2}
    =0$ gelten. Nach dem Satz vom Nullprodukt muss dann mindestens einer der Faktoren $c, r, \sqrt{r^2+1}, 
    \frac{1}{(1-\sqrt{r^2+1})^2+r^2}$ gleich 0 sein. Offensichtlich kann $\frac{1}{(1-\sqrt{r^2+1})^2+r^2}$ nicht 0 
    sein. Auch $\sqrt{r^2+1}$ ist offensichtlich positiv, also nicht 0. Wenn $r$, die Steigung von $m$, 0 wäre, 
    würde das bedeuten, dass $m$ auf der $x$-Achse, also $n$, liegt. Das kann aber ebenfalls nicht sein, denn $n$ 
    und $m$ schneiden sich nur in einem Punkt. Es muss also, damit $y_Q=0$ sein kann, $c=0$ gelten. Dann müssten 
    aber (wie am Anfang des Beweises von Satz \ref{aufgabe_3} bewiesen) $M$ und $N$ gleichzeitig den Punkt $P$ 
    passieren, was ebenfalls nicht sein kann! Es kann also nicht $y_M=0$ sein, also kann auch nicht $P=Q$ sein.

    Auch bewegt sich der Punkt $Q$ offensichtlich nicht, denn seine Koordinaten sind nur von den Konstanten $c$ und 
    $r$ abhängig. Also ist $Q$ ein fester, von $P$ verschiedener Punkt.

    Wie im Beweis von Satz \ref{aufgabe_3} folgen nun Beweise von ein paar Gleichungen:
    \begin{lem}\label{dumm_P}
        Es ist $\overline{PS}=\sqrt{x_S^2+y_S^2}$.
    \end{lem}
    \begin{proof}[Beweis des Hilfssatzes]
        Die Behauptung ergibt sich direkt mit der Distanzformel:
        \[
        \overline{PS}=\sqrt{(x_S-x_P)^2+(y_S-y_P)^2}=\sqrt{(x_S-0)^2+(y_S+0)^2}=\sqrt{x_S^2+y_S^2}    
        \]
    \end{proof}
    \begin{lem}\label{dumm_M}
        Es ist $\overline{MS}=\sqrt{x_S^2+y_S^2}$.
    \end{lem}
    \begin{proof}[Beweis des Hilfssatzes]
        Zunächst ergibt sich durch Umformen die Gleichung $x_M^2+y_M^2=2x_Mx_S+2y_My_S$. Diese Umformung ist 
        zwar etwas länger, allerdings in jedem Schritt elementar genug, dass ich einen Kommentar für überflüssig 
        halte.

        \begin{align*}
            x_M^2+y_M^2&=\left(\frac{x_N+c}{\sqrt{r^2+1}}\right)^2+\left(\frac{r(x_N+c)}{\sqrt{r^2+1}}\right)^2
            =\frac{(x_N+c)^2}{r^2+1}+\frac{r^2(x_N+c)^2}{r^2+1}\\
            &=\frac{(x_N+c)^2+r^2(x_N+c)^2}{r^2+1}=\frac{(1+r^2)(x_N+c)^2}{r^2+1}=(x_N+c)^2\\
            &=\frac{(x_N+c)x_N}{\sqrt{r^2+1}}-\frac{(x_N+c)x_N}{\sqrt{r^2+1}}+(x_N+c)^2\\
            &=\frac{(x_N+c)x_N}{\sqrt{r^2+1}}-\frac{(x_N+c)x_N}{\sqrt{r^2+1}}+\frac{(x_N+c)^2\sqrt{r^2+1}}{\sqrt{r^2+1}}\\
            &=\frac{(x_N+c)x_N}{\sqrt{r^2+1}}+\frac{-(x_N+c)x_N+(x_N+c)^2\sqrt{r^2+1}}{\sqrt{r^2+1}}\\
            &=\frac{(x_N+c)x_N}{\sqrt{r^2+1}}+\frac{(x_N+c)^2\sqrt{r^2+1}-(x_N+c)x_N}{\sqrt{r^2+1}}\\
            &=\frac{(x_N+c)x_N}{\sqrt{r^2+1}}+\frac{(x_N+c)\cdot(x_N+c)\sqrt{r^2+1}-(x_N+c)\cdot x_N}{\sqrt{r^2+1}}\\
            &=\frac{(x_N+c)x_N}{\sqrt{r^2+1}}+\frac{x_N+c}{\sqrt{r^2+1}}\cdot \left((x_N+c)\sqrt{r^2+1}-x_N\right)\\
            &=\frac{x_N+c}{\sqrt{r^2+1}}\cdot x_N+2r\cdot\frac{x_N+c}{\sqrt{r^2+1}}\cdot\frac{(x_N+c)\sqrt{r^2+1}-x_N}{2r}\\
            &=2\cdot x_M\cdot\frac{x_N}{2}+2\cdot\frac{r(x_N+c)}{\sqrt{r^2+1}}\cdot y_S\\
            &=2x_Mx_S+2y_My_S
        \end{align*}

        Aus $x_M^2+y_M^2=2x_Mx_S+2y_My_S$ ergibt sich mit Hilfssatz \ref{cor_hilfe} direkt die Behauptung.
    \end{proof}
    \begin{lem}\label{dumm_N}
        Es ist $\overline{NS}=\sqrt{x_S^2+y_S^2}$.
    \end{lem}
    \begin{proof}[Beweis des Hilfssatzes]
        Die Behauptung ergibt sich direkt mit der Distanzformel:
        \begin{align*}
            \overline{NS}&=\sqrt{(x_N-x_S)^2+(y_N-y_S)^2}=\sqrt{\left(x_N-\frac{x_N}{2}\right)^2+(0-y_S)^2}\\
            &=\sqrt{\left(\frac{x_N}{2}\right)^2+(-y_S)^2}=\sqrt{x_S^2+y_S^2}
        \end{align*}
    \end{proof}
    \begin{lem}\label{dumm_Q}
        Es ist $\overline{QS}=\sqrt{x_S^2+y_S^2}$.
    \end{lem}
    \begin{proof}[Beweis des Hilfssatzes]
        Zunächst ergibt sich durch simples Umformen die Gleichung $x_Q^2+y_Q^2=2x_Qx_S+2y_Qy_S$. Auch diese 
        Umformung ist ziemlich lang, aber auch hier ist wieder jeder Schritt sehr elementar; die Umformung sollte 
        also trotz ihre Länge nachvollziehbar sein.

        % TODO: Umformung einfügen. Zu finden unter Downloads --> Umformung.jpeg

        Aus der Gleichung $x_Q^2+y_Q^2=2x_Qx_S+2y_Qy_S$ ergibt sich mit Hilfssatz \ref{cor_hilfe} direkt die Behauptung.
    \end{proof}
    Aus den Hilfssätzen \ref{dumm_P}, \ref{dumm_M}, \ref{dumm_N} und \ref{dumm_Q} folgt direkt: 
    $\overline{PS}=\overline{MS}=\overline{NS}=\overline{QS}=\sqrt{x_S^2+y_S^2}$. Wenn $n$ nicht orthogonal zu $m$ 
    ist, gibt es also einen festen Punkt $Q\neq P$ so, dass $P, M, N, Q$ den gleichen Abstand zu einem Punkt $S$ 
    haben, also stets auf einem gemeinsamen Kreis liegen.
    \renewcommand{\qedsymbol}{$\blacksquare$}
\end{proof}

Nach Satz \ref{aufgabe_3} gibt es also,  wenn $m$ orthogonal zu $n$ ist, einen festen Punkt $Q\neq P$ so, dass $P, M, 
N, Q$ stets den gleichen Abstand zu einem Punkt $S$ haben, also stets auf einem gemeinsamen Kreis liegen. 
Nach Satz \ref{augabe_3_dumm} gibt es einen solchen Punkt $Q$ auch, wenn $m$ nicht orthogonal zu $n$ ist. Es gibt 
also stets einen festen Punkt $Q\neq P$ so, dass $P, M, N, Q$ zu jedem Zeitpunkt auf einem gemeinsamen Kreis liegen!