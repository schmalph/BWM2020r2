\section{Aufgabe 3}

\subsection*{Anmerkungen zur Notation}

Im Folgenden wird $x_A$ die $x$- und $y_A$ die $y$-Koordinate des allgemeinen Punktes $A$ bezeichnen. D.h. hier, dass 
$P=(x_P, y_P), N=(x_N, y_N), M=(x_M, y_M), S=(x_S, y_S), Q=(x_Q, y_Q)$.\\
$\overline{AB}$ ist die Distanz von $A$ zu $B$ (stets positiv).

\subsection*{"`Algebraisierung"' der Aufgabe mit Hilfe eines kartesischen Koordinatensystems}

\subsubsection*{Definition des Koordinatensystems}

Um die Aufgabe von einer Geometrischen in eine eher Algebraische zu übersetzen, habe ich mich dazu entschieden, die 
in der Aufgabenstellung beschriebene Konstruktion mit Hilfe eines kartesischen Koordinatensystems zu beschreiben. 
Hierzu definiere man folgendermaßen ein kartesisches Koordinatensystem: $P$ sei der Urspung und $n$ die $x$-Achse; 
dabei sei die $x$-Achse so beschriftet, dass der Punkt $N$ eine positive $x$-Koordinate hat, bevor er $P$ passiert, 
und die $y$-Achse so, dass der Punkt $M$ eine positive $y$-Koordinate hat, bevor er $P$ passiert. Dies ist 
offensichtlich möglich, da $P$ auf $n$ liegt. Offensichtlich ost auch die Beschriftung der $y$-Achse (also der 
Orthonogonalen zu $m$, die durch $P$ geht) möglich, denn die Geraden $m$ und $n$ liegen nicht aufeinander

\subsubsection*{Direkte Folgerungen aus der Definition des Koordinatensystems}

Der Punkt $N$ hat, da er auf $n$, also der $x$-Achse, liegt, natürlich eine $y$-Koordinate von 0. Es ist also: 
$N=(x_N, 0)$. Die Gerade $m$ ist, da sie durch $P$, den Urspung des kartesischen Koordinatensystems, geht, natürlich 
eine Urspungsgerade. Im folgenden werde ich nur die Fälle betrachten, in denen $m$ nicht auf der $y$-Achse liegt, 
also nicht orthogonal zu $n$ ist. Dann ist $m$ eine Ursprungsgerade (da $P$ auf $m$ liegt) mit reeler Steigung, die 
im Folgenden $r$ genannt wird. Ihre Geradengleichung ist also: $y=r\cdot x$. Sei nun $d_M$ die Distanz $\overline{PM}$, 
allerdings mit negativem Vorzeichen, wenn $y_M$ ein negatives Vorzeichen hat, also nachdem $M$ den Punkt $P$ passiert 
hat.