\section{Aufgabe 3}

\subsection*{Anmerkungen zur Notation und einer häufig verwendeten Formel}

Im Folgenden wird $x_A$ die $x$- und $y_A$ die $y$-Koordinate des allgemeinen Punktes $A$ bezeichnen. D.h. hier, dass 
$P=(x_P, y_P), N=(x_N, y_N), M=(x_M, y_M), S=(x_S, y_S), Q=(x_Q, y_Q)$.\\
$\overline{AB}$ ist die Distanz von $A$ zu $B$ (stets positiv).

Sehr oft habe ich die Distanzformel 
\[\overline{AB}=\sqrt{(x_A-x_B)^2+(y_A-y_B)^2}=\sqrt{(x_B-x_A)^2+(y_B-y_A)^2}\text{    (für alle $A,B$)}\] 
verwendet. Ich habe bei den Gleichungen, bei denen ich sie verwendet habe, nicht dazugeschrieben, dass sich die 
Gleichungen mit der Distanzformel ergeben. Diese Entscheidung habe ich getroffen, da ich diese Formel so oft verwendet 
habe, dass, wenn ich sie jedes mal erwähnt hätte, dies einige Sätze deutlich komplizierter und teilweise sogar 
relativ unverständlich gemacht hätte. Ich denke, dass dies nicht problematisch ist; ich habe ja auch nicht dazugeschrieben, 
wenn ich z.B. einen Bruch gekürzt habe.

\subsection*{Algebraisierung der Aufgabe mit Hilfe eines kartesischen Koordinatensystems}

Um die Aufgabe von einer Geometrischen in eine eher Algebraische zu übersetzen, habe ich mich dazu entschieden, die 
in der Aufgabenstellung beschriebene Konstruktion mit Hilfe eines kartesischen Koordinatensystems zu beschreiben. 
Hierzu definiere man folgendermaßen ein kartesisches Koordinatensystem: $P$ sei der Urspung und $n$ die $x$-Achse; 
dabei sei die $x$-Achse so beschriftet, dass der Punkt $N$ eine positive $x$-Koordinate hat, bevor er $P$ passiert, 
und die $y$-Achse so, dass der Punkt $M$ eine positive $y$-Koordinate hat, bevor er $P$ passiert. Dies ist 
offensichtlich möglich, da $P$ auf $n$ liegt. Offensichtlich ist auch die Beschriftung der $y$-Achse (also der 
Orthogonalen zu $n$, die durch $P$ geht) möglich, denn die Geraden $m$ und $n$ liegen nicht aufeinander (sonst 
würden sie sich in mehr als einem Punkt schneiden); der Punkt $M$ muss also, bevor er $P$ passiert, auf einer Seite 
der $x$-Achse liegen.

Der Punkt $N$ hat, da er auf $n$, also der $x$-Achse, liegt, natürlich eine $y$-Koordinate von 0. Es ist also: 
$N=(x_N, 0)$. Sei nun $d_M$ die Distanz $\overline{PM}$, allerdings mit negativem Vorzeichen, wenn $y_M$ ein 
negatives Vorzeichen hat, also nachdem $M$ den Punkt $P$ passiert hat. Nun ist $x_N$ wegen $y_N=0$ die Distanz 
$\overline{PN}$, allerdings mit negativem Vorzeichen, nachdem $N$ den Punkt $P$ passiert hat. Offensichtlich ist 
dann, da $N$ und $M$ sich mit der gleichen, konstanten Geschwindigkeit von $P$ weg/zu $P$ hin bewegen, $d_M-x_N$ 
konstant. Sei diese Konstante mit $c$ bezeichnet. Dann ist $d_M-x_N=c\Leftrightarrow d_M=x_N+c$ und damit 
$\overline{PM}=|x_N+c|$. Nun ist $|x_N+c|=\overline{PM}=\sqrt{(x_M-x_P)^2+(y_M-y_P)^2}=\sqrt{x_M^2+y_M^2}$, da 
$y_P=x_P=0$. Die Gerade $m$ ist, da sie durch $P$, den Urspung des kartesischen Koordinatensystems, geht, natürlich 
eine Urspungsgerade. Im Folgenden werde ich nur die Fälle betrachten, in denen $m$ nicht auf der $y$-Achse liegt, 
also nicht orthogonal zu $n$ ist. Dann hat $m$ eine reele Steigung, die nicht Null ist, da $m$ sonst auf $n$ liegen 
würde, was einen Widerspruch dazu darstellt, dass $m$ und $n$ sich nur in einem Punkt $P$ schneiden. Diese Steigung 
wird im Folgenden $r$ genannt. Die Geradengleichung von $m$ ist also: $y=r\cdot x$ bzw. $x=\frac{y}{r}$. Da $M$ auf 
$m$ liegt, muss $x_M=\frac{y_M}{r}$ gelten. Zusammen mit $|x_N+c|=\sqrt{x_M^2+y_M^2}$ ergibt das dann: $|x_N+c|
=\sqrt{x_M^2+y_M^2}=\sqrt{(\frac{y_M}{r})^2+y_M^2}=\sqrt{\frac{1}{r^2}\cdot y_M^2+y_M^2}=\sqrt{(1+\frac{1}{r^2})
\cdot y_m^2}=|y_M|\cdot\sqrt{1+\frac{1}{r^2}}$. Da aber $y_M$ das gleiche Vorzeichen wie $d_M=x_N+c$ hat, sind die 
Betragsstriche nicht notwendig:
\begin{alignat*}{2}
    x_N+c&=y_M\cdot\sqrt{1+\frac{1}{r^2}}&&\\
    \Leftrightarrow y_M&=\frac{x_N+c}{\sqrt{1+\frac{1}{r^2}}}&&=\frac{r\cdot (x_N+c)}{r\cdot\sqrt{1+\frac{1}{r^2}}}\\
    &=\frac{r(x_N+c)}{\sqrt{r^2\left(1+\frac{1}{r^2}\right)}}&&=\frac{r(x_N+c)}{\sqrt{r^2+1}}
\end{alignat*}
Es ist also: $y_M=\frac{r(x_N+c)}{\sqrt{r^2+1}}$.\\
Nun ist $x_M=\frac{y_M}{r}$, also mit $y_M=\frac{r(x_N+c)}{\sqrt{r^2+1}}$:
\[
x_M=\frac{1}{r}\cdot y_M=\frac{1}{r}\cdot \frac{r(x_N+c)}{\sqrt{r^2+1}}=\frac{x_N+c}{\sqrt{r^2+1}}
\]
Es ist also:
\[
M=\left( \frac{x_N+c}{\sqrt{r^2+1}} , \frac{r(x_N+c)}{\sqrt{r^2+1}} \right)
\]
Die Punkte $P, N, M$ haben also folgende Koordinaten, wenn $m$ nicht orthogonal zu $n$ ist:
\[
P=(0, 0); N=(x_N, 0); M=\left( \frac{x_N+c}{\sqrt{r^2+1}} , \frac{r(x_N+c)}{\sqrt{r^2+1}} \right)
\]

Wenn $m$ orthogonal zu $n$ ist, also $m$ auf der $y$-Achse liegt, hat $M$ natürlich eine $x$-Koordinate von 0. 
Es gilt aber immer noch $|x_N+c|=\sqrt{x_M^2+y_M^2}$, was mit $x_M=0$ dann $|x_N+c|=\sqrt{y_M^2}=|y_M|$ ergibt. Da 
$y_M$ das gleiche Vorzeichen wie $d_M=x_N+c$ hat, gilt die Gleichung auch ohne die Betragsstriche; es ist also $y_M
=x_N+c$. Zusammen mit $x_M=0$ ist dann also, wenn $m$ orthogonal zu $n$ ist, $M=(0, x_N+c)$.

\subsection*{Beweis der zu zeigenden Aussage}

Nun muss nur noch bewiesen werden, dass es einen Punkt $Q$ gibt, sodass $P, M, N, Q$ stets auf einem Kreis liegen. 
Es muss also stets einen Punkt $S$ geben, von dem $P, M, N, Q$ den Gleichen Abstand haben.

Folgender Satz wird später hilfreich sein:
\begin{thm}\label{cor_hilfe}
        $x_A^2+y_A^2=2x_Ax_S+2y_Ay_S$ ist äquivalent zu $\overline{AS}=\sqrt{x_S^2+y_S^2}$ für alle 
        Punkte $A=(x_A, y_A), S=(x_S, y_S)$.
\end{thm}
\begin{proof}
    Die Behauptung ergibt sich direkt durch Umformen:
    \begin{alignat*}{3}
        &\overline{AS}=\sqrt{(x_A-x_S)^2+(y_A-y_S)^2}&&=\sqrt{x_S^2+y_S^2}\quad&&\|(\ldots)^2\\
        &\Leftrightarrow (x_A-x_S)^2+(y_A-y_S)^2&&=x_S^2+y_S^2\quad &&\|\text{2. Binomische Formel}\\
        &\Leftrightarrow (x_A^2-2x_Ax_S+x_S^2)+(y_A^2-2y_Ay_S+y_S^2)&&=x_S^2+y_S^2\quad &&\|-x_S^2-y_S^2\\
        &\Leftrightarrow x_A^2-2x_Ax_S+y_A^2-2y_Ay_S&&=0\quad &&\|+2x_Ax_S+2y_Ay_S\\
        &\Leftrightarrow x_A^2+y_A^2&&=2x_Ax_S+2y_Ay_S&&
    \end{alignat*}
\end{proof}
Der folgende Satz beschäftigt sich nur mit dem Fall "`$m$ orthogonal zu $n$"':
\begin{thm}\label{aufgabe_3}
    Die Punkte $P, M, N$ haben stets den gleichen Abstand zu einem Punkt $S$ wie der Punkt $Q\neq P$, wenn $m$ 
    orthogonal zu $n$ ist. Hierbei ist $S=\left(\frac{x_N}{2}, \frac{x_N+c}{2}\right)$ und $Q=\left(-\frac{c}{2}, 
    \frac{c}{2}\right)$.
\end{thm}
\begin{proof}
    \renewcommand{\qedsymbol}{$\square$}
    Wenn $P=Q$ wäre, wäre offensichtlich auch $x_P=x_Q$, also $x_Q=0$. Dann wäre aber auch $c=0$ und damit $d_M
    =x_N+c=x_N$. Wenn $N$ den Punkt $P$ passiert, also $x_N=0$ ist, wäre also auch $d_M=0$ und damit auch $|d_M|
    =\overline{PM}=0$. Damit  würden aber $M$ und $N$ beide auf $P$ liegen, also würden $M$ und $N$ gleichzeitig 
    den Punkt $P$ passieren, was ja nicht sein kann. Also muss $P\neq Q$ sein. Auch bewegt sich $Q$ offensichtlich 
    nicht, da $c$ und damit auch $\frac{c}{2}$ und $-\frac{c}{2}$ konstant sind.
    
    Zunächst werde ich ein paar Gleichungen in den folgenden Lemmata beweisen, aus denen dann mit Satz \ref{cor_hilfe}
    direkt die Korollare folgen:
    \begin{lem}\label{no_P}
        Es ist $x_P^2+y_P^2=2x_Px_S+2y_Py_S$.
    \end{lem}
    \begin{proof}
        Die Behauptung ergibt sich direkt durch Umformen:
        \begin{align*}
            x_P^2+y_P^2&=0^2+0^2\\
            =0=2\cdot 0\cdot x_S+2\cdot 0\cdot y_S&=2x_Px_S+2y_Py_S
        \end{align*}
    \end{proof}
    \begin{cor}\label{noc_P}
        Mit Satz \ref{cor_hilfe} folgt aus Lemma \ref{no_P} direkt: $\overline{PS}=\sqrt{x_S^2+y_S^2}$.
    \end{cor}
    \begin{lem}\label{no_M}
        Es ist $x_M^2+y_M^2=2x_Mx_S+2y_My_S$.
    \end{lem}
    \begin{proof}
        Die Behauptung ergibt sich direkt durch Umformen:
        \begin{align*}
            x_M^2+y_M^2&=0^2+(x_N+c)^2=0+2\cdot \frac{(x_N+c)^2}{2}=0+2\cdot (x_N+c)\cdot\frac{x_N+c}{2}\\
            &=2\cdot 0\cdot \frac{x_N}{2}+2\cdot y_M\cdot y_S=2x_Mx_S+2y_My_S
        \end{align*}
    \end{proof}
    \begin{cor}\label{noc_M}
        Mit Satz \ref{cor_hilfe} folgt aus Lemma \ref{no_M} direkt: $\overline{MS}=\sqrt{x_S^2+y_S^2}$
    \end{cor}
    \begin{lem}\label{no_N}
        Es ist $x_N^2+y_N^2=2x_Nx_S+2y_Ny_S$.
    \end{lem}
    \begin{proof}
        Die Behauptung ergibt sich direkt durch Umformen:
        \begin{align*}
            x_N^2+y_N^2&=x_N^2+0^2=x_N^2+0=2\cdot \frac{x_N^2}{2}+0=2x_N\frac{x_N}{2}+0\\
            &=2x_Nx_S+2\cdot 0\cdot \frac{x_N+c}{2}=2x_Nx_S+2y_Ny_S
        \end{align*}
    \end{proof}
    \begin{cor}\label{noc_N}
        Mit Satz \ref{cor_hilfe} folgt aus Lemma \ref{no_N} direkt: $\overline{NS}=\sqrt{x_S^2+y_S^2}$.
    \end{cor}
    \begin{lem}\label{no_Q}
        Es ist $x_Q^2+y_Q^2=2x_Qx_S+2y_Qy_S$.
    \end{lem}
    \begin{proof}
        Die Behauptung ergibt sich direkt durch Umformen:
        \begin{align*}
            x_Q^2+y_Q^2&=\left(-\frac{c}{2}\right)^2+\left(\frac{c}{2}\right)^2=\frac{c^2}{4}+\frac{c^2}{4}\\
            &=2\cdot \frac{c^2}{4}=\frac{c^2}{2}=0\cdot \frac{c}{2}+\frac{c^2}{2}=(-x_N+x_N)\frac{c}{2}+\frac{c^2}{2}\\
            &=-x_N\frac{c}{2}+x_N\frac{c}{2}+c\cdot\frac{c}{2}=x_N\left(-\frac{c}{2}\right)+(x_N+c)\frac{c}{2}\\
            &=2\frac{x_N}{2}\left(-\frac{c}{2}\right)+2\frac{x_N+c}{2}\frac{c}{2}=2x_Sx_Q+2y_Sy_Q\\
            &=2x_Qx_S+2y_Qy_S
        \end{align*}
    \end{proof}
    \begin{cor}\label{noc_Q}
        Mit Satz \ref{cor_hilfe}  folgt aus Lemma \ref{no_Q} direkt: $\overline{QS}=\sqrt{x_S^2+y_S^2}$
    \end{cor}
    Aus den Korollaren \ref{noc_P}, \ref{noc_M}, \ref{noc_N} und \ref{noc_Q} folgt direkt: 
    $\overline{PS}=\overline{MS}=\overline{NS}=\overline{QS}$, da sie alle gleich $\sqrt{x_S^2+y_S^2}$ sind. Die 
    Punkte $P, M, N, Q$ haben also alle den gleichen Abstand zu $S$, wenn $m$ orthogonal zu $n$ ist.
    \renewcommand{\qedsymbol}{$\blacksquare$}
\end{proof}

Wenn $n$ orthogonal zu $m$ ist, gibt es also eine Punkt $Q$, der sich nicht bewegt, sodass $P, M, N, Q$ den gleichen 
Abstand zu einem Punkt $S$ haben, also auf einem Kreis liegen.

Nun muss noch bewiesen werden, dass es auch einen solchen Punkt $Q$ gibt, wenn $m$ nicht orthogonal zu $m$ ist:

\begin{thm}
    Wenn $m$ nicht orthogonal zu $n$ ist, haben die Punkte $P, M, N$ den gleichen Abstand zu einem Punkt $S$ wie 
    ein Punkt $Q\neq P$
\end{thm}