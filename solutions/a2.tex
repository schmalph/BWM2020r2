\section{Aufgabe 2}

\subsection*{Anmerkungen zur Notation}

Im Folgenden wird $x_A$ die $x$- und $y_A$ die $y$-Koordinate des allgemeinen Punktes $A$ bezeichnen. D.h. hier, dass 
$P=(x_P, y_P), N=(x_N, y_N), M=(x_M, y_M), S=(x_S, y_S), Q=(x_Q, y_Q)$.

\subsection*{"`Algebraisierung"' der Aufgabe mit Hilfe eines kartesischen Koordinatensystems}

\subsubsection*{Definition des Koordinatensystems}

Um die Aufgabe von einer Geometrischen in eine eher Algebraische zu übersetzen, habe ich mich dazu entschieden, die 
in der Aufgabenstellung beschriebene Konstruktion mit Hilfe eines kartesischen Koordinatensystems zu beschreiben. 
Hierzu lege man ein kartesisches Koordinatensystem so auf die Ebene, dass der Ursprung auf $P$ und die $x$-Achse auf 
$n$ fällt. Dies ist offensichtlich möglich, da $P$ auf $n$ liegt. Die $x$-Achse soll hierbei so beschriftet sein, 
dass der Punkt $N$, bevor er $P$ passiert, eine positive $x$-Koordinate hat. Die $y$-Achse so, dass der Punkt $M$, 
bevor er $P$ passiert, eine positive $y$-Koordinate hat.

\subsubsection*{Direkte Folgerungen aus der Definition des Koordinatensystems}

Der Punkt $N$ hat, da er auf $n$, also der $x$-Achse, liegt, natürlich eine $y$-Koordinate von 0. Es ist also: 
$N=(x_N, 0)$. Die Gerade $m$ ist, da sie durch $P$, den Urspung des kartesischen Koordinatensystems, geht, natürlich 
eine Urspungsgerade. Sei ihre Steigung mit $r$ bezeichnet. Dann ist ihre Geradengleichung: $y=r\cdot x$. Da $M$ auf 
$m$ liegt, gilt also folgender Zusammenhang von $x$-und $y$-Koordinate von $M$: $y_M=r\cdot x_M$. Sei der Abstand 
von $P$ und $M$ in dem Moment, in dem $N$ auf $P$ liegt, mit $c$ bezeichnet und sei dieses $c$ mit $-1$ multipliziert, 
also sei $c=-\overline{PM}$, wenn $M$ zu diesem Zeitpunkt eine negative $y$-Koordinate hat.