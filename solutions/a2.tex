\section{Aufgabe 2}

\subsection*{Hinweise zur Darstellung}

$\gcd(\alpha, \beta)$ steht für den größten gemeinsamen Teiler (engl.: greatest common divisor) der Zahlen $\alpha$ 
und $\beta$. Also ist z.B. $\gcd(\alpha, \beta) = 1$ äquivalent zu "`$\alpha$ und $\beta$ sind teilerfremd"'.\\
"`O.B.d.A."' (oder mitten im Satz "`o.B.d.A."') ist kurz für "`ohne Beschränkung der Allgemeinheit"'.\\
$\Q$ ist die Menge der rationalen Zahlen.\\
$\N$ ist die Menge der natürlichen Zahlen (ohne die 0) und $\N_0$ die Menge der natürlichen Zahlen mit der 0.\\
$\alpha \mid \beta$ heißt, dass $\alpha$ ein Teiler von $\beta$, also $\beta$ ein Vielfaches von $\alpha$ ist.

\subsection*{Die tatsächliche Lösung von Aufgabe 3}

Die in Aufgabe 3 zu zeigende Aussage habe ich in folgenden Satz "`umformuliert"':

\begin{thm}
    Es gibt keine drei Zahlen $x, y, z \in \Q$ mit $x + y + z = 0$ und $x^2 + y^2 + z^2 = 100$. \label{aufgabe_3}
\end{thm}

\begin{proof}
    Es folgt ein Beweis durch Widerspruch. Nehme man hierzu an, es gäbe drei Zahlen $x, y, z \in \Q$, 
    die beide Gleichungen erfüllen. Offenbar können nicht alle drei Zahlen $x, y, z$ nicht-positiv und auch 
    nicht alle drei Zahlen $x, y, z$ nicht-negativ sein, denn damit dann $x+y+z = 0$ erfüllt wäre, müsste $x = y = z 
    = 0$, was der zweiten Gleichung $x^2+y^2+z^2 = 100$ widerspricht. Es sind also entweder zwei der drei Zahlen $x, 
    y, z$ nicht-negativ und eine nicht-positiv oder es sind zwei nicht-positiv und eine nicht-negativ. Unter der 
    Annahme, dass es ein Lösungstripel $x, y, z$ gibt, müsste es auch eines mit zwei nicht-negativen und einer 
    nicht-positiven Zahl geben. Denn zu jedem Lösungstripel $x, y, z$, dass zwei nicht-positive Zahlen und eine 
    nicht-negative Zahl beinhaltet, gibt es ein Lösungstripel $-x,-y,-z$ (das, wenn $x, y, z$ hinreichend für die 
    beiden Gleichungen ist, dies offensichtlich ebenfalls ist), das dann zwei nicht-negative Zahlen und eine 
    nicht-positive Zahl beinhaltet. Wenn es also überhaupt Lösungstripel gibt (wovon ja für den Widerspruchsbeweis 
    ausgegangen wurde), so muss es also auch welche mit zwei nicht-negativen Zahlen und einer nicht-positiven Zahl 
    geben. Sei nun $x, y, z$ ein solches Tripel und sei o.B.d.A. $z$ die nicht-positive Zahl (ansonsten benenne man 
    einfach um, die Gleichungen ändern dadurch ihre Form nicht). Es ist also $z \leq 0$ und $x, y \geq 0$. Da $x, y$ 
    nicht-negative rationale Zahlen sind, muss es vier Zahlen $a', b', c', d' \in \N_0$ ($b, d \neq 0$) geben, für die $x 
    = \frac{a'}{b'}$ und $y = \frac{c'}{d'}$. Sei nun $\frac{a}{b}$ die vollständig gekürzte Form des Bruches 
    $\frac{a'}{b'}$ und $\frac{c}{d}$ die des Bruches $\frac{c'}{d'}$. D.h., es ist $x = \frac{a}{b}$ und $y = 
    \frac{c}{d}$ mit $\gcd(a, b) = \gcd(c, d) = 1$. 
    %Sei nun o.B.d.A. $b \geq d$ (ansonsten bennen man einfach um, die Bedingungen für $x$ und $y$ und den jeweils 
    %zugehörigen Variablen $a, b, c, d$ ändern dadurch ihre Form nicht). 
    Nun ist wegen $x+y+z=0$ offenbar $z = -x-y = -\frac{a}{b} - \frac{c}{d} = -\frac{a \cdot d}{b \cdot d} - 
    \frac{c \cdot b}{d \cdot b} = - \frac{ad+cb}{db}$. Damit gilt dann wegen $x^2+y^2+z^2 = 100$:
    \begin{align*}
        100 = x^2+y^2+z^2 &= \left( \frac{a}{b} \right)^2 + \left( \frac{c}{d} \right)^2 + \left( -\frac{ad+cb}{db} \right)^2\\
        &= \left( \frac{a}{b} \right)^2 + \left( \frac{c}{d} \right)^2 + \left( \frac{ad+cb}{db} \right)^2\\
        &= \left( \frac{a \cdot d}{b \cdot d} \right)^2 + \left( \frac{c \cdot b}{d \cdot b} \right)^2 + \left( 
        \frac{ad+cb}{db} \right)^2\\
        &= \frac{(ad)^2}{(db)^2} + \frac{(cb)^2}{(db)^2} + \frac{(ad)^2 + 2abcd + (cb)^2}{(db)^2}\\
        &= \frac{2 \left( (ad)^2 + abcd + (cb)^2 \right)}{(db)^2} \quad \quad \quad \| \cdot \frac{(db)^2}{2}\\
        \iff 50 \cdot (db)^2 &= (ad)^2 + abcd + (cb)^2 \numberthis \label{frac_manip}
    \end{align*}\\
    Sei nun $\gcd(b, d) = t$ und seien die dann natürlichen Zahlen $\frac{b}{t}$ und $\frac{d}{t}$ mit $p$ bzw. $q$ 
    bezeichnet. $p$ und $q$ sind dann teilerfremde natürliche Zahlen. Wenn man nun $b = pt$ und $d = qt$ in 
    (\refeq{frac_manip}) einsetzt, erhält man:
    \begin{align*}
        50 \cdot (db)^2 &= (ad)^2 + abcd + (cb)^2\\
        = 50 \cdot p^2t^2 \cdot q^2 t^2 &= a^2 \cdot q^2 t^2 + ac \cdot pt \cdot qt + c^2 \cdot p^2t^2 \quad \| \cdot \frac{1}{t^2}\\
        \iff 50 \cdot t^2 \cdot  p^2 q^2 &= a^2 q^2 + acpq +c^2p^2 \numberthis \label{divisib}
    \end{align*}\\
    Nun ist in (\refeq{divisib}) offensichtlich die linke Seit durch $p$ und durch $q$ teilbar, also muss es die 
    rechte Seite auch sein. Der mittlere Summand $acpq$ muss dabei nicht beachtet werden, schließlich ist er bereits 
    durch sowohl $p$ als auch $q$ teilbar. Es muss also $p \mid a^2 q^2 + c^2 p^2$ und $q \mid a^2 q^2 + c^2 p^2$. 
    Aus $p \mid a^2 q^2 + c^2 p^2$ folgt wegen $p \mid c^2 p^2$ offenbar, dass $p \mid a^2 q^2$, während aus $q \mid 
    a^2 q^2 + c^2 p^2$ wegen $q \mid a^2 q^2$ folgt, dass $q \mid c^2 p^2$. Es gilt also:
    \[ p \mid a^2 q^2 \text{ und } q \mid c^2 p^2 \]
    Da aber $p$ und $q$ teilerfremd sind, folgt direkt:
    \[ p \mid a^2 \text{ und } q \mid c^2 \iff \frac{b}{t} \mid a^2 \text{ und } \frac{d}{t} \mid c^2 \]
    Nun sind aber $a$ und $b$ teilerfremd, d.h. kein Teiler von $b$ (was $\frac{b}{t}$ ja ist) außer 1 kann $a$ bzw. 
    $a^2$ teilen. Es muss also $\frac{b}{t} = 1$ und damit $b = t$. Da auch $q$ und $c$ teilerfremd sind, folgt mit 
    gleicher Begründung, dass $d = t$. Damit ist dann $b = d$. Setzt man dies in (\refeq{frac_manip}) ein, so erhält 
    man:
    \begin{align*}
        50 \cdot b^2 \cdot b^2 &= a^2 b^2 + ac \cdot b^2 + c^2 b^2 \quad \| \cdot \frac{1}{b^2}\\
        \iff 50b^2 &= a^2 + ac + c^2 \numberthis \label{even_odd}
    \end{align*}
    Nun ist die linke Seite von (\refeq{even_odd}) offensichtlich gerade und damit muss es die rechte auch sein. Wenn 
    nun aber beide Zahlen $a, c$ ungerade sind, so sind alle drei Summanden ungerade (das Produkt zweier ungerader 
    Zahlen ist stets eine ungerade Zahl) und damit dann die gesamte rechte Seite ungerade. Es muss also mindestens 
    eine der beiden Zahlen gerade sein. Wenn nun aber nur genau eine gerade ist, so ist auch das Quadrat dieser und 
    $ac$ gerade, jedoch nicht das Quadrat der anderen dann ungeraden Zahl. Die rechte Seite wäre die Summe zweier 
    gerader und  einer ungeraden Zahl, also auch ungerade. Da nicht beide und nicht genau eine der beiden Zahlen 
    ungerade sein können, müssen beide gerade sein. D.h., es gibt zwei Zahlen $r, s \in \N$, für die $a = 2r$ und 
    $c = 2s$ ist. Setzt man dies in (\refeq{even_odd}) ein, so erhält man:
    \begin{align*}
        50 b^2 &= (2r)^2 + (2r)(2s) + (2s)^2 = 4 \left( r^2 + rs + s^2 \right) \quad \| \cdot \frac12\\
        \iff 25b^2 &= 2 \left( r^2 + rs + s^2 \right)
    \end{align*}
    Da nun $25b^2$ anscheinend eine gerade Zahl ist (schließlich ist $2 \left( r^2 + rs + s^2 \right)$ offensichtlich 
    gerade), muss (da 25 ungerade ist) $b^2$ und damit $b$ erade sein. Nun sind also sowohl $a$ also auch $b$ gerade, 
    d.h. durch 2 teilbar. $a$ und $b$ haben also den gemeinsamen Teiler 2. Dies stellt aber einen Widerspruch zu 
    $\gcd(a, b) = 1$ dar! Denn offensichtlich ist ihr größter gemeinsame Teiler nicht 1, wenn 2 ein gemeinsamer 
    Teiler von $a$ und $b$ ist. Dies vervollständigt dann den Widerspruchsbeweis von Satz \ref{aufgabe_3}.
\end{proof}
