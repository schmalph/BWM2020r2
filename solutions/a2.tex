\section{Aufgabe 2}

\subsection*{Hinweise zur Darstellung}

$\gcd(\alpha, \beta)$ steht für den größten gemeinsamen Teiler (engl.: greatest common divisor) der Zahlen $\alpha$ 
und $\beta$. Also ist z.B. $\gcd(\alpha, \beta) = 1$ äquivalent zu "`$\alpha$ und $\beta$ sind teilerfremd"'.\\
$\Q$ ist die Menge der rationalen Zahlen.\\
$\mathbb{Z}$ ist die Menge der ganzen Zahlen.\\
$\alpha \mid \beta$ heißt, dass $\alpha$ ein Teiler von $\beta$, also $\beta$ ein Vielfaches von $\alpha$ ist.

\subsection*{Lösung von Aufgabe 2}

Die in Aufgabe 2 zu zeigende Aussage ist nun die des folgenden Satzes:

\begin{thm}
    Es gibt keine drei Zahlen $x, y, z \in \Q$ mit $x + y + z = 0$ und $x^2 + y^2 + z^2 = 100$. \label{aufgabe_2}
\end{thm}
\begin{proof}[Beweis des Satzes durch Widerspruch]
    Nehme man an, es gäbe drei Zahlen $x, y, z \in \Q$, die beide Gleichungen erfüllen. Da $x, y$ rationale Zahlen 
    sind, muss es vier Zahlen $a', b', c', d' \in \mathbb{Z}$ ($b', d' \neq 0$) geben, für die $x = \frac{a'}{b'}$ 
    und $y = \frac{c'}{d'}$. Sei nun $\frac{a}{b}$ die vollständig gekürzte Form des Bruches $\frac{a'}{b'}$ und 
    $\frac{c}{d}$ die des Bruches $\frac{c'}{d'}$ und sei hierbei $b, d>0$, also, wenn $x$ bzw. $y$ negativ ist, ist 
    es $a$ bzw. $c$, aber nie $b$ bzw. $d$. D.h. es ist $x = \frac{a}{b}$ und $y = \frac{c}{d}$ mit 
    $\gcd(a, b) = \gcd(c, d) = 1$ (wären die größten gemeinsamen Teiler größer als 1, wären die Brüche offensichtlich 
    nicht vollständig gekürzt). Nun ist wegen $x+y+z=0$ offenbar $z = -x-y = -\frac{a}{b} - \frac{c}{d} = -\frac{a 
    \cdot d}{b \cdot d} - \frac{c \cdot b}{d \cdot b} = - \frac{ad+cb}{db}$. Damit gilt dann wegen $x^2+y^2+z^2 = 100$:
    \begin{align*}
        100 = x^2+y^2+z^2 &=\left(\frac{a}{b}\right)^2+\left(\frac{c}{d}\right)^2+\left(-\frac{ad+cb}{db}\right)^2\\
        &= \left( \frac{a}{b} \right)^2 + \left( \frac{c}{d} \right)^2 + \left( \frac{ad+cb}{db} \right)^2\\
        &= \left( \frac{a \cdot d}{b \cdot d} \right)^2 + \left( \frac{c \cdot b}{d \cdot b} \right)^2 + \left( 
        \frac{ad+cb}{db} \right)^2\\
        &= \frac{(ad)^2}{(db)^2} + \frac{(cb)^2}{(db)^2} + \frac{(ad)^2 + 2(ad)(cb) + (cb)^2}{(db)^2}\\
        &= \frac{(ad)^2 + (cb)^2 + (ad)^2 + 2abcd + (cb)^2}{(db)^2}\\
        &= \frac{2 \left( (ad)^2 + abcd + (cb)^2 \right)}{(db)^2} \quad \quad \quad \| \cdot \frac{(db)^2}{2}\\
        \iff 100\cdot \frac{(db)^2}{2}=  50 \cdot (db)^2 &= (ad)^2 + abcd + (cb)^2 \numberthis \label{frac_manip}
    \end{align*}\\
    Sei nun $\gcd(b, d) = t$ und seien die dann natürlichen Zahlen $\frac{b}{t}$ und $\frac{d}{t}$ mit $p$ bzw. $q$ 
    bezeichnet (sie sind natürlich, da, weil $t$ der größte gemeinsame Teiler von $b$ und $d$ ist, natürlich $t$ von 
    sowohl $b$ als auch $d$ Teiler sein muss).
    \begin{lem}
        Die Zahlen $p$ und $q$ sind teilerfremd.
    \end{lem}
    \begin{proof}[Beweis des HS durch Widerspruch]
        \renewcommand{\qedsymbol}{$\square$}
        Sei $k$ ein gemeinsamer Teiler von $p$ und $q$. Sei ferner $k^e$ die größte Potenz von $k$, die $p$ und $q$ teilt. 
        Es ist also $k^e|p$ und $k^e|q$, aber gleichzeitig $k^{e+1}\nmid p$ oder $k^{e+1}\nmid q$. Sei nun $k^i$ die 
        größte Potenz von $k$, die $t$ teilt. Dann ist offensichtlich $k^{e+i}$ die größte Exponenz von $k$, die 
        $pt=b$ und $qt=d$ teilt. Es ist also $k^{e+i}|b$ und $k^{e+i}|d$. Damit ist dann aber offensichtlich auch 
        $k^{e+i}|\gcd(b, d)=t$. Gleichzeitig ist aber $k^{i}$ die größte Exponenz von $k$, die $t$ teilt! Es darf 
        also nicht $e+i>i$ sein, also muss v.a. $e=0$ gelten. Damit teilt aber $k$ nicht $p$ und $q$. Da die Annahme, 
        $p$ und $q$ hätten einen gemeinsamen Teiler, zu einem Widerspruch führt, muss sie falsch sein und damit müssen 
        $p$ und $q$ teilerfremd sein!
    \end{proof}
    Wenn man nun $b = pt$ und $d = qt$ in \eqref{frac_manip} einsetzt, erhält man:
    \begin{align*}
        50 \cdot (db)^2 &= (ad)^2 + abcd + (cb)^2\\
        = 50 \cdot q^2t^2 \cdot p^2 t^2 &= a^2 \cdot q^2 t^2 + ac \cdot pt \cdot qt + c^2 \cdot p^2t^2 \quad \| 
        \cdot \frac{1}{t^2}\\
        \iff 50 \cdot t^2 \cdot  q^2 p^2 &= a^2 q^2 + acpq +c^2p^2 \numberthis \label{divisib}
    \end{align*}\\
    Nun ist in \eqref{divisib} offensichtlich die linke Seite durch $p$ und durch $q$ teilbar, also muss es die 
    rechte Seite auch sein. Es muss also $p \mid a^2 q^2 + acpq + c^2 p^2$ und $q \mid a^2 q^2 + acpq + c^2 p^2$. 
    Aus $p \mid a^2 q^2 + acpq + c^2 p^2$ folgt wegen $p \mid acpq + c^2 p^2$ offenbar, dass $p \mid a^2 q^2$, während 
    aus $q \mid a^2 q^2 + acpq + c^2 p^2$ wegen $q \mid a^2 q^2 + acpq$ folgt, dass $q \mid c^2 p^2$. Es gilt also:
    \[
        p \mid a^2 q^2 \text{ und } q \mid c^2 p^2
    \]
    Da aber $p$ und $q$ teilerfremd sind, folgt direkt:
    \[
        p \mid a^2 \text{ und } q \mid c^2 \iff \frac{b}{t} \mid a^2 \text{ und } \frac{d}{t} \mid c^2
    \]
    Nun sind aber $a$ und $b$ teilerfremd, d.h. kein Teiler von $b$ (was $\frac{b}{t}$ ja ist) außer 1 kann $a$ bzw. 
    $a^2$ teilen. Es muss also $\frac{b}{t} = 1$ und damit $b = t$. Da auch $d$ und $c$ teilerfremd sind, folgt mit 
    gleicher Begründung, dass $d = t$. Damit ist dann $b = d$. Setzt man dies in (\refeq{frac_manip}) ein, so erhält 
    man:
    \begin{align*}
        50 \cdot b^2 \cdot b^2 &= a^2 b^2 + ac \cdot b^2 + c^2 b^2 \quad \| \cdot \frac{1}{b^2}\\
        \iff 50b^2 &= a^2 + ac + c^2 \numberthis \label{even_odd}
    \end{align*}
    Nun ist die linke Seite von \eqref{even_odd} offensichtlich gerade und damit muss es die rechte auch sein. Wenn 
    nun aber beide Zahlen $a, c$ ungerade sind, so sind alle drei Summanden ungerade (das Produkt zweier ungerader 
    Zahlen ist stets eine ungerade Zahl) und damit dann die gesamte rechte Seite ungerade. Es muss also mindestens 
    eine der beiden Zahlen gerade sein. Wenn nun aber nur genau eine gerade ist, so ist auch das Quadrat dieser und 
    $ac$ gerade, jedoch nicht das Quadrat der anderen dann ungeraden Zahl. Die rechte Seite wäre die Summe zweier 
    gerader und  einer ungeraden Zahl, also auch ungerade. Da nicht beide und nicht genau eine der beiden Zahlen 
    ungerade sein können, müssen beide gerade sein. D.h. es gibt zwei Zahlen $r, s \in \mathbb{Z}$, für die $a = 2r$ und 
    $c = 2s$ ist. Setzt man dies in \eqref{even_odd} ein, so erhält man:
    \begin{align*}
        50 b^2 &= (2r)^2 + (2r)(2s) + (2s)^2 = 4 \left( r^2 + rs + s^2 \right) \quad \| \cdot \frac12\\
        \iff 25b^2 &= 2 \left( r^2 + rs + s^2 \right)
    \end{align*}
    Da in dieser Gleichung die rechte Seite offensichtlich gerade ist, muss auch die linke, also $25b^2$, gerade sein. 
    Da aber 25 nicht gerade ist, muss $b^2$ gerade sein, wozu offensichtlich $b$ gerade sein muss. Nun sind also 
    sowohl $a$ also auch $b$ gerade, d.h. durch 2 teilbar. $a$ und $b$ haben also den gemeinsamen Teiler 2. Dies 
    stellt aber einen Widerspruch zu $\gcd(a, b) = 1$ dar! Denn offensichtlich ist ihr größter gemeinsamer Teiler 
    nicht 1, wenn 2 ein gemeinsamer Teiler von $a$ und $b$ ist. Da die Annahme, es gäbe drei Zahlen $x, y, z\in \Q$, 
    für die die beiden Gleichungen von Satz \ref{aufgabe_2} gelten, zu einem Widerspruch führt, muss sie falsch sein, 
    was dann den Widerspruchsbeweis von Satz \ref{aufgabe_2} vervollständigt.
\end{proof}