\fancyhead[L]{\leftmark}

\section{Aufgabe 1}

\subsection*{Hinweise zur Notation}

$\N_0$ ist die Menge der natürlichen Zahlen mit der Null, also $\{ 0, 1, 2, 3, \ldots \}$.

$\text{[g]}$ heißt, dass die Einheit, in der gerechnet wurde, Gramm ist, dies aber ggf. in Zwischenrechnungen nicht 
angegeben wurde. Es sollte stets klar sein, in welcher Einheit gerechnet wird, da die Masse die einzige physikalische 
Größe ist, von der Gebrauch gemacht wird.

"`HS"' ist kurz für (die entsprechend deklinierte Form von) "`Hilfssatz"'.

\subsection*{Kurze Antwort}

Smilla kann mindestens 1.021.615g Gold für sich garantieren.

\subsection*{Beweis (d.h. die etwas längere Antwort)}

Zu zeigen ist also der folgende Satz:
\begin{thm}
    Die größte Masse an Gold, die Smilla mit Sicherheit mindestens gewinnen kann, ist 1.021.615g.
\end{thm}

\begin{proof}[Beweis des Satzes]
    Der Beweis besteht aus zwei Teilen:
    \begin{enumerate}
        \item Dem Beweis, dass Smilla stets 1.021.615g Gold gewinnen kann
        \item Dem Beweis, dass es keine größere Masse Gold gibt, die Smilla immer gewinnen kann
    \end{enumerate}
    \renewcommand{\qedsymbol}{$\square$}
    \begin{lem}\label{smilla}
        Smilla kann stets versichern, dass sie mindestens 1.021.615g Gold gewinnt.
    \end{lem}
    \begin{proof}[Beweis des HS]
        Um mindestens 1.021.615g Gold zu gewinnen, kann Smilla folgenden vierschrittigen Plan verfolgen (wobei die 
        Schritte 2 und 3 ggf. mehrfach eintreten):
        \begin{enumerate}
            \item Die 2020 Gold-Nuggets auf drei Haufen aufteilen, und zwar nach folgendem Schema:
            \begin{enumerate}
                \item Auf Haufen 1 kommt jedes Gold-Nugget, dessen Masse in Gramm eine der Formen $4k+1$ und $4k+2$ 
                annimmt (für ein $k \in \N_0$). Das sind also die Nuggets mit den Massen 1g ($1 = 4 \cdot 0+1$), 2g 
                ($2 = 4\cdot 0+2$), 5g ($5 = 4 \cdot 1+1$), 6g ($6=4\cdot 1+2$), \dots, 2017g ($2017=4\cdot 504+1$), 
                2018g ($2018=4\cdot 504+2$).
                \item Auf Haufen 2 kommt jedes Gold-Nugget, dessen Masse in Gramm eine der Formen $4k$ und $4k+3$ 
                annimmt (für ein $k \in \N_0$) und dessen Masse nicht 2020g ist. Das sind also die Nuggets mit den 
                Massen 3g ($3=4\cdot 0+3$), 4g ($4=4\cdot 1$), 7g ($7=4\cdot 1+3$), 8g ($8=4\cdot 2$), \dots, 2016g 
                ($2016=4\cdot 504$), 2019g ($2019=4\cdot 504+3$).
                \item Auf "`Haufen"' 3 kommt das Nugget mit der Masse 2020g.
            \end{enumerate}
            \item Wenn Leo die rote Truhe wählt, das leichteste noch verfügbare Nugget von Haufen 1 hineinlegen (das 
            "`leichteste"' heißt das mit der geringsten Masse). Falls von Haufen 1 kein Nugget mehr verfügbar ist, das 
            Nugget mit der Masse 2020g hineinlegen und falls auch das nicht mehr verfügbar ist, das leichteste noch 
            verfügbare Nugget (das dann offensichtlich von Haufen 2 kommen muss) hineinlegen.
            \item Wenn Leo die blaue Truhe wählt, das leichteste noch verfügbare Nugget von Haufen  2 hineinlegen. Falls 
            von Haufen 2 kein Nugget mehr verfügbar ist, das Nugget mit der Masse 2020g hineinlegen und falls auch das 
            nicht mehr verfügbar ist, das leichteste noch verfügbare Nugget (das dann offensichtlich von Haufen 1 
            sein muss) hineinlegen.
            \item Wenn alle Nuggets verteilt sind, die Truhe wählen, in der sich das Nugget mit der Masse 2020g 
            befindet.

        \end{enumerate}

        Anm.: Im Folgenden werde ich Haufen 1, Haufen 2 und Haufen 3 mit $H_1$ bzw. $H_2$ bzw. $H_3$ bezeichnen.

        Zunächst muss natürlich (kurz) bewiesen werden, dass die Aufteilung auf die drei Haufen so, wie sie in 
        Schritt 1 beschrieben wurde, überhaupt möglich ist. D.h., dass jedes Nugget auf genau einen Haufen kommt. 
        Oder: Dass jedes Nugget auf einen Haufen kommt und kein Nugget auf zwei Haufen kommt. Offensichtlich kann 
        jedes Nugget nur einen der Reste 0, 1, 2, 3 haben, wenn man seine Masse in g durch 4 teilt. \\
        Betrachte man zunächst die mit Rest 0: Sie kommen alle bis auf das mit der Masse 2020g auf $H_2$ und auf 
        keinen anderen (nicht auf $H_1$, da sie weder einen Rest von 1 noch einen von 2 haben, wenn man ihre 
        Masse in g durch 4 teilt; nicht auf $H_3$, da auf diesen nur das Nugget mit der Masse 2020g kommt, 
        welches ja in diesem Fall ausgeschlossen wurde). Das mit der Masse 2020g kommt auf $H_3$ und 
        offensichtlich auch nur auf $H_3$. \\
        Betrachte man nun die mit Rest 1: Sie kommen ausnahmslos auf $H_1$ und auf keinen anderen (nicht auf $H_2$, da sie 
        weder einen Rest von 0 noch von 3 haben, wenn man ihre Massen in g durch 4 teilt; nicht auf $H_3$, da auf 
        diesen nur das Nugget mit der Masse 2020g kommt). \\
        Betrachte man nun die mit Rest 2: Sie kommen alle auf $H_1$ und auf keinen anderen (nicht auf $H_2$, da sie weder 
        einen Rest von 0 noch von 3 haben; nicht auf $H_3$, da auf diesen nur das Nugget mit der Masse 2020g kommt). \\
        Und zu guter Letzt betrachte man nun die mit Rest 3: Sie kommen alle auf $H_2$ (nicht auf $H_1$, da sie weder 
        einen Rest von 1 noch von 2 haben; nicht auf $H_3$, da auf diesen nur das Nugget mit der Masse 2020g kommt).

        Im ersten Schritt wird Smilla also die 2020 Nuggets auf drei Haufen aufteilen, wobei jedes Nugget auf genau 
        einen Haufen kommt.
        
        Ähnlich zu dem Beweis, dass die Aufteilung tatsächlich möglich ist (also, dass jedes Nugget auf genau 
        einen Haufen kommt), werde ich hier noch beweisen, dass bei dem Verteilen der Nuggets auf die 
        Truhen nach dem in den Schritten 2 und 3 beschriebenen Schema stets eindeutig ist, welches Nugget Smilla in 
        die von Leo gewählte Truhe legen muss, und, dass dieses Nugget auch stets zur Verfügung steht.

        Dies ist jedoch ziemlich offensichtlich. Denn man kann sich die Wahl auch folgendermaßen vorstellen:\\
        Wenn Leo die rote Truhe wählt, geht Smilla im Kopf alle Nuggets (auch die, die schon gelegt wurden) durch, 
        allerdings nicht in irgendeiner Reihenfolge, sondern zunächst vom Leichtesten zum Schwersten von $H_1$, 
        dann das mit der Masse 2020g, dann vom Leichtesten zum Schwersten von $H_2$. Dabei nimmt sie dann das Erste, 
        das ihr noch zur Verfügung steht, und legt es in die rote Truhe. \\
        Wenn Leo die blaue Truhe wählt, geht sie auch alle Nuggets durch, allerdings zuerst vom Leichtesten zum 
        Schwersten von $H_2$, dann das mit der Masse 2020g, dann vom Leichtesten zum Schwersten von $H_1$.

        In beiden Fällen wird, da es keine zwei Nuggets mit der gleichen Masse gibt, eindeutig bestimmt sein, welches 
        Nugget Smilla in die von Leo gewählte Truhe legen muss.

        Es bleibt also zu beweisen, dass, wenn alle Nuggets verteilt sind, die Truhe, in der das Nugget mit der Masse 
        2020g liegt, Nuggets mit einer Gesamtmasse von mindestens 1.021.615g beinhaltet. Denn dann wird Smilla gemäß 
        Schritt 4 diese Truhe wählen und mindestens 1.021.615g Gold gewinnen.

        Zunächst berechne man hierzu folgende Werte: die Masse an Gold, die auf $H_1$ bzw. $H_2$ (bzw. $H_3$) liegt, 
        welche ich im Folgenden mit $m(H_1)$ bzw. $m(H_2)$ bzw. $m(H_3)$ bezeichnen werde.

        Offensichtlich befindet sich auf $H_3$ genau ein Nugget (das mit der Masse 2020g). Die Gesamtmasse der 
        Nuggets auf $H_3$ besteht also nur aus dem Nugget mit der Masse 2020g, also ist $m(H_3) = 2020 [\text{g}]$ 
        (von hier an werde ich bei $m(H_1), m(H_2), m(H_3)$ die Einheit g nicht mehr dazuschreiben). Betrachte man 
        nun die noch übrigen Nuggets, also die mit Massen von höchstens 2019g. Diese lassen sich folgendermaßen 
        schreiben (wobei ich "`das Nugget mit der Masse"' und die Einheit Gramm weggelassen habe): 
        $4\cdot 0+1, 4\cdot 0+2, 4\cdot 0+3, 4\cdot 1+0, 4\cdot 1+1, 4\cdot 1+2, 4\cdot 1+3, 4\cdot 2+0, 4\cdot 2+1, 
        \ldots, 4\cdot 504+0\hspace{5pt} (=2016), 4\cdot 504+1, 4\cdot 504+2, 4\cdot 504+3$. Offensichtlich sind 
        genau 505 von diesen von der Form $4k +1$ (für ein $k \in \N_0$), nämlich $4\cdot 0+1, 4\cdot 1+1, 4\cdot 2+1, 
        \ldots, 4\cdot 503+1, 4\cdot 504+1$. Gleichzeitig sind 505 von der Form $4k+2$ (für ein $k\in\N_0$), nämlich 
        $4\cdot 0+2, 4\cdot 1+2, 4\cdot 2+2, 4\cdot 3+2, \ldots, 4\cdot 503+2, 4\cdot 504+2$. Es gibt unter den 2019 
        Nuggets also 505 mit Massen (in g) der Form $4k+1$ und 505 mit Massen (in g) der Form $4k+2$. Nun sind dies 
        aber genau die Nuggets, die auf $H_1$ kommen.
        Also sind die 1010 Nuggets, die sich auf $H_1$ befinden, die mit folgenden Massen (in g): $4\cdot 0+1, 
        4\cdot 0+2, 4\cdot 1+1, 4\cdot 1+2, 4\cdot 2+1, 4\cdot 2+2, \ldots, 4\cdot 503+1, 4\cdot 503+2, 4\cdot 504+1, 
        4\cdot 504+2$. Die Gesamtmasse dieser Nuggets, also $m(H_1)$, ist also:
        \begin{align*}
            &m(H_1)\\
            =&(4\cdot 0+1)+(4\cdot 0+2)+(4\cdot 1+1)+(4\cdot 1+2)+\cdots+(4\cdot 504+1)+(4\cdot 504+2)\\
            = &(4\cdot 0+1)+(4\cdot1+1)+\cdots+(4\cdot 503+1)+(4\cdot 504+1)\\
            &+(4\cdot 0+2)+(4\cdot1+2)+\cdots+(4\cdot 503+2)+(4\cdot 504+2)\\
            = &(4\cdot0+4\cdot1+\cdots+4\cdot503+4\cdot504)+(1+1+\cdots+1)\\
            &+(4\cdot0+4\cdot1+\cdots+4\cdot503+4\cdot504)+(2+2+\cdots+2)\\
        \end{align*}
        Da $4\cdot0+1, 4\cdot1+1, \ldots, 4\cdot504+1$ insgesamt 505 Summanden sind, handelt sich bei der Summe
        $(1+1+\cdots+1)$ in der zweitletzten Zeile um 505 Einer (aus jedem der Summanden $(4k+1)$ für ein $k\in \N_0$ 
        wurde die Eins genommen). Da auch $4\cdot0+2,5\cdot1+2,\ldots,4\cdot504+2$ insgesamt 505 Summanden sind, handelt es sich 
        mit gleicher Begründung bei der Summe $(2+2+\cdots+2)$ in der letzten Zeile um 505 Zweier. Es ist also:
        \begin{align*}
            &m(H_1)\\
            = &4\cdot(0+1+2+3+\cdots+504)+(505\cdot1)\\
            &+4\cdot(0+1+2+3+\cdots+504)+(505\cdot2)\\
            = &4\cdot\left(\frac{504\cdot(504+1)}{2}\right)+505
            +4\cdot\left(\frac{504\cdot(504+1)}{2}\right)+2\cdot 505\\
            =&2\cdot504\cdot505+505+2\cdot 504\cdot505+2\cdot505\\
            =&1008\cdot505+505+1008\cdot505+2\cdot505=(1008+1+1008+2)\cdot 505\\
            =&(2016+3)\cdot 505=2019\cdot505= 1.019.595
        \end{align*}
        (Berechnung von $2019\cdot 505$ ohne Taschenrechner\footnote{) $2019\cdot 505= 2019\cdot 1010 \cdot \frac12
        =(2019\cdot 1000+2019\cdot 10)\cdot \frac12=\frac12(2.019.000+20.190)=1.009.500+10.095=1.019.595$})\\
        Anm.: Zur Berechnung von $0+1+2+3+\cdots+504$ habe ich die aus der Schule bekannte Gaußsche Summenformel 
        verwendet, nach welcher $1+2+3+\cdots+(n-1)+n=\frac{n(n+1)}{2}$, also hier (mit $n=504$): 
        $0+1+2+3+\cdots+504=1+2+3+\cdots+504=\frac{504\cdot(504+1)}{2}$.

        Es ist also $m(H_1)= 1.019.595$ und $m(H_3)=2020$. Gleichzeitig ist aber, da jedes der Nuggets auf einem der 
        drei Haufen, allerdings keines auf zweien landet, die Gesamtmasse der drei Haufen gleich der Gesamtmasse 
        aller Nuggets; d.h.: $m(H_1)+m(H_2)+m(H_3)=1+2+3+\cdots+2019+2020=\frac{2020\cdot2021}{2}=2.041.210$ (nach der 
        Gaußschen Summenformel mit $n=2020$; Berechnung ohne Taschenrechner\footnote{) $\frac{2020\cdot2021}{2}=1010
        \cdot2021=1000\cdot2021+10\cdot2021=2.021.000+20.210=2.041.210$}). Es ist also $m(H_2)=2.041.210-m(H_1)-m(H_2) 
        = 2.041.210-1.019.595-2020=1.019.595$ (Überprüfung der Berechnung ohne Taschenrechner\footnote{) $2.041.210
        -1.019.595-2020=1.019.595\Leftrightarrow 2.041.210-2020=2\cdot 1.019.595=2.039.190\Leftrightarrow 2.041.210
        =2.039.190+2020=2.039.210+2000=2.041.210$})

        Die berechneten Werte sind also: $m(H_1)=1.019.595, m(H_2)=1.019.595, m(H_3)=2020$.

        Nun ist in den Regeln des Spiels festgelegt, dass erst aufgehört wird, wenn alle Nuggets in einer der beiden 
        Truhen liegen. Also muss auch das Nugget mit der Masse 2020g in einer der beiden Truhen liegen. Um nicht zu 
        allgemein formulieren zu müssen (was vermutlich unverständlicher wäre), kann man die beiden Fälle einzeln 
        betrachten:

        \textit{Fall 1: Das Nugget mit der Masse 2020g landet in der roten Truhe.} Offensichtlich können dann zu dem 
        Zeitpunkt, zu dem Smilla das Nugget mit der Masse 2020g in die rote Truhe legt, noch keine Nuggets von $H_1$ in 
        der blauen Truhe liegen. Denn wann immer sie eines hätte legen können (also wann immer Leo die blaue Truhe 
        gewählt hat), hätte sie auch noch das mit der Masse 2020g legen können (da sie es jetzt zur Verfügung hat, 
        hätte sie es zu diesem früheren Zeitpunkt auch zur Verfügung gehabt), was sie dann gemäß Schritt 3 zunächst 
        hätte tun müssen. Es liegt also kein Nugget von $H_1$ in der blauen Truhe. Da sie aber auch kein Nugget von 
        $H_1$ mehr zur Verfügung hat (sonst müsste sie das zuerst legen, wenn Leo die rote Truhe wählt), müssen alle 
        Nuggets von $H_1$ bereits in einer der beiden Truhen liegen. Weil diese Truhe nicht die blaue sein kann (in 
        ihr liegen ja keine Nuggets von $H_1$), müssen also alle Nuggets von $H_1$ in der roten Truhe liegen.

        \textit{Fall 2: Das Nugget mit der Masse 2020g landet in der blauen Truhe.} Offensichtlich können dann zu dem 
        Zeitpunkt, zu dem sie das Nugget mit der Masse 2020g in die blaue Truhe legt, noch keine Nuggets von $H_2$ in 
        der roten Truhe liegen. Denn wann immer sie eines hätte legen können (also wann immer Leo die rote Truhe 
        gewählt hat), hätte sie auch noch das mit der Masse 2020g legen können (da sie es jetzt zur Verfügung hat, 
        hätte sie es zu diesem früheren Zeitpunkt auch zur Verfügung gehabt), was sie dann gemäß Schritt 2 zunächst 
        hätte tun müssen. Es liegt also kein Nugget von $H_2$ in der roten Truhe. Da sie aber auch kein Nugget von 
        $H_2$ mehr zur Verfügung hat (sonst müsste sie das zuerst legen, wenn Leo die blaue Truhe wählt), müssen alle 
        Nuggets von $H_2$ bereits in einer der beiden Truhen liegen. Weil diese Truhe nicht die rote sein kann (in 
        ihr liegen ja keine Nuggets von $H_2$), müssen also alle Nuggets von $H_2$ in der blauen Truhe liegen.

        Wenn das Nugget mit der Masse 2020g in der roten Truhe landet, müssen in ihr also auch alle Nuggets von $H_1$ 
        sein, und wenn es in der blauen Truhe liegt, müssen in dieser auch alle Nuggets von $H_2$ liegen. In erstem 
        Fall hat also die rote Truhe eine Gesamtmasse von mindestens $m(H_1)+2020=1.021.615 [\text{g}]$, da in ihr 
        alle Nuggets von $H_1$ (die eine Gesamtmasse von $m(H_1)$ haben) und das mit der Masse 2020g liegen. In 
        letzterem Fall muss die blaue Truhe eine Gesamtmasse von mindestens $m(H_2)+2020=1.021.615 [\text{g}]$ haben,
        da in ihr alle Nuggets von $H_2$ (die eine Gesamtmasse von $m(H_2)$ haben) und das mit der Masse 2020g 
        liegen. (Berechnungen von $m(H_1)+2020$ und $m(H_2)+2020$ ohne Taschenrechner\footnote{) $m(H_1)+2020=
        1.019.595+2020=1.019.615+2000=1.021.615$ und $m(H_2)+2020=1.019.595+2020=m(H_1)+2020=1.021.615$})

        In beiden Fällen beinhaltet also die Truhe, in der das Nugget mir der Masse 2020g liegt, Gold mit einer 
        Gesamtmasse von mindestens 1.021.615g. Wenn Smilla dann gemäß Schritt 4 diese Truhe, in der das Nugget mit 
        der Masse 2020g liegt, wählt, wählt sie stets eine mit mindestens 1.021.615g Gold.

        Smilla kann also stets mit Sicherheit mindestens 1.021.615g Gold gewinnen!
    \end{proof}
    \begin{lem}\label{leo}
        Leo kann stets verhindern, dass Smilla mehr als 1.021.615g Gold bekommt.
    \end{lem}
    \begin{proof}[Beweis des HS]
        Dieser Beweis ist recht simpel, da Leo, um zu verhindern, dass Smilla mehr als 1.021.615g Gold bekommt, 
        einfach folgenden zweischrittigen Plan verfolgen kann:
        \begin{enumerate}
            \item Solange die blaue Truhe wählen, bis in ihr insgesamt mehr als 1.019.595g Gold sind.
            \item Sobald in der blauen Truhe mehr als 1.019.595g Gold sind, nur noch die rote Truhe wählen.
        \end{enumerate}
        Am Ende sind damit in der blauen Truhe auf jeden Fall mindestens 1.019.595g Gold. Wie im Beweis von HS 
        \ref{smilla} berechnet, ist die Gesamtmasse der 2020 Nuggets 2.041.210g. Wenn also in der blauen Truhe 
        mindestens 1.019.595g Gold sind, können in der roten Truhe nur noch höchstens $2.041.210\text{g}-1.019.595
        \text{g}=1.021.615\text{g}$ Gold sein(Überprüfung der Berechnung ohne Taschenrechner\footnote{) $2.041.210-
        1.019.595=1.021.615\Leftrightarrow 2.041.210=1.021.615+1.019.595=2.021.615+19.595=2.031.615+9.595=2.040.615
        +595=2.041.115+95=2.041.210$}).

        Offensichtlich kann Smilla in einem Zug nur maximal 2020g Gold  in eine der beiden Truhen legen. Wenn nun 
        irgendwann mehr als 1.021.615g Gold in der blauen Truhe wären, muss es einen Zug geben, in dem Smilla das 
        Nugget in die Truhe legt, das dafür sorgt, dass die Gesamtmasse der in der blauen Truhe befindlichen Nuggets 
        größer als 1.021.615g ist. Dieses Nugget kann höchstens 2020g wiegen. Bevor sie das Nugget hineinlegt, 
        müssten sich also schon Nuggets in der Truhe befinden, die eine Gesamtmasse von mehr als $1.021.615\text{g}-
        2020\text{g}=1.019.595\text{g}$ haben (Überprüfung ohne Taschenrechner\footnote{) $1.021.615-2020=
        1.019.5959 \Leftrightarrow 1.021.615=1.019.595+2020=m(H_1)+2020$, s. Fußnote 5}), 
        denn sonst würden die zusätzlichen maximal 2020g nicht genügen, um die Gesamtmasse über 1.021.615g zu 
        bringen. Wenn sich aber bereits mehr als 1.019.595g Gold in der blauen Truhe befunden hätten, hätte Leo -- 
        gemäß seinem Plan -- die rote Truhe gewählt; Smilla hätte also gar kein Nugget mehr in die blaue Truhe 
        legen können! Wenn Leo seinen simplen Plan verfolgt, kann es also nicht passieren, dass sich in der blauen 
        Truhe mehr als 1.021.615g Gold befinden. 

        Es können sich also weder in der blauen noch in der roten Truhe mehr als 1.021.615g Gold befinden. Demnach 
        kann Smilla natürlich auch keine Truhe wählen, in der sich mehr als 1.021.615g Gold befinden. Leo kann also 
        stets verhindern, dass Smilla mehr als 1.021.615g Gold bekommt!
    \end{proof}
    \renewcommand{\qedsymbol}{$\blacksquare$}
    Smilla kann also nach HS \ref{smilla} stets 1.021.615g Gold gewinnen. Allerdings kann sie keine größere Masse 
    Gold mit Sicherheit gewinnen, denn das kann Leo nach HS \ref{leo} stets verhindern. Die gröte Masse an Gold, 
    die Smilla mit Sicherheit gewinnen kann, ist also 1.021.615g.
\end{proof}