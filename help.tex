\section*{Benutzte Hilfsmittel}
\addcontentsline{toc}{section}{Benutzte Hilfsmittel}

\subsection*{Für Aufgabe 1 benutzte Hilfsmittel}

Für Aufgabe 1 habe ich regelmäßig einen Taschenrechner verwendet. Allerdings habe ich in der endgültigen Fassung 
darauf geachtet, dass alle Berechnungen auch ohne einen Taschenrechner nachvollziehbar sind. In der Regel habe 
ich hierzu Fußnoten verwendet. Dann steht in Klammern "`Berechnung ohne Taschenrechner"' o.Ä. und am Ende ein 
Superskript mit einer Zahl. Unten auf der Seite sind dann die Inhalte der Fußnoten mit  den entsprechenden Zahlen 
zu finden (Beispiel\footnote{) Hier steht dann die Berechnung oder Berechnungsüberprüfung, die auch ohne einen 
Taschenrechner nachvollziehbar sein sollte.}).

% ? Muss das Beispiel sein?

\subsection*{Für Aufgabe 2 benutzte Hilfsmittel}

Für sowohl die Findung als auch die Ausarbeitung der Lösung von Aufgabe 2 habe ich keinerlei Hilfsmittel benutzt.

\subsection*{Für Aufgabe 3 benutzte Hilfsmittel}

Für sowohl die Findung als auch die Ausarbeitung der Lösung von Aufgabe 3 habe ich keinerlei Hilfsmittel benutzt.

\subsection*{Für Aufgabe 4 benutzte Hilfsmittel}

Für sowohl die Findung als auch die Ausarbeitung der Lösung von Aufgabe 4 habe ich keinerlei Hilfsmittel benutzt.